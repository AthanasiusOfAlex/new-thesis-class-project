% !TEX TS-program = xelatex
% !TEX encoding = UTF-8 Unicode
% !BIB TS-program = biber

\chapter{\emph{Resolutio} of \emph{Bonum} to \emph{Esse}}
\label{chap:bonum}

In order to begin our inquiry, we will examine in detail what is meant by ``good'' (\emph{bonum}) and how it is founded. After a brief historical overview of the notion of \emph{bonum} (especially as a ``transcendental''), we will resolve the good into its intrinsic causes. The \emph{resolutio} in this chapter will enable us to make a ``noetical'' foundation, in which we will discover what we can learn about \emph{bonum} based on its very notion. In order to do this, we will proceed first \emph{in via inventionis}, beginning with the fundamental and most well known notion of \emph{bonum}, which is desirability or ``appetibility,'' and finishing with \emph{esse} as the underlying principle. Eventually, we will resolve \emph{esse} itself into its radical principle, \emph{esse ut actus} as measured by essence, but that analysis will need to wait until we have discussed the intrinsic composition of \emph{ens} in greater detail (\autoref{chap:ens_resolutio}). Then, \emph{in via iudicii}, we will perform a \emph{compositio} so as to draw a sketch of how \emph{esse} founds \emph{bonum}, leaving the definitive ``ontological'' foundation for \autoref{chap:dynamism}.%
%
\footnote{\emph{Resolutio} and \emph{compositio} refer to methods proper to metaphysics. The former, which roughly means ``analysis,'' signifies investigation into the causes of the object of metaphysics (\emph{ens}), starting with the ``data;'' that is, the incontrovertible ``givens'' that are accessible to everyone. In its etymological sense, \emph{resolutio} means the ``decomposition'' of a whole into its parts, although, borrowing from the Greek root from which it comes (\gk{ἀναλύω}), it can also mean more generally a rising up (\gk{ἀνά}) to the ultimate causes, whether they are intrinsic or extrinsic.  \emph{Compositio} indicates the opposite process, the joining together of parts into a whole, or going from the causes to their effects.

For an overview of \emph{resolutio} as understood by Thomas Aquinas and interpreted by C. Fabro, see \cite{villagrasa:resolutio}, especially 57--62; also, \cite[339--359]{mitchell:resolutio}. Aquinas himself describes \emph{resolutio} and \emph{compositio} in \cite[q.~6, a.~1]{st:detrinitate}, especially in the \emph{responsum} to the third question: ``Ratio enim, ut prius dictum est, procedit quandoque de uno in aliud secundum rem, ut quando est demonstratio per causas vel effectus extrinsecos: componendo quidem, cum proceditur a causis ad effectus; quasi resolvendo, cum proceditur ab effectibus ad causas, eo quod causae sunt effectibus simpliciores et magis immobiliter et uniformiter permanentes. Ultimus ergo terminus resolutionis in hac via est, cum pervenitur ad causas supremas maxime simplices, quae sunt substantiae separatae. Quandoque vero procedit de uno in aliud secundum rationem, ut quando est processus secundum causas intrinsecas: componendo quidem, quando a formis maxime universalibus in magis particulata proceditur; resolvendo autem quando e converso, eo quod universalius est simplicius.''}
%

\section{Historical Background of \emph{Bonum}}
\subsection{Plato, Plotinus, and Aristotle}

The attempt to discover characteristics common to all reality without exception---what Saint Thomas Aquinas calls the \emph{transcendentia} and later Scholastics the \emph{transcendentalia} or \emph{passiones entis}---dates at least as far back as Plato, who held that reality consists, not so much in the sensible world, but in the separated ideas (\gk{τὰ εἴδη}) in which sensible things participate.%
%
\footnote{Probably the best exposition of his theory is to be found in the \emph{Phaedo}. See especially \cite[78c--79a, and 100b--101d]{plato:phaedo}. The principal ideas contributed by Plato, Aristotle and Pseudo-Dionysius regarding \emph{bonum} are well documented in \cite[290-298]{aertsen:transcendentals}.

Regarding the term \emph{participation}: participation entails an active element (whatever has the perfection \emph{proprie et per se}) and a passive one (whatever receives the perfection from the former). Latin and Romance languages can render both aspects using the terms \emph{participo} and \emph{participatio}; however, in English usage, the verb \emph{to particpate} is strictly intransitive. Therefore, to designate active participation, I will use \emph{to communicate}, and for passive participation, \emph{to participate in} or \emph{to take part in}. Regarding participation, see also \cite[453--456]{fabro:intensive}.}
%
In the \emph{Sophist}, he posits not only ideas that unify sensible things, but also what could be termed \enquote{meta-ideas,} which unify the ideas themselves: movement, rest, being (\gk{τὸ ὄν}), \enquote{same,} and \enquote{different.} These, in turn, are said to participate in the Idea of the Good and---supposing that Plato had \enquote{esoteric} teachings that are not directly referred to in his dialogues---ultimately the Principles of the One and the Dyad.%
%
\footnote{We will discuss the doctrine and development of the meta-ideas below, in section \ref{esse-presocratics-to-aristotle}. The Idea of the Good is developed above all in \cite[VI, 508c--509a]{plato:republic}. 

It is a matter of debate whether or not Plato had unwritten doctrines, in particular ones that resolve all of the ideas to ``principles'' that transcend the world of ideas. Aristotle reports that Plato reduced all of the ideas (hence all of reality) to the One and the Dyad. \parencite[See][Α, 6, 987b19-988a7.]{aristotle:metaphysics} In general the school of Tübingen (represented by G. Reale and  H.J. Krämer) has held that Plato did have esoteric doctrines. For an exposition of this position, see \cite{kramer:platone}, and \cite{reale:nuova}.

On the other hand, F. Copleston, for example, sustains that the One should not be distinguished from the Idea of the Good. See \cite[177]{copleston:history:01} ``Hence it would seem only reasonable to conclude that the One, the Good and the essential Beauty are the same for Plato, and that the intelligible world of Forms owes its being in some way to the One.''}
%
Although it is debatable whether or not Plato intended to identify the One and the Good, many of his followers did so, notably Plotinus and other Neoplatonists, who explicitly posited the Good or One as the supreme transcendent cause of all reality.%
%
\footnote{Plotinus specifies that the principle of the Intellect (\gk{Νοῦς}) is the Good (\gk{τὸ ἀγαθόν or simply τἀγαθόν}), for example, in \cite[V,~1,~8.]{plotinus:enneads}. This same principle, however, is clearly identified as the One (\gk{τὸ ἕν}) in \cite[V,~1,~1]{plotinus:enneads}. See also \cite[VI,~7--9]{plotinus:enneads}.}
%
Since, according to Plotinus, all things participate in their supreme principle, it follows that they are all in some measure good.%
%
\footnote{How \emph{entia} participate in the Good is discussed in \cite[I, 7]{plotinus:enneads}.}
%
For Plato and his followers, however, properties such as goodness are generally considered extrinsic to the things that possess them---that is, the \enquote{goodness} as such is to be found principally in the Principle, not so much in the entity that participates in it---and for this reason, the notion of goodness (and of any other \enquote{transcendental}) is necessarily univocal. In other words, even though it is evident that different things have different degrees of participated goodness, in reality the only thing that can really be said to be truly \enquote{good} is the Good itself. The Good as such, therefore, remains the same, even though the ``amount'' goodness found in, or communicated to, each concrete reality is different. Hence, goodness is differentiated extrinsically, not intrinsically. Nevertheless, for Plato, the Good is not simply a ``property'' of things, but rather is ``generous,'' ``fecund'' and ``expansive,'' an idea taken up by the Neoplatonists and through them by Thomas Aquinas.%
%
\footnote{See \cite[62]{definance:etre-et-agir}: ``Et, pour Platon, le Bien, s'il dit proportion et mesure, dit aussi expansion généreuse. Le démiurge du Timée a organisé le monde parce qu'« il était bon, et de ce qui est bon, nulle envie ne naît jamais à nul sujet. Exempt d'envie, il a voulu que toutes choses naquissent, le plus possible, semblables à lui ». The author is quoting from \emph{Timaeus}, 29e: ``\gk{ἀγαθὸς ἦν, ἀγαθῷ δὲ οὐδεὶς περὶ οὐδενὸς οὐδέποτε ἐγγίγνεται φθόνος· τούτου δ΄ ἐκτὸς ὢν πάντα ὅτι μάλιστα ἐβουλήθη γενέσθαι παραπλήσια ἑαυτῷ}'' \parencite{plato:timaeus}.}
%

Aristotle, although a disciple of Plato, rejected his teacher's theory of separated ideas, giving his primary emphasis to concrete realities, including sensible ones; that is, the First Science for him was the science of \emph{ens qua ens} (\gk{τὸ ὂν ᾗ ὂν}). For Aristotle, \gk{τὸ ὄν} refers primarily to substance (\gk{οὐσία}), and secondarily to the \gk{ὄντα} that inhere in substance (what we would call accidents).%
%
\footnote{The passage from \emph{Metaphysics}, Γ, 2, in which he reduces the manifold meanings of \gk{τὸ ὄν} to \gk{οὐσία}, is well known, as is his refutation of the Platonic \enquote{universal} (\gk{καθόλου}), or idea, in Ζ, 13. See \cite[Γ, 2, 1003a33-1003b4, and Ζ, 13, 1038b35-1039a3]{aristotle:metaphysics}.}
%
Nevertheless, Aristotle affirms that \gk{τὸ ὄν} has certain coextensive properties. He is most clear regarding oneness (\gk{τὸ ἕν}): in \emph{Metaphysics}, \gk{Γ}, 2, for example he says,
%
\begin{greekQuotation}
εἰ δὴ τὸ ὂν καὶ τὸ ἓν ταὐτὸν καὶ μία φύσις [\ldots] καὶ οὐδὲν ἕτερον τὸ ἓν παρὰ τὸ ὄν, ἔτι δ᾽ ἡ ἑκάστου οὐσία ἕν ἐστιν οὐ κατὰ συμβεβηκός, ὁμοίως δὲ καὶ ὅπερ ὄν τι: ὥσθ᾽ ὅσα περ τοῦ ἑνὸς εἴδη, τοσαῦτα καὶ τοῦ ὄντος.%
%
\footnote{\Cite[1003b23, 33-34]{aristotle:metaphysics}: \enquote{Now if Being and Unity are the same, i.e. a single nature, [\ldots] and Unity is nothing distinct from Being; and further if the substance of each thing is one in no accidental sense, and similarly is of its very nature something which is---then there are just as many species of Being as of Unity} \parencite[translation from][]{aristotle:metaphysics:en}.}
%
\end{greekQuotation}
%
In other words, \enquote{one} can be predicated of all of the meanings of \gk{τὸ ὄν} mentioned in Γ, 2, and above all of \gk{οὐσία} itself; nevertheless, in reality, \gk{τὸ ἕν} is not distinct from \gk{τὸ ὄν}. Moreover, Aristotle, in contrast to Plato, clearly considers the First Philosophy to be primarily the study of \emph{ens qua ens}, not of the One.%
%
\footnote{It is sufficient to see the opening sentence of \emph{Metaphysics}, Γ: \enquote{\gk{ἔστιν ἐπιστήμη τις ἣ θεωρεῖ τὸ ὂν ᾗ ὂν καὶ τὰ τούτῳ ὑπάρχοντα καθ᾽ αὑτό}} (\enquote{There is a certain science that studies \emph{ens qua ens}, and the properties in it \emph{per se}} [my translation]), \cite[Γ, 1, 1003a21]{aristotle:metaphysics}.}
%
This passage, therefore, shows, at least implicitly, that \gk{τὸ ἕν} has all of the characteristics of the \emph{transcendentia}; namely, \emph{additio}, \emph{conversio}, and \emph{processio}: \gk{τὸ ἕν} renders a characteristic of \gk{τὸ ὄν} explicit without being really distinct, it has the same logical \enquote{extension} as \gk{τὸ ὄν}, and it \enquote{proceeds} from \gk{τὸ ὄν} as one of its \enquote{properties.}

In Book \gk{Α} of his \emph{Nicomachean Ethics}, Aristotle makes a similar analysis with a property of \emph{ens} that is closer to the subject matter of this paper: \gk{τὸ ἀγαθόν}. He states, {\gk{τὸ δ᾽ ἀγαθὸν λέγεται καὶ ἐν τῷ τί ἐστι καὶ ἐν τῷ ποιῷ καὶ ἐν τῷ πρός τι}.%
%
\footnote{\enquote{The good, however, is said in the \enquote{what is it} [i.e., essence or substance] and in the \enquote{which} [quality] and in the \enquote{towards what} [relation]} (my translation), \cite[\gk{Α}, 1, 1096a19-23]{aristotle:ethics}.}
%
The good, therefore, can to be found in precisely those categories---substance, quality, and relation---that are common to all substances, including separated substances (such as human souls---inasmuch as they are subjects of purely spiritual accidents---and angels, to use Christian parlance). Aristotle goes further, stating,

\begin{greekQuotation}
ἔτι δ᾽ ἐπεὶ τἀγαθὸν ἰσαχῶς λέγεται τῷ ὄντι (καὶ γὰρ ἐν τῷ τί λέγεται, οἷον ὁ θεὸς καὶ ὁ νοῦς, καὶ ἐν τῷ ποιῷ αἱ ἀρεταί, καὶ ἐν τῷ ποσῷ τὸ μέτριον, καὶ ἐν τῷ πρός τι τὸ χρήσιμον, καὶ ἐν χρόνῳ καιρός, καὶ ἐν τόπῳ δίαιτα καὶ ἕτερα τοιαῦτα).%
%
\footnote{\enquote{Again, \enquote{good} means the same as \enquote{\emph{ens},} for with respect to the \enquote{what} [substance] it means, for example, God and the intellect; and with respect to the \enquote{which} [quality], virtue; and with respect to the \enquote{how much} [quantity], the just measure; and with respect to the \enquote{toward which} [relation], usefulness; and with respect to time, occasion [or \enquote{acceptable time}]; and with respect to place, dwelling; and other similar things} (my translation), \cite[\gk{Α}, 1, 1096a24-26]{aristotle:ethics}.}
%
\end{greekQuotation}
%
Aristotle specifically rejects Plato's \enquote{univocal} notion of the good (which, as we saw, arises because the ideas---even the meta-ideas and Principles---are extrinsic to the realities that participate in them), saying quite explicitly that \gk{τὸ ἀγαθόν} can be predicated of all the categories: \enquote{\gk{δῆλον ὡς οὐκ ἂν [τἀγαθὸν] εἴη κοινόν τι καθόλου καὶ ἕν: οὐ γὰρ ἂν ἐλέγετ᾽ ἐν πάσαις ταῖς κατηγορίαις, ἀλλ᾽ ἐν μιᾷ μόνῃ.}}\,%
%
\footnote{\Cite[\gk{Α}, 1, 1096a28-30]{aristotle:ethics}: \enquote{It clearly follows that [the good] cannot be a common and unique \enquote{universal,} for [if it were] it could not be said of all the categories, but only of one} (my translation).}
%
In Aristotle, therefore, the \enquote{properties} of \gk{τὸ ὄν} can be said \enquote{in many ways,} just like \gk{τὸ ὄν} itself: they are intrinsically (not extrinsically) differentiated and therefore \enquote{analogical.}\,%
%
\footnote{In other words, from each category follows a different type of goodness, which, however, is recognizably \enquote{goodness} in every case. Aristotle clearly makes what the Scholastics would call an analogy of proportionality between \gk{τὸ ὄν} and its properties (\gk{τὸ ἕν} and \gk{τὸ ἀγαθόν}). It is less clear from these passages whether he intends to make an analogy of reference or \enquote{attribution} to the \emph{priceps analogatum}, as he does for \gk{τὸ ὄν}. It should also be noted that Aristotle is clearly rejecting a number of key Platonic notions: the priority of the One and the Good over \emph{ens}, the univocity and extrinsicness of the Principles and meta-ideas, and their  subsistence as \enquote{universals.}

For an overview of how Aristotle discusses the \gk{τὸ ἕν} and \gk{τὸ ἀγαθόν}, see \cite{loux:aristotle}. I cannot, however, be in agreement with the author in denying, in his words, ``the claim that such notions [as Being, One, and Good] are not univocally predicable of all things'' (225). The author seems to be making an alternative between univocal predication and what he calls ``ambiguity'' (which seems to be equivalent to equivocity in Scholastic terminology). The simple solution to this dilemma, it seems to me, is to realize that there is a \emph{tertium quid}; namely, analogy. In any case it seems contrary to fact to argue that an accident ``is,'' or else is ``good'' or ``one,'' in exactly the same way (\emph{univoce}) as a substance. The problem disappears entirely once we realize (as we will see below) that things ``are,'' and are ``good'' and ``one,'' to the degree that they receive \emph{esse} by participation from the radical \emph{actus essendi}.}
%
It is also Aristotle who first frames the notion of the good in terms of desire or appetite: in this same passage of the \emph{Ethics}, he describes \gk{τὸ ἀγαθόν} as \gk{οὗ παντ᾿ ἐφίεται},%
%
\footnote{\enquote{That which all things desire,} \cite[I, 1, 1094a3]{aristotle:ethics}.}
%
which is the notion that St.~Thomas accepts.%
%
\footnote{He translates the phrase as \enquote{quod omnia appetunt;} see \cite[I, q.~5, a.~1, co.]{st:summa}.}
%

\subsection{Pseudo-Dionysius the Areopagite}

It is the merit of Pseudo-Dionysius the Areopagite to adapt the Neoplatonic reflection on \emph{bonum} to Christian revelation.%
%
\footnote{It is thought that Pseudo-Dionysius wrote his works in the late fifth or early sixth century, since his works bear a resemblance to and seem to depend on Proclus, who died in 485. See \cite[§~1]{stanford:pseudo-dionysius}.}
%
He could not accept the theory proposed by Plotinus and his followers that the One or Good causes its effects necessarily and that each sphere or level is caused by the mediation of the sphere immediately above it, but instead affirms that God freely and directly creates all of reality outside of himself.%
%
\footnote{Plotinus seems to suggest that the overflow (\gk{τὸ προϊέναι}) from the One is necessary, not free. He compares it to a ``circumradiation'' (\gk{περίλαμψις}) or eternal irradiation (\gk{ἐπιλάμπειν ἀεὶ}); hence the One seems no more free to produce the \gk{Νοῦς} than the sun or fire to produce light and heat. See \cite[V,~1,~6, and V,~3,~12]{plotinus:enneads}. Regarding Pseudo-Dionysius himself, see \cite[§~4.1]{stanford:pseudo-dionysius}.}
%
Following other Neoplatonists, Dionysius adheres to the priority of the Good over Being: in his treatise \emph{On the Divine Names}, the first attribute of God he discuss is Goodness, followed by Being.%
%
\footnote{\cite[See][65--66]{orourke:pseudo-dionysius}. The first three chapters of \emph{De divinis nominibus} regard the One and Good, whereas only the fourth chapter deals with Being.}
%
To justify this priority, Dionysius argues that, although the good extends to all things, even things that are (in Thomistic parlance) \emph{in potentia}, being extends only to things \emph{in actu}. Hence, for Dionysius, only the name of \enquote{Good} can express God's nature:
%
\begin{greekQuotation}
Καὶ γὰρ ἡ τἀγαθοῦ θεωνυμία τὰς ὅλας τοῦ πάντων αἰτίου προόδους ἐκφαίνουσα καὶ εἰς τὰ ὄντα καὶ εἰς τὰ οὐκ ὄντα ἐκτείνεται καὶ ὑπὲρ τὰ ὄντα καὶ ὑπὲρ τὰ οὐκ ὄντα ἔστιν. Ἡ δὲ τοῦ ὄντος εἰς πάντα τὰ ὄντα ἐκτείνεται καὶ ὑπὲρ τὰ ὄντα ἔστιν.%
%
\footnote{\enquote{For the Divine Name of the Good, as making known entire processions of the Cause of all, is extended, both to things being, and things not being, and is above things being, and things not being. But the Name of Being is extended to all things being, and is above things being,} \cite[V, 1]{pg:dionysius:DN}; translation from \cite[73]{pg:dionysius:DN:en}.}
%
\end{greekQuotation}
%
This conclusion may result from the tendency of Platonists to describe goodness as \enquote{generosity,} rather than \enquote{appetibility;} that is, as referring more to efficient than to final cause. Hence, for Dionysius the Good formally signifies overflowing (\gk{τὸ προϊέναι}) into the things that participate in it.%
%
\footnote{See \cite[89]{fabro:nozione}, and \cite[160]{definance:connaissance}. Thomas Aquinas, preferring Aristotle's notion of \gk{τὸ ἀγαθόν}, considers that the Good, properly speaking, is associated with perfection and act---and hence is based on and presupposes \emph{esse}---as we will see in greater detail below. The two conceptions are not completely incompatible: for example, Aquinas accepts the Dionysian maxim, \emph{Bonum est diffusivum sui esse}; significantly, however, he interprets it in terms of \emph{final}, not \emph{efficient}, causality. As the development below will show, associating efficient causality properly with \emph{esse} and final causality with \emph{bonum} seems to be more congruous. Regarding this point, see Aquinas' objection and answer in \cite[I, q.~5, a.~4, arg.~2 and ad 2]{st:summa}; as well as \cite[86--87]{orourke:pseudo-dionysius}, and \cite[68]{definance:etre-et-agir}.}
%

Dionysius differs from earlier Neoplatonists in fully identifying God with his attributes---especially Goodness and Being (more precisely \gk{τὸ ὑπερ\-άγαθον} and \gk{τὸ ὑπερ\-εῖναι})%
%
\footnote{For Dionysius, strictly speaking, being (\gk{τὸ εἶναι}) applies only to creatures: it is the first and greatest gift of the Creator to his creation. God, strictly speaking, is the ``Super-good'' and the ``Super-being''---that is, beyond both creaturely goodness and creaturely being. Other terms related to ``super-being'' include  \gk{τὸ ὑπερ\-ούσιον} and \gk{ἡ ὑπερ\-ούσιος οὐσία}. See \cite[II, 3; V, 1; and V, 5]{pg:dionysius:DN}, and \cite[69]{orourke:pseudo-dionysius}.}
%
---whereas Plotinus and his followers maintain that the Good (or One) is a separate hypostasis from those that participate in it: the Intellect (\gk{Νοῦς}) and the Soul (\gk{Ψυχή}).\footcite[136--137]{fpascual:plotinus} Although Dionysius does not directly address the problem of the intrinsic constitution of \emph{ens}---his works dealing with the Good and Being study the attributes of God---it is largely from the Areopagite that Thomas Aquinas obtains his notion of participation, as well as his ``intensive'' conception of being, both of which are central to the problem of the intrinsic foundation of \emph{bonum}.%
%
\footnote{See \cite[90-98]{fabro:nozione}, for a discussion of how much Aquinas assimilates from Dionysius. Fabro notes that a superficial reading of Dionysius could easily construe his doctrine as being excessively close to pagan Neoplatonistm: for example, Dionysius says in \emph{De divin. nom.} V, 4, \gk{Ἀλλ᾿ αὐτός ἐστι τὸ εἶναι τοῖς οὖσι καὶ οὐ τὰ ὄντα μόνον, ἀλλὰ καὶ αὐτὸ τὸ εἶναι τῶν ὄντων, ἐκ τοῦ προαιωνίως ὄντος}, (``But he [God] is the \emph{esse} in \emph{entia}, and not only \emph{entia}, but also the very \emph{esse} of the \emph{entia} [is] eternally from him'' [my translation]), which could be interpreted as pantheism. Aquinas, however, makes many efforts to defend and correct him. This example, for instance, can easily be understood in an orthodox way using the notion of participation.} 
%

It seems likely, then, that Dionysius provided Aquinas with the necessary groundwork for his doctrine regarding \emph{actus essendi}, albeit with a few modifications.     Dionysius correctly saw that God's \emph{esse} (or ``\emph{super-esse}'') is infinitely superior to that of his creatures (so much so that he goes so far as to say that God is above all \emph{esse} and even could even be characterized as \emph{non-esse}).%
%
\footnote{\cite[V, 4]{pg:dionysius:DN}: \gk{Ὁ ὢν ὅλου τοῦ εἶναι κατὰ δύναμιν ὑπερ\-ούσιός ἐστιν. [\ldots] μᾶλλον δὲ οὔτε ἐστίν}, ``The One Who Is [i.e., God] is super-substantially above all \emph{esse} according to power. [\ldots] Rather, he [God] is not'' (my translation). Note the similarity to Plato's dialectic method: the One Who Is, in reality is not.}
%
Aquinas corrects Dionysius' excessive apophatism, reminiscent of Plato's dialectic, and---following Aristotle---gives both logical and ontological priority to \emph{esse}, rather than \emph{bonum}.%
%
\footnote{For a full discussion regarding the primacy of \emph{esse} in Aquinas, see \cite[109--113]{orourke:pseudo-dionysius}.} Therefore, Aquinas argues, \emph{Esse} is the proper name for God, and the proper effect of God in his creatures is precisely their \emph{esse} or \emph{actus essendi}.%
%
\footnote{Aquinas finds a confirmation of his doctrine in the Divine Name given in Exodus 3:14, ``I Am Who Am.'' See \cite[I, cap.~22, n.~10 (Marietti n.~211)]{st:contragent}:
``Hanc autem sublimem veritatem Moyses a domino est edoctus, qui cum quaereret a domino, Exod.~3 dicens: \emph{si dixerint ad me filii Israel, quod nomen eius? Quid dicam eis?} Dominus respondit: \emph{ego sum qui sum. Sic dices filiis Israel: qui est misit me ad vos}, ostendens suum proprium nomen esse qui est. Quodlibet autem nomen est institutum ad significandum naturam seu essentiam alicuius rei. Unde relinquitur quod ipsum divinum esse est sua essentia vel natura.''
Regarding this topic, see Gilson's discussion in chapter 3 of \emph{Le Thomisme}: \cite[99-112]{gilson:thomisme}.}

\begin{DONE}
\CITEME{Arabs?}

\CITEME{Medieval?}
\end{DONE}

\section{\emph{Resolutio Secundum Rationem} of the Good}

Having seen a brief historical overview of \emph{bonum}, especially in view of the profound influence that Pseudo-Dionysius had on Saint Thomas, we are in a position to begin the \emph{resolutio}. In doing so, we will be guided by Thomas Aquinas' masterful treatise on \emph{bonum} in the \emph{Summa theologia}, I, q.~5, a.~1, and his \emph{Quaestiones disputatae de veritate}, espeically \emph{quaestio} 21.%
%
\footnote{See \cite[I, q.~5, a.~1]{st:summa}, and \cite[q.~21]{st:deveritate}. See also \cite[63-69]{aertsen:good_as_transcendental}, for an excellent summary of Saint Thomas' reasoning. I have taken the idea of founding the good noetically, ontologically, and theologically from \cite{mitchell:transcendentals}.}

\subsection{Derivation \emph{in Via Inventionis}}

The first step of our ``inventive'' phase regards whether \emph{bonum} can in fact be considered a \emph{transcendens} (as the philosophical tradition has clearly maintained, at least until the modern era). This assertion must be demonstrated, because it is evident that there are certain realities that \emph{cannot} be considered good, at least not without qualification; for example, natural disasters, wars, sinful actions, and fallen angels.%
%
\footnote{Thomas raises this very objection in his \emph{De veritate}. See \cite[q.~21, a.~5, arg.~3.]{st:deveritate}: ``de quocumque praedicatur aliquid essentialiter, oppositum eius de eo praedicari non potest. Sed oppositum boni praedicatur de aliqua creatura, scilicet malum. Ergo creatura non est bona per essentiam.''} Hence, it is necessary to begin with the most easily understood notion of \emph{bonum}---the \emph{ratio boni}---and examine how it applies analogically to \emph{entia}.

It is an evident fact that certain \emph{entia} are the object of man's desire and that others are not: in this sense, \emph{bonum} is one of the first notions that is known immediately by the intellect with the help of the \emph{intellectus agens}: what first falls into the intellect is \emph{ens}, but some \emph{entia} fall into the intellect as \emph{attractive}, others as \emph{repulsive}, and for some the attractive or repulsive force is so small as to render them practically neutral.%
%
\footnote{As we saw above, Thomas defines the \emph{ratio boni} as follows: ``Ratio enim boni in hoc consistit, quod aliquid sit appetibile, unde philosophus, in I \emph{Ethic}., dicit quod bonum est quod omnia appetunt'' \parencite[I, q.~5, a.~1, co.]{st:summa}. Since \emph{omnia appetunt} the good, Thomas evidently regards \emph{bonum} as a notion that everyone grasps \emph{in actu exercito}.}
%
Since \emph{bonum} is a primitive notion, it is sufficient to recognize that its \emph{ratio} is that of desirability or appetibility: no demonstration is possible.%
%
\footnote{Against the opinion of the Platonist tradition---that the \emph{ratio boni} is ``generosity''---we may simply state that ``generosity'' is not an immediately grasped notion. It is something that must be discovered first by \emph{resolutio}. We could say---following a good insight by M. Heidegger---that we spontaneously grasp the thing desired (as ``ready-to-hand'' or \emph{zuhanden}) and only afterward reflect on its capacity to perfect (which would be a kind of ``presence-at-hand'' or \emph{Vorhandenheit}). See \cite[42 and 69]{heidegger:being} (English translation in \cite[67 and 98]{heidegger:being:en}).}
%
When we investigate the matter further---argues Aquinas---we discover that when something attracts us, it invariably has some quality or other property capable of perfecting us: what is good is desirable precisely because it is \emph{perfective}:%
%
\begin{quotation}
Alio modo ens est perfectivum alterius non solum secundum rationem speciei, sed etiam secundum esse quod habet in rerum natura. Et per hunc modum est perfectivum bonum.%
%
\footcite[q.~21, a.~1, co.]{st:deveritate}
%
\end{quotation}%
%
It is, of course, perfective to different degrees, depending on the ontological degree of the perfection desired: Ice cream is desirable---and in that sense ``good''---because it provides satisfaction to the nutritive appetites; studies are good because they perfect the intellect; friendship is good because it fulfills man's natural need for companionship and also because he cannot truly be fulfilled, paradoxically, without making a gift of himself to others; morally good acts are good because they provide man with virtues that enable him to reach his final end. The ``desirability'' and ``perfectiveness'' of the good is analogical, just like \emph{ens}.

If, however, the \emph{bonum} desired is perfective, it must have an inherent perfection that makes it so: potency cannot be reduced to act except by an agent that is in act.%
%
\footnote{This is the principle of causality as described in the \emph{prima via}. See \cite[I, q.~2, a.~3, co.]{st:summa}.}
%
In this case the potential principle is the individual that desires, the act to be obtained is the perfection that the individual desires for itself, and the agent is whatever produces that perfection.%
%
\footnote{For example, if I am cold and desire warmth, I am warm only \emph{in potentia}. In order to obtain the warmth, I must approach the fire, which is warm \emph{in actu}. The fire is ``good'' and hence desirable because it has an inherent perfection (warmth). Therefore, the individual, or \emph{suppositum}, is related to the agent with the desired perfection as \emph{participans} to \emph{participatum}. We will see this type of participation many times further on in our investigation; it is a key notion for resolving our problem.}
%
``Perfection,'' as can be seen, is really just another term for ``act,'' especially ``ultimate'' or ``second'' act.%
%
\footnote{In its proper sense, ``perfection'' means ``completeness;'' it is the end result of a development, or passage from potency to act. More broadly, it can refer to any \emph{actus secundus} added to a substance, even if it is not the result of a process; or more broadly still to any act. The term can even be applied to God, with the caution that God has no potency and suffers no movement whatsoever.
See \cite[193]{definance:etre-et-agir}: ``La perfection---la structure même du mot l'indique---s'entend comme un achèvement, une fin. Et si épuré qu'on le suppose, ce concept garde toujours de son origine une référence à un progrès; il représente le terme d'un accroissement idéal de valeur. La perfection divine elle-même est pour nous le point à l'infini vers lequel monte la courbe de nos ascensions dialectiques.''}
%
Hence, although desirability was our port of entry to \emph{bonum}, we see that it leads us to a more profound reality: that of intrinsic perfection, which Aquinas calls the \emph{natura boni}.%
%
\footnote{Aquinas makes this useful distinction in the \emph{Summa contra gentiles}: ``Communicatio esse et bonitatis ex bonitate procedit. Quod quidem patet et ex ipsa natura boni, et ex eius ratione. Naturaliter enim bonum uniuscuiusque est actus et perfectio eius. [\ldots] Ratio vero boni est ex hoc quod est appetibile. Quod est finis.'' See \cite[I, cap.~37, n.~5 (Marietti n.~307)]{st:contragent}.
See also \cite[65]{aertsen:good_as_transcendental}; and \cite[190]{definance:etre-et-agir}.}
%
 Clearly, not every perfection in a thing produces a corresponding desire in every subject, but by analogy we can call anything ``good'' to the degree that it has obtained the perfection expected of it.%
%
\footnote{Since desire requires two ``poles''---the subject that desires and the object that satisfies the desire---the degree to which the latter produces a desire in the former depends on the intrinsic makeup of the former, which in turn depends on the thing's essence. A non-rational animal is incapable of desiring intellectual knowledge, for example, but that does not take away the intrinsic ``goodness'' (perfection) of, say, learning Aristotle's \emph{Metaphysics}.} In this sense, Aquinas says, ``unumquodque in tantum bonum sit in quantum est perfectum.''\,%
%
\footnote{\Cite[III, cap.~24, n.~6 (Marietti n.~2051)]{st:contragent}; the same language is found in \cite[III, cap.~20, n.~2 (Marietti n.~2010)]{st:contragent}.}
%
Moreover---as will be seen in greater detail in the next chapter---all proper act in a substance can be characterized as one of various kinds of \emph{esse in actu} (of substance, accident, and \emph{operari}) and is ultimately derived from and ``virtually'' contained in that substance's \emph{actus essendi} (or \emph{esse ut actus}). For this reason, Aquinas can state ``esse enim est actualitas omnis rei,''\,\footcite[I, q.~5, a.~1, co.]{st:summa} \emph{Bonum}, therefore, is founded on \emph{esse}, and a thing's goodness is in proportion to the ``quantity'' of \emph{esse} that it possesses.
Therefore, \emph{ens}---that is, \emph{id quod habet esse}---can be said to be coextensive with \emph{bonum}.%
%
\footnote{The term ``quantity'' here is, of course, taken in an analogical sense. We will see that we can, in a way, speak of both an ``extensive'' quantity of \emph{esse} and an ``intensive'' quantity, which correspond to the ``predicamental'' and ``transcendental'' participations of \emph{esse}.}
%

In fact, it is the very \emph{esse} that makes a thing good:
%
\begin{quotation}
Ipsum igitur esse habet rationem boni. Unde sicut impossibile est quod sit aliquid ens quod non habeat esse, ita necesse est ut omne ens sit bonum ex hoc ipso quod esse habet; quamvis etiam et in quibusdam entibus multae aliae rationes bonitatis superaddantur supra suum esse quo subsistunt.\footcite[q.~21, a.~2]{st:deveritate}
\end{quotation}
%
Since \emph{ens} is \emph{id quod habet esse} and \emph{esse} is what makes a thing good, it follows that in reality, there is no difference between \emph{ens} and \emph{bonum}: the distinction is strictly conceptual (\emph{rationis tantum}). \emph{Bonum} adds the notion of ``desirability'' or ``perfection'' to the notion of \emph{ens}, but in so doing it merely renders explicit something that is already present in \emph{ens}.%
%
\footnote{Aquinas mentions three ways that a notion can be ``added'' to another: the \emph{additio} can be \emph{realis}, in which case the two terms in question are really distinct from another; or else \emph{rationis}, in which case the two terms are distinct only conceptually. The latter, in turn, can be \emph{cum contractione}---as when one narrows a genus to a smaller genus or to a species---or \emph{sine contractione}---when the terms are perfectly coextensive. All of the \emph{transcendentia} are ``additive'' in the third sense only. See \cite[q.~21, a.~1, co., and q.~1, a.~1, co.]{st:deveritate}, as well as \cite[See also][192]{wippel:metaphysical_thought}.} From this reflection, it is clear that \emph{bonum} fulfills the three characteristics of a \emph{transcendens}: \emph{additio}, because it renders explicit a property of \emph{ens}; \emph{conversio}, because it is coextensive with \emph{ens}, and \emph{processio}, because it is posterior logically (it is known after \emph{ens}) and ontologically (a thing is good \emph{because} it is \emph{ens}).

\subsection{\emph{Bonum Simpliciter} and  \emph{Secundum Quid}}

At this point, we are in a position to make a definitive answer to the objection that we identify certain things as ``evils''---either because they are repulsive or because they are evidently lacking in inherent perfection---and hence that \emph{bonum} cannot be coextensive with \emph{ens}. Aquinas finds the solution in the different ways that \emph{ens} and \emph{bonum} are predicated.%
%
\footnote{In this discussion we will follow Thomas Aquinas' reasoning as found in \cite[I, q.~5, a.~1, co.]{st:summa}.} \emph{Ens}, argues Aquinas, is understood as in proportion to act, as we saw above, and so it applies most properly where there is the most radical reduction of potency to act. Two such reductions are readily knowable: that of substrate to accident, and that of matter to form.%
%
\footnote{The composition of \emph{actus essendi} with \emph{essentia} can only be known by a rigorous and scientific \emph{resolutio}.} The latter is far more radical, and so it is the \emph{esse in actu} (the \emph{esse} pertaining to the form, or what Thomas calls \emph{esse substantiale}) of the substance that makes a thing \emph{ens simpliciter}.%
%
\footnote{In the following chapter, we will see that every \emph{ens} is founded on a radical ontological ``\gk{ἐνέργεια}'' called \emph{actus essendi} or \emph{esse ut actus}. This \gk{ἐνέργεια} in turn communicates itself to three levels of act called \emph{esse in actu}.} Any posterior \emph{esse in actu}---the \emph{actus superadditi}, those acts proper to the accidents and operation---are only \emph{ens} in qualified sense (``secundum quid''). This distinction in predication is born out in practice: when we say, ``Fifi (the cat) is,'' without qualification, we mean simply that it exists.%
%
\footnote{Here, I use the verb ``to exist'' in its non-technical sense. We will see that \emph{esse} cannot be reduced to mere \emph{existentia}, or ``placement'' outside a thing's causes.} On the other hand, when we say, ``Fifi is white,'' we refer to a certain aspect or manner (\emph{modus}) of its being.%
%
\footnote{For another discussion of the distinction between \emph{esse simpliciter} and \emph{esse secundum quid} along the same lines, see \cite[I, q.~76, a.~4, co.]{st:summa}.}
%

\emph{Bonum}, however, works in precisely the opposite way. A thing is desirable because it possesses a perfection, and (in creatures) a perfection, properly speaking, is an \emph{actus superadditus}.%
%
\footnote{See \cite[lc.~4]{st:dehebdo}: ``Alia vero bonitas consideratur in eis absolute, prout scilicet unumquodque dicitur bonum, inquantum est perfectum in esse et in operari. Et haec quidem perfectio non competit creatis bonis secundum ipsum esse essentiae eorum, sed secundum aliquid superadditum, quod dicitur virtus eorum, ut supra dictum est;'' also \cite[IV, lc.~1, n.~269]{st:divnomin}: ``res aliae, etsi inquantum sunt, bonae sint, tamen perfectam bonitatem consequuntur per aliquod superadditum supra eorum esse.''}
%
Whether we are referring the good as desirability or as intrinsic perfection, therefore, a thing is called good without qualification (\emph{simpliciter}) only when it possesses an \emph{actus superadditus}, its \emph{ultimate} perfection. On the other hand, the goodness that something possesses simply by being only permits it to be characterized as good in a qualified sense (\emph{secundum quid}).%
%
\footnote{For example, a pizza is good and desirable to the degree that it is tasty and crispy; only this type of pizza is good ``\emph{simpliciter},'' but tastiness and crispiness are dispositions of the pizza---not the pizza itself---and hence \emph{actus superadditi}. A mediocre pizza, or even a bad pizza, might be good in some respect (``\emph{secundum quid}''), but it cannot be called good without qualification, simply because it exists.} This analysis, therefore, provides the answer: all \emph{entia} are good to the degree that they possess \emph{esse substantiale}---even things that we characterize as evil. However, since things are \emph{primarily} or \emph{simply} called good because they are desirable or perfect---both of which entail ``ultimate'' or ``second'' act (\emph{actus superadditus})---and not simply because they are substances.

\subsection{Derivation \emph{in Via Iudicii}}

The result of our inventive process is that \emph{bonum}, known first through its desirability and then upon reflection by its intrinsic perfection, is founded on an ontological source of act---an \gk{ἐνέργεια}---that Saint Thomas calls \emph{esse}.%
%
\footnote{A more detailed \emph{resolutio} of \emph{esse} will be done in the next chapter.} Having done a demonstration of this foundation, we are now in a position to examine briefly \emph{in via iudicii} how it is that \emph{esse} founds \emph{bonum} noetically. As we said above, the most fundamental and self-evident notion is \emph{ens}---the \emph{ratio entis}---and it is the first to fall into the intellect;%
%
\footnote{Among many examples, see ``Ad primum ergo dicendum quod id quod primo cadit in intellectu, est ens, unde unicuique apprehenso a nobis attribuimus quod sit ens; et per consequens quod sit unum et bonum, quae convertuntur cum ente,'' \parencite[I-II, q.~55, a.~4, ad 1]{st:summa}; and ``primo sint intelligenda ens et non ens'' \parencite[pars~2, q.~4, a.~1, co.~2]{st:detrinitate}.}
%
the rest of the \emph{transcendentia}---including \emph{bonum}, as we saw---are logically posterior. It is difficult to describe the precise sequence in which the intellect derives these notions, but we can say the following: from \emph{ens} is immediately derived the opposite notion, \emph{non-ens}, which permits the first speculative \emph{divisio} (the foundation for the principle of non-contradiction). The rest follow from other \emph{divisiones} and \emph{compositiones}: the next \emph{transcendens} to be derived would be \emph{unum}, which entails a denial of division; from \emph{unum} is derived \emph{multum}, which implies \emph{aliquid}, indicating the ``separation'' or ``division'' that one \emph{ens} has with respect to another;%
%
\footnote{In Aquinas' works, \emph{aliquid}, when considered as a \emph{transcendens}, does not take on the classical Latin meaning ``something,'' but is closer in meaning to \emph{aliud} (``something else'').}
%
and finally the \emph{transcendentia} that have to do with the \emph{convenientia} of an \emph{ens} with a subject: \emph{bonum} and \emph{verum}.%
%
\footnote{This reflection is guided by \cite[q.~21, a.~1, co.]{st:deveritate}{}: ``Id autem quod est rationis tantum, non potest esse nisi duplex, scilicet negatio et aliqua relatio. [\ldots] Illa autem relatio, secundum philosophum in V \emph{Metaph}., invenitur esse rationis tantum, secundum quam dicitur referri id quod non dependet ad id ad quod refertur, sed e converso, cum ipsa relatio quaedam dependentia sit, sicut patet in scientia et scibili, sensu et sensibili.'' Where does \emph{res} fit in? It is somewhat difficult to say, because on the one hand Aquinas says that it results from an affirmation (\emph{De veritate}, q.~1, a~1., co.), but this passage from \emph{quaestio} 21 asserts that of affirmations, only a relation can be \emph{rationis tantum}. Since \emph{res} signifies the essence, and at this stage the intellect is only grasping these notions \emph{in actu exercito}, one presumes that \emph{res} is derived close to the beginning, as soon as the intellect does its first abstraction. Whether \emph{res} is a relation, however, is more problematic. One possibility is that \emph{res} signifies the relation of ``measurement'' or ``delimitation'' that the essence has with the \emph{actus essendi}.} The last two, Aquinas argues, both have to do with the capacity of the \emph{ens} to perfect the subject. \emph{Verum}, however, only does so \emph{secundum rationem speciei tantum}---which is another way of saying that the intellect (or sensibility) of the subject receives the form of \emph{ens} only \emph{intentionally}.%
%
\footnote{The \emph{species} is the ``image''---sensible or intelligible---produced in the knower by the thing known. Regarding the \emph{species}, see \cite[277--279]{definance:etre-et-agir}; also the chapter IV of \cite[73--142]{lucas:hombre}, especially 114--115 and 133--138.}
%
\emph{Bonum}, on the other hand, perfects a subject \emph{secundum esse quod habet in rerum natura}; that is, it actually produces or changes an accidental form in the subject, as fire makes the subject warm.%
%
\footnote{One consequence of this distinction is that the subject of \emph{bonum} could, analogically speaking, be any substance, because it simply permits the subject to reach its end, however banal that end might be. \emph{Verum}, on the other hand, can only be experienced by a subject able to receive a \emph{species}, hence capable at least of sensible knowledge.

Looking at \emph{verum} and \emph{bonum} in this way helps to understand why Saint Thomas---following Aristotle---considers the intellect to be a more noble faculty than the will: \emph{verum} is accessible only to those creatures that are sufficiently noble to receive it.
See \cite[III, cap.~26, n.~8 (Marietti n.~2078)]{st:contragent}: ``Cum enim beatitudo sit proprium bonum intellectualis naturae, oportet quod secundum id intellectuali naturae conveniat quod est sibi proprium. Appetitus autem non est proprium intellectualis naturae, sed omnibus rebus inest: licet sit diversimode in diversis.''
Regarding the relationship between intellect and will in general, see \cite[I, q.~82, a.~3]{st:summa}. For Thomas, beatitude consists in an exercise of the most noble faculty. See \cite[378--381]{izquierdo:vita}.}
%
In this way, we are able to reverse our \emph{resolutio} of \emph{bonum}: at the root is \emph{esse} (in its various forms, all rooted in the \emph{actus essendi}), which founds an \emph{ens}'s intrinsic perfection (\emph{natura boni}); this perfection makes the \emph{ens} perfective to subjects capable of receiving the perfection; and it is this perfectiveness that makes \emph{entia} objects of desire.

\section{Conclusions Regarding \emph{Bonum}}

From the reflections made above, we can draw conclusions to prepare for the more profound \emph{resolutio} of the principles of \emph{ens} that will take place in the next chapter. First of all, the distinction made by Saint Thomas between \emph{ratio boni} and \emph{natura boni}---between \emph{bonum ut appetibile} and \emph{bonum ut perfectum}---helps us to see more clearly that \emph{entia} desire \emph{bonum} because they also desire their \emph{own} perfection. There is an intrinsic drive or impulse in all \emph{entia} reach the fulfillment that is proper to them: to reach the maximum ``quantity'' of \emph{esse in actu} possible, a state that could be used as a provisional definition of ``happiness'' or \gk{εὐδαιμονία},%
%
\footnote{See, for example, \cite[\gk{Α}, 7]{aristotle:ethics}: \gk{τέλειον δή τι φαίνεται καὶ αὔταρκες ἡ εὐδαιμονία, τῶν πρακτῶν οὖσα τέλος}, ``Happiness, therefore, appears as something fulfilled and as self-sufficient, being the end of actions.'' Aristotle maintains that this fulfillment is obtained by accomplishing the ``work'' (\gk{ἔργον}) proper to each thing, which depends on its ontological level (whether it merely ``lives,'' or else also has sentient life, or else is also rational).
He says in further on in 1098a7 and 1098a16--17, \gk{εἰ δ᾽ ἐστὶν ἔργον ἀνθρώπου ψυχῆς ἐνέργεια κατὰ λόγον ἢ μὴ ἄνευ λόγου, [\ldots] τὸ ἀνθρώπινον ἀγαθὸν ψυχῆς ἐνέργεια γίνεται κατ᾽ ἀρετήν, εἰ δὲ πλείους αἱ ἀρεταί, κατὰ τὴν ἀρίστην καὶ τελειοτάτην}, ``If, however, the work of man is the act of the soul that corresponds to reason, or [at least is] not without reason, [\ldots] then the good of man is the act of the soul that arises according to virtue (\gk{κατ᾽ ἀρετήν}). If, however, there are more virtues, then according to the best and most fulfilled [among them]'' (my translation in both cases).

Regarding happiness in Aristotle, see also \cite[203--204]{izquierdo:vita}; for a survey of how it is treated in various philosophies, see \cite[254--275]{millan-puelles:libre}.}
%
which Aristotle would say is attained when a thing has attained its proper \emph{virtus}.%
%
\footnote{See \cite[\gk{Η},~2, 264a13-15]{aristotle:physics}: \gk{ἡ μὲν ἀρετὴ τελείωσίς τις---ὅταν γὰρ λάβῃ τὴν αὑτοῦ ἀρετήν, τότε λέγεται τέλειον ἕκαστον}, ``Virtue (\gk{ἀρετή}) is a certain fulfillment---whenever each one attains its own virtue (\gk{ἀρετή}), then it is said to be perfect'' (my translation); also \cite[34]{orourke:virtus}.}
%
The remaining task of this investigation, therefore, is to determine exactly what this ``expansion'' of \emph{esse in actu} consists in, and how, ontologically speaking, it can be founded on the very structure of \emph{ens}. In order to do this, we will need to discover the constitution of \emph{ens} and resolve its ultimate intrinsic principles; we will see that the desire for fulfillment or happiness (for apt subjects) is founded on the \emph{actus essendi}, which can be characterized as a \emph{virtus}, an inexorable power to ``expand,'' as determined and measured by the essence co-created with it.

\begin{DONE}

The idea is to do a \emph{resolutio secundum rationem} of \emph{bonum}, so as to discover its underlying cause: \emph{esse}. The notion of the good---\emph{ratio boni}---is a first notion that is grasped immediately as \enquote{desirability.}
Our goal is to discover what it is founded on, namely \emph{esse} and eventually \emph{esse ut actus}.

At the end of this section, the conclusion should be that the good means not only \emph{desirability} but also possession of operational \emph{esse in actu}, \emph{bonum simpliciter}. This is related to \emph{happiness} (\gk{εὐδαιμονία}, \gk{μακαρία}, \emph{felicitas}, \emph{beatitudo}).


The question becomes \emph{why does \emph{ens} desire its fulfillment?} To answer that question, we need to look at the very structure of \emph{ens}. We will show that \emph{ens} is composed of \emph{esse ut actus} and the measuring \emph{essentia} (\autoref{chap:ens_resolutio}).

\emph{Bonum simpliciter} is found in perfections that are accidents (\emph{habitus}) or actions (\emph{operari}).

\section{\emph{Exitus} and \emph{reditus}: transcendent and immanent}
A twofold \emph{reditus} is discernible in all \emph{entia}: extrinsic (the return to its Creator) and intrinsic (realization of its full potential: e.g., the accidents and operations are for the sake of the substance).

We will show that that composition---together with the expansive, communicative character of \emph{esse ut actus}---makes the \enquote{desire} for (or better said, intention toward) one's fulfillment \emph{necessary}. Eventually, we will discover that \emph{actus essendi} is a \emph{virtus} that is like an efficient, exemplary, and final cause for operation, and hence a microcosm of the efficient, exemplary, and final causality of \emph{Ipsum Esse} (\autoref{chap:dynamism}).


{The goodness \emph{simpliciter} of a substance}

{The goodness of any substance}


A substance is good \emph{simpliciter} to the degree that its actions (and any secondary principles of actions—i.e., \emph{habitus}) are \emph{in conformity with its nature}.

{The goodness \emph{simpliciter} of created spiritual substances}
Whereas the destiny of lower \emph{entia} is entirely outside their control, man has ability to \emph{choose} his fate (within parameters determined by his nature).

This segues into the notion of \emph{happiness}.
t
We can make a brief theological foundation too, although that is not the scope of this paper.

\end{DONE}