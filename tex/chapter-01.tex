% !TEX TS-program = xelatex
% !TEX encoding = UTF-8 Unicode
% !BIB TS-program = biber

\chapter{Introduction}\label{chap:intro}

David Hume provokingly observes in his \emph{Treatise of Human Nature}, at the end of his argument against deriving moral distinctions from reason,
%
\begin{quotation}
In every system of morality, which I have hitherto met with, I have always remarked, that the author proceeds for some time in the ordinary way of reasoning, [\ldots] when of a sudden I am surprized to find, that instead of the usual copulations of propositions, \emph{is}, and \emph{is not}, I meet with no proposition that is not connected with an \emph{ought}, or an \emph{ought not}. [\ldots] For as this \emph{ought}, or \emph{ought not}, expresses some new relation or affirmation, it is necessary that it should be observed and explained; and at the same time that a reason should be given, for what seems altogether inconceivable, how this new relation can be a deduction from others, which are entirely different from it.\footcite[Bk.~III, sec.~i, 521]{hume:treatise}
\end{quotation}
%
In other words, Hume argues that, from the mere \emph{fact} of existence, it does not seem possible to deduce any sort of moral obligation. This study is not primarily an investigation in ethics, but rather in metaphysics; nevertheless, Hume’s remark provides a thought-provoking starting point. It is an incontrovertible fact that men experience moral obligation.%
%
\footnote{Consistently with his empiricism, Hume practically reduces the apprehension of moral obligation to sentiment; however not even Hume denies the \emph{fact} of obligation. See, for example, \cite[Bk.~III, sec.~i, 520]{hume:treatise}: “Take any action allowed to be vicious: Wilful murder, for instance. Examine it in all lights, and see if you can find that matter of fact, or real existence, which you call \emph{vice}. In which-ever way you take it, you find only certain passions, motives, volitions and thoughts. There is no other matter of fact in the case. The vice entirely escapes you, as long as you consider the object. You never can find it, till you turn your reflection into your own breast, and find a sentiment of disapprobation, which arises in you, towards this action. Here is a matter of fact; but it is the object of feeling, not of reason.”}
%
More broadly, man experiences reality not only as being (\emph{ens}) but also as desirable (\emph{bonum}). Some of these \emph{bona} are only relative, making him more perfect with respect to some concrete aspect (for example, food, which nourishes the body), and others are “absolute,” making him more perfect inasmuch as he is a man (for example, giving alms to the poor, in the right measure). This fact reveals something more profound: that man naturally and spontaneously seeks his own perfection, and that in fact, it is impossible for him not to do so. Man cannot desire his own misery directly. At most, he can desire something that \emph{appears} to be good for him, but which in reality brings about his misery. However, his basic tendency to perfection remains unaltered even in these cases; it is the means that are ill chosen.

I choose man as the example because it is the one best known to us, but in fact this principle applies analogically to all \emph{entia}. Angels seek angelic perfection; man seeks perfection as man; animals and plants seek to reach adulthood and reproduce their species; even stones in their own way seek their own perfection, as when they are placed precariously on a height and “seek” a stable place below by falling. It is this mysterious drive to self-fulfillment, present in all \emph{entia}, that is the topic of this investigation. Why, when it is raised up, “must” the stone fall? Why, when it is placed in the sun, “must” the plant grow toward it? Why, when food is placed before it, “must” the animal eat? Why, when man sees his neighbor in need, does he experience, as an obligation, that he “must” come to his neighbor’s aid?

Generalizing even more, we can say that \emph{ens} does not simply “be;” it also acts. When a man stubs his toe on a rock, he experiences pain; the stone---seemingly the most static of creatures---\emph{acts} on him. No \emph{ens}, therefore, is ever encountered that does not produce \emph{operari}. Indeed, if there were a completely static \emph{ens}, we would be unable to know it, because our intellect depends on the \emph{operari} of \emph{ens} to put it into act. The fact of \emph{operari} and in particular of the tendency for things to their own fulfillment is incontrovertible. This paper, however, is less interested in the \emph{fact itself} as in its metaphysical \emph{foundation}. Our focus will be on the intrinsic causes of \emph{bonum} (not without some mention of extrinsic causes), which we will reduce, by a \emph{resolutio secundum rationem}, to the \emph{actus essendi}, measured by an \emph{essentia} that is understood as \emph{potentia essendi}. In this investigation, we will make use principally of the writings of Saint Thomas Aquinas regarding the intrinsic principles of \emph{ens}, aided by the preliminary work done by Aristotle. We will also make use of the interpretation made by C. Fabro, who (along with others, most notably E. Gilson)%
%
\footnote{Other contributors to this interpretation, cited in this work, include J. de Finance, F. O’Rourke, and W.N. Clarke.}
%
shows that Thomas conceives the \emph{actus essendi} as a rich and fecund act, a \emph{virtus essendi} that is the source of all the actuality in \emph{ens}, not merely a static “fact” of being, or \emph{existentia}.

We will see in our investigation that reducing \emph{actus essendi} from \emph{virtus essendi} to \emph{existentia} produces a simultaneous reduction of \emph{ens} to \emph{essentia} or \emph{ens possibile} and a radical separation between \emph{esse} (\emph{actus primus}) and \emph{agere} (\emph{actus secundus}). It is not difficult to see why: if \emph{esse} is not an act, but just a “fact,” it seems that the only consistent part of \emph{ens} is its essence; whether it exists or not seems unimportant. In this way, metaphysics, which ought to be the science of being, is paradoxically transformed into the science that \emph{abstracts} from being, concentrating on what is possible. We will see that this tendency has been the dominant one in Western philosophy: Thomas Aquinas, who masterfully stitched Aristotle’s science of \emph{ens qua ens} together with Pseudo-Dionysius the Areopagite’s notion of perfection as intensive act, was practically unique in proposing \emph{esse} as the single original act (what we will call \emph{esse ut actus}) that communicates itself and flows forth to all the actuality (what we will call \emph{esse in actu}) found in a substance. Unless there is this flow, it is difficult to understand the relationship between \emph{esse} and \emph{agere}: \emph{ens} appears to be static, and \emph{agere} appears as completely \emph{sui generis}. Such a separation, aside from being philosophically unsatisfying (requiring, as it does, a \emph{reductio ad duo}), is also dangerous: if \emph{agere} (which includes \emph{operari}, the domain of human acts) is completely independent of the domains of \emph{esse} and \emph{essentia}, then the natural law,%
%
\footnote{In this paper, the term \emph{natural law}, in accord with its common usage in English, will always refer to the natural \emph{moral} law proper to rational creatures. To refer to the analogous but deterministic reality in non-rational creatures, the term term \emph{physical law} is ordinarily employed.}
%
if the philosophy in question admits of one, must be imposed extrinsically on rational beings. In this way, ethics can degenerate into a set of rules or norms to follow, and it is difficult to see their intrinsic goodness. In reality, however, it seems to me that we can demonstrate a profound analogy and homogeneity between \emph{esse} and \emph{agere}. \emph{Agere sequitur esse}, not only by an analogy of proportionality, but more importantly by an analogy of reference, because \emph{esse ut actus} is the source of all actuality.

The problem that this paper will discuss, therefore, can be summed up as follows: why does \emph{ens} always seek its own perfection? Why does \emph{omne agens} always \emph{agit propter finem}? Is there an answer to this problem on the \emph{intrinsic} level, or must we make recourse to the extrinsic causes? Our investigation into the answer will be divided into three main parts: a \emph{resolutio secundum rationem} of \emph{bonum} to its underlying reality, \emph{esse} (\autoref{chap:bonum}); a detailed study of the intrinsic constitution of \emph{ens}, including a \emph{resolutio secundum rationem} of \emph{ens} to its original act, \emph{actus essendi}, and (working in reverse) a \emph{compositio} of \emph{ens} to see how it proceeds from its ultimate causes to the various layers of actuality that it possesses, ending with a section justifying our characterization of \emph{actus essendi} as \emph{virtus essendi} (\autoref{chap:ens_resolutio}); and finally, an investigation, based on what we have learned about \emph{bonum} and \emph{actus essendi}, into the intrinsic dynamism of \emph{ens} (\autoref{chap:dynamism}). Chapter \ref{chap:bonum} and \autoref{chap:ens_resolutio} both begin with a succinct history of each chapter’s topic of discussion (\emph{bonum} as a transcendental “property” of \emph{ens} and \emph{esse} as its original act). The authors are chosen to so as to show the ones that most heavily influenced Thomas Aquinas regarding our problem (especially Aristotle and Pseudo-Dionysius the Areopagite), as well as those that can be seen as models for how to interpret \emph{actus essendi}: the metaphysics of essence; what I call the “metaphysics of dual act,” which includes classical or Neo-Scholastic Thomism; and what I will call the “metaphysics of action,” represented by M. Blondel and transcendental Thomism. We will discover the answer to our problem in the intrinsic dynamism of \emph{ens}: the very structure of a \emph{suppositum}---an original \emph{actus essendi}, measured by an \emph{essentia ut potentia essendi}, that flows forth into three levels of \emph{esse in actu}---necessarily entails also an intrinsic \emph{ordo ad finem}. In short, a \emph{suppositum} is obliged, by metaphysical necessity, to seek the actuation of the “potential” (\emph{virtus}) that is present in its \emph{actus essendi}.