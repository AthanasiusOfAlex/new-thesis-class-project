% !TEX TS-program = xelatex
% !TEX encoding = UTF-8 Unicode
% !BIB TS-program = biber

\chapter{The Intrinsic Dynamism of \emph{Ens}}
\label{chap:dynamism}

As we saw in our \emph{compositio} of the principles of \emph{ens}, God exercises a threefold causality: inasmuch as his creatures \emph{proceed} from him, he is their efficient and exemplary cause; inasmuch as they \emph{return} to him, he is their final cause. We can say that, like any \emph{agens}---although naturally more perfectly than any created \emph{agens}---God creates (gives or communicates \emph{esse}), while discerning what he creates (using his own Essence as a divine idea or exemplar to co-create the essence), and keeping in mind both his own end (his glory) and the end of the thing created (union with him).%
%
\footnote{God, of course, cannot in any sense be perfected: his glory is manifested to his creatures.}
%
We have discovered in our investigation that a creature's \emph{actus essendi} mediates between God's extrinsic causality and the internal communication of \emph{esse}: it serves, as we saw, as a \emph{suppositum}'s \emph{virtus essendi}, the source of all its actuality. Morever, a creature's essence mediates between the \emph{actus essendi} (\emph{esse ut actus}) and its various predicamental actuations (\emph{esse in actu}).  Inasmuch as it communicates itself, \emph{esse ut actus} is the efficient cause of \emph{esse in actu}; inasmuch as \emph{esse in actu} is modeled after its source and measured by the essence, \emph{esse ut actus} is an exemplary cause.  Therefore, a \emph{suppositum} seems to be a microcosm or ``imitation'' of at least two of the extrinsic causes of \emph{ens}. The task of this chapter is to show that the intrinsic causes of \emph{ens} also include a \emph{final} causality, in which \emph{actus essendi} serves as both source and end: the \emph{actus essendi} seeks to expand into \emph{esse in actu}, and conversely the \emph{esse in actu} seeks \emph{actus essendi} as its end, seeking to ``return'' to its principle. In this way, the \emph{exitus} of creatures from their Creator and their \emph{reditus} to him is represented in and communicated to the very intrinsic dynamism of \emph{ens}; God communicates not only his \emph{esse} and essence, but also his \emph{operari}. In order to see this more clearly, we will look in detail at the intrinsic dynamism of \emph{ens}, in which the various levels of \emph{esse in actu} proceed or ``expand'' from \emph{actus essendi}, as measured by \emph{essence}, and in that way bring \emph{ens} to its final fulfillment or ``happiness.'' 

\section{The Origin and Inherence of Accidents}

\subsection{\emph{Aporia} of the Richness and Poverty of \emph{Ens}}
\label{sec:aporia}

In order to investigate the intrinsic dynamism of \emph{ens}, we can begin with an \emph{aporia} regarding \emph{esse} and and the \emph{operari} that flows from it, which W.N. Clarke sums up well:
%
\begin{quotation}
Aquinas [speaks] of an intrinsic dynamism in every being to be self-communica\-tive, to share its own goodness with others, to pour over into the production of another actuality in some way like itself. This is what Maritain has aptly called ``the basic generosity of existence.''

It follows that, for Aquinas, finite, created being pours over naturally into action for \emph{two} reasons: (1) because it is \emph{poor}, i.e., lacking the fullness of existence, and so strives to enrich itself as much as its nature allows from the richness of those around it; but (2) even more profoundly because it is \emph{rich}, which it tends naturally to communicate and share with others.%
%
\footnote{\Cite[605]{clarke:person-being}. See also \cite[163--164]{definance:etre-et-agir}: ``Fils de Poros et de Pénia, le désir naît de cette opposition même de l'acte et de la limite. Et c'est pourquoi toute forme créée s'accompagne d'une inclination qui, est l'appétit naturel.''}
\end{quotation}
%
In other words, the \emph{actus essendi} is rich, inasmuch as it tends to ``overflow'' into the different levels of \emph{esse in actu} by an ``excess'' of \emph{virtus}.%
%
\footnote{For another point of view regarding act and its interaction with limit, see \cite{kambembo:essai}, especially 377--387: ``Où enraciner le dynamisme de l'être, et en particulier de l'être limité~? Est-ce dans la forme (principe actif), est-ce dans la matière (principe potentiel)~? Ou faudrait-il, au contraire, l'enraciner dans le composé de matière et de forme, d'essence et d'existence, de principe actif et de principe potentiel, c'est-à-dire dans l'acte d'être (energeia)~? La thèse du P. de Finance nous semble bloquer le dynamisme de l'être dans la seule forme (actus essendi), principe actif de l'être. Chez lui, la limitation de l'«~esse~» par l'essence serait à la source du dynamisme de l'être.'' We can now answer this objection by noting that the \emph{actus essendi} cannot be identified with the form (which is, rather, \emph{esse substantiale} or \emph{esse in actu}), and that the radical potential principle is not matter but \emph{essentia ut potentia essendi}.}
%
On the other hand, it is poor inasmuch as it ``depends'' of the \emph{esse superadditum} of the accidents and \emph{operari} in order to reach its fulfillment. In reality, it is the same \emph{aporia} posed (at least implicitly) by Heidegger when he says that \emph{Sein} “discloses itself” in \emph{Seiendes} while ``withdrawing'':%
%
\footnote{For a discussion of Heidegger's doctrine on the concealment and unconcealment of being, see \cite[200--203]{contat:confronto}.}
%
inasmuch as \emph{Seiendes} reveals \emph{Sein} it is rich; inasmuch as it conceals \emph{Sein} it is poor.
Aristotle described a similar reality when elucidating the intrinsic principles of physical \emph{ens}: matter, form and privation (\gk{στέρησις}):%
%
\footnote{See \cite[160--163]{definance:etre-et-agir}.}
%
privation, which is what Thomas would call a \emph{per accidens} principle (not really distinct from matter),\footcite[9]{st:deprincipiis} means in essence having the capacity to receive a form, but not having it \emph{in actu}. Hence, the substantial form is rich inasmuch as it is in act; but it is poor inasmuch as it is ``deprived'' of the perfections that the accidental forms can bestow on it.
%
When we frame the \emph{aporia} in this way, we see that it shows forth another curious characteristic of substance: it is both \emph{in actu} and \emph{in potentia} with respect to the \emph{accidentia propria} that inhere in it. The accidents receive their act from the substance and inhere in it, but they are also in a different way the ``act'' and ``perfection'' of the substance.

This \emph{aporia} can be resolved by making use of the distinctions we have made, between intensive and extensive quantity and above all between \emph{esse ut actus} and \emph{esse in actu.} We will make use, in this section, of \emph{quaestio} 77 of the \emph{prima pars}, article 6, as a basis for resolving this \emph{aporia}. Although it deals directly with the \emph{accidentia propria}, we will be able to make use of the conclusions drawn from it to shed light on our problem, which has to do more closely with \emph{operari}.

\begin{DONE}

The idea is to show that \emph{esse ut actus} and \emph{essentia} necessarily entail an \emph{ordo ad finem} (which is the maximization of \emph{esse in actu}. Hence, there is an internal dynamism (a microcosm of God's causality).

  \item Begin with the \emph{aporia} of the ``rich'' being that is also ``poor'' (``contracted'' and ``expanded'' at the same time). Resolved with the difference between \emph{intensive} and \emph{extensive} quantity. See 

\end{DONE}

\subsection{Thomas Aquinas Regarding the Origin of the Powers}
\label{sec:origin-of-powers}

\emph{Quaestio} 77 regards the potencies (or ``powers'')%
%
\footnote{The powers of a substance are, according to Thomas, qualities of the second species. For the fourfold division of quality in Aristotle, see \cite[8b25-10a14]{aristotle:categories}; for the same division in Aquinas, see \cite[I-II, q.~49, a.~2, co.]{st:summa}. Thomas may have corrected Aristotle slightly on this point. Aristotle seems to regard the second-species as ``abilitities'' such as being good at boxing. (See \cite[9a14-17]{aristotle:categories}.) Aquinas, however, would say that faculties such as the intellect and the will would also be included. For example, speaking of powers of the soul (\emph{potentiae animae}), he says, ``potentia est in secunda specie qualitatis''  \parencite[I-II, q.~56. a.~1, arg.~3]{st:summa}.} of the soul and what belongs to them in general, and article 6 specifically has to do with whether these powers flow from the essence. Thomas' answer is affirmative, despite three objections: the (supposed) impossibility of manifold powers to flow from a simple essence; the impossibility that essence could be the cause of the powers, and the impossibility of emanation (which would require movement) from the immobile soul. Regarding the first objection, Thomas says,
%
\begin{quotation}
Ad primum ergo dicendum quod ab uno simplici possunt naturaliter multa procedere ordine quodam. Et iterum propter diversitatem recipientium. Sic igitur ab una essentia animae procedunt multae et diversae potentiae, tum propter ordinem potentiarum, tum etiam secundum diversitatem organorum corporalium.\footcite[I, q.~77, a.~6, ad 1]{st:summa}
\end{quotation}
%
In this way, he makes use of the notion of what we have called \emph{Diremtion}: a simple cause may produce multiple effects, provided there is an order (which implies a \emph{modus} or ``measure'') and a \emph{subiectum} available to receive the effect. Otherwise, there would be no way to distinguish the participated from the cause that communicates it.

Regarding the second objection, Thomas simply denies the major premise:
%
\begin{quotation}
Ad secundum dicendum quod subiectum est causa proprii accidentis et finalis, et quodammodo activa; et etiam ut materialis, inquantum est susceptivum accidentis. Et ex hoc potest accipi quod essentia animae est causa omnium potentiarum sicut finis et sicut principium activum; quarundam autem sicut susceptivum.\footcite[I, q.~77, a.~6, ad 2]{st:summa}
\end{quotation}
%
In fact, he argues, the \emph{subiectum} (in this case, the essence) is indeed the cause of the proper accidents, in three ways: as final, ``active'' (that is, quasi-efficient), and quasi-material cause. The last two are rather straightforward: it is the essence (\emph{in act}) that communicates its \emph{esse} (also \emph{in act}) to the \emph{propria}, which makes it the ``active'' cause; it is also a sort of material cause inasmuch as the accidents relate to the substance as potency to act. To show that the essence is the final cause, we must make recourse to the maxims \emph{omne agens agit propter finem} and \emph{actiones sunt suppositorum} (which are justified by our \emph{resolutio} of \emph{bonum} in \autoref{chap:bonum}): the substance emanates the \emph{propria} in order to reach its own fulfillment. It is the task of the sections that follow to elucidate more clearly the metaphysical foundation for this principle.

The third objection is simpler to answer: 
%
\begin{quotation}
Ad tertium dicendum quod emanatio propriorum accidentium a subiecto non est per aliquam transmutationem; sed per aliquam naturalem resultationem, sicut ex uno naturaliter aliud resultat, ut ex luce color.\footcite[I, q.~77, a.~6, ad 3]{st:summa}
\end{quotation}
%
It is sufficient to recall that, \emph{per se}, the cause is not altered by its effect; it is only in material compounds, which are simultaneously affected by the \emph{patiens}, that there can be a \emph{transmutatio}.%
%
\footnote{See, for example, \cite[I-II, q.~22, a.~1, ad~1]{st:summa}: ``pati, secundum quod est cum abiectione et transmutatione, proprium est materiae, unde non invenitur nisi in compositis ex materia et forma.''} The causality that the substance (or \emph{essentia in actu})  exercises on the accidents is not, therefore, like that of two interacting matter-form compounds, but is a true communication of \emph{esse}.

Turning to the central problem, since Thomas is attempting to resolve the origin of the powers of the human soul, the substances that he is dealing with are, naturally, human beings. In this context, he notes that substantial form and accidental form coincide in that both are ``act'':
%
\begin{quotation}
Respondeo dicendum quod forma substantialis et accidentalis partim conveniunt, et partim differunt. Conveniunt quidem in hoc, quod utraque est actus, et secundum utramque est aliquid quodammodo in actu.\footcite[I, q.~77, a.~6, co.]{st:summa}
\end{quotation}
%
In other words, both reduce a potency to act. Of course, they also differ, and for Thomas, the difference has to do with which principle---active or potential---causes the other:
%
\begin{quotation}
Primo quidem, quia forma substantialis facit esse simpliciter, et eius subiectum est ens in potentia tantum. Forma autem accidentalis non facit esse simpliciter; sed esse tale, aut tantum, aut aliquo modo se habens, subiectum enim eius est ens in actu. Unde patet quod actualitas per prius invenitur in forma substantiali quam in eius subiecto, et quia primum est causa in quolibet genere, forma substantialis causat esse in actu in suo subiecto.\footcite[I, q.~77, a.~6, co.]{st:summa}
\end{quotation}
%
We saw above in what sense the substantial form makes a substance be \emph{simpliciter}: without the substantial form, the substance simply ceases to exist, \emph{tout court}. On the other hand---as we also saw---an accidental form is \emph{superadditum} and hence  only makes a substance be \emph{secundum quid} (or equivalently \emph{esse tale aut tantum}). It is obvious, argues Thomas, that \emph{actualitas} is to be found in the substantial form before it is found in its \emph{subiectum} (which in this case is the prime matter).

Based on what we have discussed above, and using the very words of Thomas, we can give a name to that actuality of the substantial form: it is the substance's \emph{esse in actu}, which it communicates to the matter. That the matter should thus participate in the \emph{esse in actu} given to it by the substantial form follows from the maxim \emph{primum est causa in quolibet genere}---a principle we first saw in section \ref{arguments-real-composition} in the arguments for the real composition---which recalls similar language found in the \emph{quarta via} and in other places: ``Quod autem dicitur maxime tale in aliquo genere, est \emph{causa omnium quae sunt illius generis}, sicut ignis, qui est maxime calidus, est causa omnium calidorum.''\,%
%
\footnote{\Cite[I, q.~2, a.~3]{st:summa} (emphasis added).}
%
In other words, whatever has a perfection \emph{per participationem} must receive it (at least ultimately) from something that possesses it \emph{per se} (which is the \emph{primum} in the \emph{genus}). In this case, the perfection communicated is not the \emph{esse ut actus}---because as we saw the substance itself must participate in an actuality superior to it---but its \emph{esse in actu}, what Thomas calls in other places the \emph{esse substantiale}.

For accidental form, on the other hand, the situation is reversed: actuality is to be found first in the \emph{subiectum}, even though it functions as the ``matter,'' or potential principle, of the accidental form. Hence, the actuality of the accidental form (its own \emph{esse in actu}, or what Thomas calls \emph{esse accidentale}) is caused by that of the \emph{subiectum}:
%
\begin{quotation}
Sed e converso, actualitas per prius invenitur in subiecto formae accidentalis, quam in forma accidentali, unde actualitas formae accidentalis causatur ab actualitate subiecti. Ita quod subiectum, inquantum est in potentia, est susceptivum formae accidentalis, inquantum autem est in actu, est eius productivum. Et hoc dico de proprio et per se accidente, nam respectu accidentis extranei, subiectum est susceptivum tantum; productivum vero talis accidentis est agens extrinsecum.\footcite[I, q.~77, a.~6, co.]{st:summa}
\end{quotation}
%
Thomas does not say so explicitly, but this \emph{subiectum} can be none other than the \emph{essentia in actu},%
%
\footnote{Above, however, he does say ``subiectum enim eius [formae accidentalis] est ens in actu.''} which serves as the substrate for the accidents. This \emph{subiectum}, paradoxically, is both \emph{in actu} and \emph{in potentia}---evidently in different respects---because it is the substrate that receives the accidental forms, and it also produces them. This duality applies, of course, only to \emph{accidentia propria and per se}---those that find their origin in the substance, not from outside.%
%
\footnote{Returning to our example of the \emph{calidum}, fire possesses heat \emph{proprie et per se}, but iron only \emph{per accidens}, because it does not \emph{produce} the heat. In a similar way, only God can create because only he possesses \emph{esse proprie and per se}.

The analogy of the \emph{maxime calidum} in Thomas' \emph{quarta via} is frequently critiqued as if it depended on the limited scientific knowledge of the day, which considered fire as an element and as the principle of all heat (hence the hottest substance possible). For example, De Finance says, ``Il ne faut donc pas donner à un simple exemple une portée qu'il n'a jamais eue dans la pensée de saint Thomas. La comparaison du feu nous paraît simplement destinée à suggérer la correspondance entre le degré de perfection d'un être et son rayon d'influence. Elle amorce le mouvement intellectuel qui doit conduire à la causalité universelle de Dieu.'' \parencite[125]{definance:etre-et-agir}. It is true that modern science has challenged the Aristotelian concept of fire---we know now that it is not a substance (in fact, it is an aggregate of rapidly changing substances), much less an element, and that there is no ``upper limit'' to temperature. In my opinion, Thomas' argument does not depend on the Aristotelian theory of elements, because by Thomas' own reasoning, an absolutely \emph{maxime calidum} would have to be \emph{calidum subsistens}, which is impossible. In reality, the key similarity between \emph{esse} and \emph{calidum} is that each is a \emph{proprium} and the \emph{maximum} of its genus. Therefore, it seems to me, the analogy of the \emph{maxime calidum} is metaphysically exact.}

The second difference between the two types of form regards final cause:
%
\begin{quotation}
Secundo autem differunt substantialis forma et accidentalis, quia, cum minus principale sit propter principalius, materia est propter formam substantialem; sed e converso, forma accidentalis est propter completionem subiecti.\footcite[I, q.~77, a.~6, co.]{st:summa}
\end{quotation}
%
In other words, whereas the prime matter is for the sake of the substantial form, conversely, the accidental form is for the sake of the substance. There is, therefore, a  correspondence of efficient and final causes, as summarized in table \ref{tab:powers-of-soul}.
%
\begin{table}
  \centering
  \begin{OnehalfSpacing}
    \begin{tabular}{lll}
      \toprule
                           & \textbf{Substantial Form} & \textbf{Accidental Form} \\
      \midrule
        \emph{Subiectum}   & Prime matter              & \emph{Essentia in actu (ut limitata)}  \\
        \emph{Esse}        & \emph{Simpliciter}        & \emph{Tale aut tantum}   \\
        Actuality          & Prior                     & Posterior                \\
        Causes \emph{esse in actu} in & Prime matter   & \hspace{6pt}---          \\
        Is produced by     & \hspace{6pt}---           & \emph{Essentia in actu}  \\
        Is for the sake of & \hspace{6pt}---           & \emph{Essentia in actu}  \\
      \bottomrule
    \end{tabular}
  \end{OnehalfSpacing}
  \caption{Subtantial form and accidental form in I, q.~77, a.~6}
  \label{tab:powers-of-soul}
\end{table}
%

The powers, argues Aquinas, can inhere either directly in the soul itself, or else in the compound of body and soul:
%
\begin{quotation}
Manifestum est autem ex dictis quod potentiarum animae subiectum est vel ipsa anima sola, quae potest esse subiectum accidentis secundum quod habet aliquid potentialitatis, ut supra dictum est; vel compositum.\footcite[I, q.~77, a.~6, co.]{st:summa}
\end{quotation}
%
Thomas is thinking here especially of the intellect and will, which have only the soul as their \emph{subiectum}, and the faculties for sensitive knowledge and appetite, which inhere in the body-soul compound.%
%
\footnote{This distinction is explained in \cite[I, q.~77, a.~1 and 5]{st:summa}. Article 1 proves that essence and power are really distinct in creatures, and article 5 proves that the sensitive faculties cannot inhere directly in the soul, since they depend on matter.} However, since the soul is the source of the actuality in the matter---and hence the compound---the soul must be the  ultimate principle even of the sensible powers:%
%
\footnote{See \cite[I, q.~77, a.~5, ad~1]{st:summa}: ``omnes potentiae dicuntur esse animae, non sicut subiecti, sed sicut principii, quia per animam coniunctum habet quod tales operationes operari possit.''}
%
\begin{quotation}
Compositum autem est in actu per animam. Unde manifestum est quod omnes potentiae animae, sive subiectum earum sit anima sola, sive compositum, fluunt ab essentia animae sicut a principio, quia iam dictum est quod accidens causatur a subiecto secundum quod est actu, et recipitur in eo inquantum est in potentia.\footcite[I, q.~77, a.~6, co.]{st:summa}
\end{quotation}
%
From this, Thomas concludes that every power of the soul flows from a single principle; namely, the very essence of the soul (the \emph{essentia in actu}, in the terminology we have adopted). Being an accident, he says, a power is caused by the \emph{subiectum}, (and in that sense the \emph{subiectum} is \emph{in actu}), and received by the \emph{subiectum} (and in that sense this \emph{subiectum} is \emph{in potentia}).

\subsection{Resolution of the \emph{Aporia}}

In the context of article 6, Thomas is content to prove that the powers all proceed from the soul, and to leave the \emph{aporia} of the essence's simultaneous status as act and potency unanswered. However, using what we have discovered about the \emph{actus essendi}, we are now in a position to resolve it. In reality, this \emph{quaestio} is an application of the maxim \emph{forma dat esse}, which means that the form, being the active principle of the \emph{essentia in actu}, communicates its \emph{esse in actu} both to the matter (if there is any) and to the accidents (at least those that are \emph{per se et propria}). Hence we may rightly regard the form as the source of all the proper \emph{esse in actu} in a substance. If, however, the form were the substance's \emph{ultimate} source of actuality, the essence's relationship with the accidents would be contradictory, because the \emph{esse in actu} of the substance would be the only basis for the composition, but ``Non autem est possibile ut idem sit simul in actu et potentia secundum idem, sed solum secundum diversa.''\,\footcite[I, q.~2, a.~3, co.]{st:summa}

Our distinction between \emph{esse ut actus} and \emph{esse in actu} is what provides the \emph{diversa} that we need to resolve the paradox. In reality, the form (and by extension the essence) functions as a \emph{mediator} between the \emph{esse ut actus} and the \emph{esse in actu} of the substance. If we look at what we can call the ``transcendental'' composition between \emph{esse ut actus} and \emph{essentia ut potentia essendi}, ``before'' the substance is constituted, we see that \emph{essentia} is clearly receptive and hence \emph{in potentia}. If we look at the same essence, as it is ``after'' it is constituted, we see that it is already \emph{in actu}, thanks to the \emph{esse in actu} that it has received from the \emph{esse ut actus}. This \emph{esse in actu} (what Thomas calls \emph{esse substantiale}) is not the same as \emph{esse ut actus}, but the fruit of a \emph{contractio}. The essence is, therefore, in sense ``bivalent'': on the ``transcendental'' level, it is receptive of the \emph{esse ut actus}; on the ``predicamental'' level it possesses \emph{esse in actu}, which is a participation in the \emph{esse ut actus}, but not identical with it. The fully constituted, ``predicamental'' essence is, in turn capable of re-transmitting that \emph{esse in actu} to other entities (namely, the \emph{accidentia propria}). This, then, is the answer to our \emph{aporia}: essence---on the predicamental level, is \emph{in actu} inasmuch as it has received its \emph{esse in actu}; it is \emph{in potentia} inasmuch as it is capable of transmitting that \emph{esse}. It is indeed simultaneously \emph{in actu} and \emph{in potentia}, but \emph{secundum diversa}.

\subsection{Application to Our Problem}

Applying what we learned about the powers of the human soul to all substances, we see that the \emph{accidentia propria et per se}, proceed from the substance (or essence) in act by means of emanation, ``powered,'' as it were, by the \emph{actus essendi}, which functions as a \emph{virtus essendi}. In the case of the \emph{propria}, the emergence is necessary; they cannot help but come forth. This insistent tendency to ``expand'' indicates how the \emph{actus essendi} is not content, so to speak, to remain as a \emph{virtus essendi}. As we said, strictly speaking, it is not \emph{in actu}, but the \emph{source} of actuality, or that \emph{by which} the substance is \emph{in actu} (\emph{quo agit}). Therefore, it ``desires'' to reach its ``fulfillment'' in the various levels of \emph{esse in actu}. The principle of causality (namely, that potency is reduced to act only by something in act) implies that every agent produces an effect that is similar to itself: ``Unumquodque agens hoc modo agit secundum quod similitudo facti est in ipso: omne enim agens agit sibi simile.''\,%
%
\footnote{\Cite[II, cap.~23, n.~4 (Marietti n.~992)]{st:contragent}. See also \cite[II, cap.~24, n.~3 (Marietti n.~1004)]{st:contragent}: ``Omne agens agit sibi simile. Unde oportet quod secundum hoc agat unumquodque agens secundum quod habet similitudinem sui effectus.''}
%
It follows that a substance (aside from God) cannot simply perform operations directly without making use of a faculty or active potency (power): the substance merely ``is,'' and it does not have the power to communicate its \emph{esse} to other substances. Therefore, in order to perform a specific action, it must have an active potency or power adequate to that action that mediates between the substance and the operation.%
%
\footnote{Regarding active potencies, see \cite[244--246]{definance:etre-et-agir}; also \cite[212--219]{izquierdo:vita}.}
%
Just as the essence receives \emph{esse substantiale} and re-transmits it, the power receives \emph{esse accidentale} so as to re-transmit it as   \emph{operari}.

We are now in a position to elucidate the various levels of act and potency that can be found in any \emph{suppositum}.%
%
\footnote{For the levels of \emph{esse in actu} and their corresponding potencies, see \cite[57--59]{contat:esse-essentia-ordo}.} The ``levels'' can be described in a ``descending'' direction, following the communication of act, or else they can be described in ``ascending'' order, specifying the \emph{subiectum} that each \emph{esse in actu} reduces to act. With respect to act, therefore, God---\emph{Ipsum Esse}---creates a substance, endowing it with an original and radical \emph{actus essendi} (that we have called \emph{esse ut actus}), which functions as the \emph{virtus essendi} that is not yet \emph{in actu} but from which flows all the actuality in the substance. This \emph{actus essendi}---according to the measure given to it by the essence as \emph{potentia essendi}---communicates itself as \emph{esse substantiale}, the \emph{esse in actu} of the substance. The substance, inasmuch as it is \emph{in actu} (\emph{essentia in actu}) then communicates its own \emph{esse in actu} to its \emph{accidentia propria}, endowing them with \emph{esse accidentale}.%
%
\footnote{Accidents that are are not \emph{propria et per se} receive their \emph{esse accidentale} from whatever agent reduces them from potency to act, as fire heats iron.} Some of these \emph{accidentia}---the powers, or active potencies---are themselves only partially \emph{in actu} and hence are capable of communicating their \emph{esse in actu} once more in the form of specific \emph{operari}.

With respect to \emph{subiectum}, the \emph{operari} are in one sense produced by the power in question---it is the active potency that specifies the action produced, mediating between substance and   \emph{operari}---but in another, more profound sense, are produced by the \emph{suppositum}, according to the maxim \emph{actiones sunt suppositorum}:
%
\begin{quotation}
Actiones autem sunt suppositorum et totorum, non autem, proprie loquendo, partium et formarum, seu potentiarum, non enim proprie dicitur quod manus percutiat, sed homo per manum; neque proprie dicitur quod calor calefaciat, sed ignis per calorem.%
%
\footnote{\Cite[II-II, q.~58, a.~2, co.]{st:summa}. This maxim is repeated in many places in the \emph{Corpus Thomisticum}; for example, \cite[I, q.~39, a.~5]{st:summa}, ``ea quae pertinent ad actum, magis propinque se habent ad personas, quia actus sunt suppositorum;'' and in the context of Christology, \cite[20, a.~1, arg.~2]{st:deveritate}, ``operatio non attribuitur naturae, sed hypostasi; sunt enim operationes suppositorum et particularium.''}
\end{quotation}
%
The maxim is justified precisely because of what we have learned: the \emph{operari} flow from the \emph{essentia in actu}, and (since the form, the active principle of the essence, cannot be the original active principle) ultimately the \emph{esse ut actus}. As Thomas puts it, in order to refute Monoenergism:
%
\begin{quotation}
Quia vero operationes suppositorum sunt, visum est aliquibus quod sicut in Christo non est nisi unum suppositum, ita non esset nisi una operatio. Sed non recte consideraverunt: nam in quolibet individuo reperiuntur multae operationes, si sunt plura operationum principia, sicut in homine alia est operatio intelligendi, alia sentiendi, propter differentiam sensus et intellectus: sicut in igne alia est operatio calefactionis, alia ascensionis, propter differentiam caloris et levitatis. Natura autem comparatur ad operationem ut eius principium.\footcite[I, cap.~212]{st:compendium}
\end{quotation}
%
A \emph{suppositum} produces many actions, but there is only one \emph{suppositum} that acts, and all of the \emph{operari} flow from the nature. Therefore, the \emph{subiectum} of the operations is the \emph{suppositum in actu}. In turn, the \emph{accidentia propria et per se} (and any other accident that inheres in the substance) have the substance (the \emph{essentia in actu}) as their \emph{subiectum}; not, however, inasmuch as it is \emph{act}, but inasmuch as it is \emph{limited} (as we saw regarding the spiritual substances). The (\emph{esse substantiale}) also has the essence as its \emph{subiectum}, but this time inasmuch as it is a potency apt to receive a specific kind of \emph{esse in actu}. The various levels are summarized in table \ref{tab:levels-esse-in-actu}.%
%
\footnote{This table is adapted from a similar one found in \cite[58]{contat:esse-essentia-ordo}.}
%
\begin{table}
  \centering
  \begin{OnehalfSpacing}
    \begin{tabular}{lll}
      \toprule
        \textbf{Level} & \textbf{Potential Principle} & \textbf{Active Principle}     \\
      \midrule
        Transcendental& \emph{Essentia ut potentia essendi} & \emph{Esse ut actus}    \\
      \midrule
        \multirow{3}{*}{\pbox[c]{\textwidth}{Predicamental\\(\emph{esse in actu})}%
                        \hspace{18pt}$\left\lbrace{\rule[-19pt]{0pt}{48pt}}\right.$%
                        \hspace{-3pt}}
                      & Essence as specifying potency       & \emph{Esse substantiale}\\
                      & Essence as formal, limited \emph{ens in actu}
                                                            & \emph{Esse accidentale} \\
                      & \emph{Suppositum in actu primo}
                                                    & \emph{Operari in actu secundo}\\
      \bottomrule
    \end{tabular}
  \end{OnehalfSpacing}
  \caption{Levels of \emph{esse in actu} in a \emph{suppositum}}
  \label{tab:levels-esse-in-actu}
\end{table}
%
In this way, we see that each successive level entails a diminution or \emph{contractio} of the intensity of the \emph{esse in actu} that is communicated and received. On the other hand, as the \emph{actus essendi} communicates \emph{esse in actu} to successive levels, the \emph{suppositum} passes from having only the \emph{active potential} for being (\emph{virtus essendi}) to having \emph{actual} being (\emph{esse in actu}), and in this sense the substance experiences an expansion of \emph{esse}---an expansion that we could characterize as ``extensive,'' because each new \emph{esse in actu} is \emph{superadditum}, in the sense we described above in section \ref{sec:intensive-extensive}. Therefore, this distinction between the ``intensive'' and ``extensive'' dimensions of \emph{esse} effectively resolves the \emph{aporia} formulated by W.N. Clarke above (section \ref{sec:aporia}).

The conjuction of ``richness'' and ``poverty,'' moreover, occurs at every level, because every level participates in the superior one by means of a mediation. No level acts on its own, until it has received \emph{esse} from the one above it: the \emph{actus essendi} mediates between the creature (the \emph{suppositum}) and its creator (\emph{Ipsum Esse}); the essence mediates between the \emph{actus essendi} and the \emph{accidentia propria}; the powers or active potencies mediate between the essence in act and the \emph{operari}. Here we discover a profound analogy between \emph{actus essendi} and operative powers. Both contain their act virtually: \emph{actus essendi} and \emph{potentia activa} are therefore rightly called \emph{virtus essendi} and \emph{virtus operandi}, the former \emph{in actu primo}, the latter \emph{in actu secundo}.

Thomas' indebtedness to Neoplatonist philosophers shows forth in his theory of the intrinsic structure of \emph{ens}: as summarized in table \ref{tab:neoplatonism-thomas}, Plotinus' theory of cosmogenesis is structurally very similar to the theory of \emph{actus essendi}, each its respective domain. To Plotinus' unknowable and ineffable One corresponds the \emph{actus essendi}, which before its expansion into \emph{esse in actu} is only \emph{virtus essendi}, the \emph{source} of act; to the intelligible \gk{Νοῦς}, which in a certain sense measures the One and ``partitions'' it into the many, corresponds the \emph{essentia ut potentia essendi}, which limits and measures the \emph{actus essendi} and permits the first \emph{Diremtion}; and to the Soul (\gk{Ψυχή}) corresponds the various levels of \emph{esse in actu} found in the \emph{accidentia propria} and \emph{operari}. As in Plotinus, each level mediates the communication of actuality from the one above it; hence what Plotinus (erroneously) applied to extrinsic causes, Thomas brilliantly applied to the \emph{intrinsic} causes of \emph{ens}.
%
\begin{table}
  \centering
  \begin{OnehalfSpacing}
    \begin{tabular}{lll}
      \toprule
         \textbf{Neoplatonism} (Cosmogenesis) & \textbf{Thomas} (Ontogenesis) \\
      \midrule
         One: ineffable, unknowable & \emph{Esse ut actus}: \emph{virtus essendi} \\
         \gk{Νοῦς}: measure and mediator (demiurge)
                                    & \emph{Essentia}: measure and mediatrix \\
         \gk{Ψυχή}: contemplator and agent & \emph{Esse in actu} and \emph{operari}  \\
      \bottomrule
    \end{tabular}
  \end{OnehalfSpacing}
  \caption{The Neoplatonic processions and the procession
           of \emph{esse} within \emph{ens}.}
  \label{tab:neoplatonism-thomas}
\end{table}
%


\begin{DONE}

\begin{itemize}
\item The \emph{propria} are necessary, hence an ``exigency'' of the \emph{esse ut actus}.

\item Guess what! Active potencies work in just the same way!

\item Each level of \emph{esse in actu} is necessary for the ``fulfillment'' of the \emph{esse ut actus} which ``is'' not.

  \item Applies to any accidens proprium.
  \item \emph{Propria} must arise from an overabundance of \emph{virtus}
  \item  The substance does not \emph{exhaust} the virtus, so it expands.
  \item How relates to operation? Because a power acts as another mediator. (We need a mediator, because \emph{omne agens agit simili sibi}, and we need something to go from potency to act.

\end{itemize}

\end{DONE}

\begin{DONE}

\section{Exegesis of \emph{S.Th.}, I, q. 77, a.~6, co.}

Both substantial form and accidental form are act (utraque est actus).

Difference in efficient and \enquote{material} cause.
The substantial form gives esse to its subiectum; the accidental form receives esse from its subiectum.

\emph{Forma substantialis facit esse simpliciter}.
Its subiectum is pure potency (i.e., prime matter) whereas \emph{accidental form facit [\ldots] esse tale}.

Note: Not all substantial forms have a \emph{subiectum} (understood as pure matter---e.g., angels). However, he is talking about man, so here the substantial form \emph{does} have a \emph{subiectum}.

\emph{Forma substantialis causat esse in actu}.
It is easier to find the actuality of substantial form in the form itself rather than in the \emph{subiectum} (i.e., matter): \emph{quia primum est causa in quolibet genere, forma substantialis causat esse in actu in suo subiecto}.

In this sense, \emph{forma dat esse}. \emph{Esse autem in actu, non esse ut actus!}

\emph{Actualitas formae accidentalis causatur ab actualitate subiecti}.
With accidental form, the actuality is easier to find in the \emph{subiectum} than the form. Conclusion: \emph{actualitas formae accidentalis causatur ab actualitate subiecti}.

Why? \emph{Quia primum est causa in quolibet genere}.

The \emph{subiectum} is both act and potency.

Thus, the \emph{subiectum} (of the accidental form—i.e., the substance) inasmuch as it is in potency, can receive the accidents; inasmuch as in act, \emph{produces} them.%
%
\footnote{Note that this only applies to \emph{propria}; accidents that are \emph{per accidens} are produced by an \emph{extrinsic} agent.}

Difference in final cause:

\emph{Materia est propter formam substantialem.}

\emph{Forma accidentalis est propter completionem subiecti.}

Conclusions of St.~Thomas:

The potencies inhere in soul or body-soul compound.

The potencies of a soul inhere either in the soul itself%
%
\footnote{Intellect, will.} or else in the compound of soul and body.%
%
\footnote{Sense faculties.}

The substance is \emph{in actu} thanks to the substantial form.
Even if it inheres in the compound, the compound (substance) is \emph{in actu} thanks to the soul (substantial form).

\emph{\textbf{Omnes potentiae animae fluunt ab essentia animae.}}

Hence, \emph{omnes potentiae animae [\ldots] fluunt ab essentia animae sicut a principio}. (Or more simply, yes, the soul is the principle of all the potencies.)

Remarks and implications regarding our problem:

Applies to any \emph{proprium}.
The argument used here applies equally well to any \emph{proprium}, not just potencies. The central argument regards \emph{propria} of any sort; the application to the potencies in particular is done only at the end.

Potencies that are the source of acts \ldots (in living beings, evidently).

In any case, it is the potencies that are the source of acts (\emph{operari}).

\emph{Forma dat esse}.
If the substantial form is the source of the \emph{esse} for all the \emph{propria}, and the potencies are the source of the   \emph{operari}, then \emph{a fortiori}, the substantial form must be the source of the \emph{esse} of the   \emph{operari}. (\enquote{\emph{Forma dat esse}.})

\textbf{\emph{Esse in actu} is not \emph{esse ut actus}}.

Note that the use of the term \emph{esse in actu}, as distinct from \emph{esse ut actus}, is justifiable thanks to this passage.

An unresolved \emph{aporia}:
How is it possible for the substance to be both \emph{in actu} and \emph{in potentia} at the same time?

\emph{Non autem est possibile ut idem sit simul in actu et potentia secundum idem} (\emph{S.Th.} Iª q. 2 a. 3 co.).

That is a reasonable and, I think, self-evident principle, once you understand the notions of act and potency.

The problem is unstated.
St.~Thomas states simply that the substantial form is both \emph{in actu} and \emph{in potentia}. He does not prove it, because at this point, he is interested only in the problem of the soul as the source of the potencies.

Substance is \emph{subiectum} and efficient cause.
In other words, substance is also both the “material” subiectum of the accidents (in general) and their efficient cause.

Solution. The solution lies in the duality of \emph{esse ut actus} and \emph{essentia}, and the priority and dynamism of the former.

Essence as mediator.
The essence has a role of mediation, in which it has a “transcendental” aspect (receiving and measuring—or determining—the esse ut act) and a “categorical” one (as one of the supreme genera of ens and source of the esse in actu).


\end{DONE}

\begin{DONE}

In sum:

\begin{itemize}

  \item As suggested by the comparison to Neoplatonism, with the \emph{exitus} (procession) there is also \emph{reditus} (return).

  \item Note the pattern of \emph{mediations}: actus essendi (God--essence); essence (actus essendi--esse in actu); active potencies (substance--operation) (connect with he neoplatonistic ``emanations'').

  \item This is resolved in the difference between \emph{intensive} and \emph{extensive} quantity, esse ut actus (intensive); \emph{esse in actu} (additive/extensive==\emph{esse superadditum}).

  \item Just as Ipsum Esse suffers (in a way) a \emph{Diremtion} into various tupes of \emph{actus essendi}, the \emph{actus essendi} suffers a \emph{Diremtion} into the particular \emph{esse in actu}. Hence \emph{actus essendi} is \emph{much greater} than the \emph{esse in actu}.

\end{itemize}
\end{DONE}


\section{\emph{Actus Essendi}: Efficient, Formal, and Final Cause}

If the \emph{esse in actu} of \emph{ens} proceeds from the \emph{actus essendi} in a fashion analogous to the way Plontius thought \emph{ens} proceeded from the One, then it seems reasonable that the \emph{esse in actu} should seek to \emph{return} to its origin. We saw the need for a \emph{reditus} when we briefly discussed \emph{Ipsum Esse} as the ultimate extrinsic cause of \emph{ens}: God, \emph{Ipsum Esse}, communicates his \emph{esse} to his creatures. In God, of course, there is no distinction between \emph{actus primus} and \emph{actus secundus}: he is not only fullness of \emph{virtus essendi} but fullness of \emph{esse in actu}. In addition, God is the exemplar of every perfection that is desirable; whatever one discovers as \emph{bonum} has its model in God, to an infinite degree of intensity, and already \emph{in actu}. It follows that God is the \emph{Summum Bonum} to which all things tend: all things find their ultimate perfection, their ultimate expansion into \emph{esse in actu} in Him, and it is precisely this union with the Creator that constitutes the \emph{reditus} of the creature to God. God is the \emph{Summum Bonum}, not only inasmuch as he is the ultimate ``object'' of desire, but also inasmuch as creatures are perfect to the degree that the imitate him.%
%
\footnote{This is a principle already well understood by Aristotle. See, for example, \cite[\gk{Ε}, 13, 1153b31--32]{aristotle:ethics}: \gk{ἴσως δὲ καὶ διώκουσιν οὐχ [τὴν ἡδονήν] ἣν οἴονται οὐδ᾽ ἣν ἂν φαῖεν, ἀλλὰ τὴν αὐτήν [ἡδονήν]: πάντα γὰρ φύσει ἔχει τι θεῖον}, ``Likewise, however, they also pursue not [the pleasure] that they think or whatever they say, but the same [pleasure]: for all by nature have something divine.'' (my translation). In other words, behind every pleasure experience is a foretaste of the pleasure experienced by God (which we know is above all that of contemplation, \gk{νόησις νοήσεως}).

Saint Thomas comments on this passage as follows: ``natura tamen omnes inclinat in eandem delectationem sicut in optimam, puta in contemplationem intelligibilis veritatis, secundum quod omnes homines natura scire desiderant. Et hoc contingit, quia omnia habent naturaliter in se ipsis quiddam divinum, scilicet inclinationem naturae, quae dependet ex principio primo; vel etiam ipsam formam, quae est huius inclinationis principium'' \parencite[lib.~7, l.~1, n. 14]{st:ethics}.}

A similar argument can be made for the \emph{intrinsic} principles of \emph{ens}, provide we make the similar adjustments. The \emph{esse ut actus} communicates itself to \emph{esse in actu}; both \emph{actus essendi} and the \emph{essentia ut potentia essendi} that measures it, are exemplars of \emph{esse in actu}: the former, inasmuch as it is act; the latter, inasmuch as it specifies that act (in analogy to the way God is the twofold exemplary cause of \emph{ens}, inasmuch as he is \emph{esse per essentiam} and also \emph{divine idea}). A created \emph{suppositum}, however, is different from God, because it has suffered \emph{Diremtion}:%
%
\footnote{Again, we must stress that \emph{Diremtion} is a \emph{positive} thing, because it is what permits the existence of creatures really distinct from their Creator.} since it is not \emph{esse per essentiam}, it possesses its \emph{actus essendi} as a distinct principle. Likewise, it receives its \emph{actus essendi} as a \emph{virtus essendi} that is not yet \emph{in actu} until it ``expands'' into \emph{esse substantiale} and \emph{accidentale}. As we saw, the \emph{virtus essendi} is so ``potent'' that it cannot help but expand in this way, at least so far as to produce the \emph{accidentia propria}. It is evident, however, that even \emph{entia} of the same species realize the ``potential'' of their \emph{virtus essendi} to differing degrees: for example, some oak trees survive to adulthood and manage to produce acorns, but others remain stunted or even die blighted. The more ``distant'' expansions of \emph{esse} (above all \emph{operari}, but also certain \emph{habitus} and \emph{dispositiones}) are in a certain sense ``facultative'': they may arise, and they are the natural result of the \emph{impetus} given by the \emph{actus essendi}, as measured by the essence, but they may be impeded. By being prevented from coming to fruition, they do not thereby stop being an exigency of the \emph{virtus essendi}: the essence measures which types of \emph{esse in actu} ``ought'' to be put into act, and if in fact they \emph{fail} to do so, the \emph{suppositum} experiences what could be termed ``frustration'' or ``stuntedness.'' The \emph{actus essendi}---itself not yet \emph{in actu}---as it were ``clamors'' for expansion into \emph{esse in actu}, and the \emph{suppositum} ``suffers'' if it does not reach its end. Again, W.N. Clarke sums up this idea very well:
%
\begin{quotation}
Thus it is proper to every being, insofar as it is in act, to overflow into action, to act according to its nature. [\ldots] The act of existence of any being (its \enquote{to be} or \emph{esse}) is its \enquote{first act,} its abiding inner act, which tends naturally, by the very innate dynamism of the act of existence itself, to overflow into a \enquote{second act,} which is called action or activity. Every second act of a being points back toward its first act as to its ground and source, and every first act, in turn, points forward to its natural self-expression in a second act.\footcite[64]{clarke:action}
\end{quotation}
%
In other words, it is the very structure of \emph{ens} that produces the \emph{ordo ad finem}: the \emph{suppositum} is endowed with a limited but abundant \emph{actus essendi}, measured and determined by the \emph{essentia ut potentia essendi}. It is inevitable that it should ``overflow'' into \emph{esse in actu} and ``desire'' its ``fulfillment'' in \emph{operari}.%
%
\footnote{I will venture to make an analogy for how a \emph{suppositum} is to be conceived according to Saint Thomas. If for the metaphysics of essence and classical Thomism the image is a roomful of statues, it seems to me that a hydroelectric dam better illustrates the internal dynamism of a substance. A hydroelectric dam is a concrete barrier that blocks the path of a river in an appropriate place, so that the water is allowed to accumulate in a valley behind it. The water amassed exerts enormous pressure, which can then be harnessed to produce electricity. The accumulated water is like the \emph{actus essendi}, or as we have called it, \emph{virtus essendi}; the dam is like the essence, which measures and limits the water, but also makes it possible for it to accumulate; the water actually touching the dam, which allows it to perform its work is like \emph{esse in actu} or \emph{esse substantiale}; the water flowing out at high pressure is like the \emph{esse in actu} of the \emph{propria}; and finally, the electricity is like the operation produced by the powers or active potencies. Note how the amassed water (\emph{actus essendi}) is far superior in power to the water actually found at the dam (\emph{essence}). For this reason, it exerts a tremendous ``pressure'' to spring forth from the dam. Nevertheless, the amassed water is only a \emph{virtus essendi}; it is not yet \emph{in actu} and does not reach its ``fulfillment'' until it emerges, first as a stream of pressurized water (\emph{accidentia propria}), and then as electricity (\emph{operari}). Nevertheless, the amassed water (\emph{virtus essendi}) is still \emph{limited} or, if you will, ``contracted;'' it does not exhaust all the potential that the river could generate (with a bigger dam, for example).}
%
In this way, we can justify the Thomistic maxim, ``Omnis substantia est propter suam operationem.''\footcite[I, cap.~45, n.~6 (Marietti n.~387)]{st:contragent} As Clarke points out, however, not only does \emph{esse ut actus} have \emph{esse in actu} as its end, in a different way, \emph{esse in actu} and in particular \emph{operari} has \emph{esse ut actus} as its end. More precisely, the \emph{suppositum} has the \emph{actus essendi} as its end, in the sense that its ``goal'' is the complete realization of all the ``potential'' contained in its \emph{virtus essendi}.%
%
\footnote{Regarding \emph{actus essendi} as an intrinsic final cause of \emph{esse in actu}, see \cite[61--62]{contat:esse-essentia-ordo}.}
%

\begin{DONE}
\CITEME{Book 3 C.G. 1-22 Must be ``finalized.'' Soul is Horizo et confinium.}
\end{DONE}