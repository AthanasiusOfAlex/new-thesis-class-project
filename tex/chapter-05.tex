% !TEX TS-program = xelatex
% !TEX encoding = UTF-8 Unicode
% !BIB TS-program = biber

\chapter{Conclusion}
\label{chap:conclusion}

\section{\emph{Ens} and its Intrinsic Dynamism}

Let us now summarize in more systematic way what we have discovered about \emph{ens} and its intrinsic dynamism. In a twofold \emph{resolutio}, we investigated the intrinsic causes of both \emph{bonum} and then \emph{ens} itself. There is no need to demonstrate the \emph{fact} of \emph{ens} or \emph{bonum}: both of these are original notions experienced directly and infallibly by any subject endowed with an intellect and will by the very exercise (\emph{in actu exercito}) of those faculties. Man grasps them nearly always through the exercise of his sensitive faculties as well, but regardless, \emph{ens} is known primarily because it \emph{acts} upon the intellect, and \emph{bonum} is known because \emph{ens} is presented to the will as \emph{desirable}. We found that although the notion of the good (\emph{ratio boni}) is appetibility, it is founded on a more profound reality, which is its intrinsic perfection (\emph{natura boni}).  Although \emph{bonum} is grasped primarily through perfection \emph{in actu secundo}, it is rooted nevertheless in the \emph{esse in actu} of those perfections and ultimately---through the mediation of the \emph{essentia in actu}---in the very \emph{actus essendi} of the \emph{suppositum} that is good. We also noted that a thing can be good inasmuch as it is 	\emph{perfective} (inasmuch as it makes the subject desiring it more perfect) and also inasmuch as it is \emph{perfect} (inasmuch as it already imitates, as it were, the perfection in the good that it desires). This is, so to speak, the \emph{fact} of \emph{bonum}, but our investigation seeks the metaphysical \emph{foundation} of this fact. Certainly \emph{omne agens agit propter finem}---even a staunch materialist must on some level acknowledge the \emph{fact}---but the question is \emph{why} does \emph{ens} act at all? Why must it be an \emph{agens} and seek its own perfection?

In order to discover this foundation, we must do a \emph{resolutio secundum rationem} of the intrinsic causes of \emph{ens}. With Aristotle, we discover that \emph{ens} applies to a whole series of realities, which do not simply form an unrelated ``heap,'' but are ordered in a hierarchy: first substance, because it is \gk{χωριστόν} and \gk{τόδε τι}, and then the various types of accidents. By a linguistic analysis, we can determine that \emph{ens} reduces to substance and accident, on the one hand, and act and potency on the other; and that ontologically speaking, accident depends on and reduces to substance, and potency reduces to act. Moreover, substance is most properly \gk{τὸ τί ἦν εἶναι}---what we have called \emph{essentia in actu}, the concept that distances Aristotle from any type of essentialism---and the most perfect act seems to be existence (\gk{τὸ ὑπάρχειν}). \emph{Ens}, therefore, reduces to substance in act.

With Thomas Aquinas, we can take this \emph{resolutio} further: the intrinsic principles of substance \emph{in acto primo}---matter and form---and of a suppositum \emph{in actu secundo}---substance and accident---are insufficient for explaining either the degrees of intensity in being among different species (especially among species that display completely different levels of \emph{operari}: inanimate objects, plants, animals, men, and angels) or the possibility of purely spiritual creatures.
Thanks to the Platonic (and especially Neoplatonic) reflection on the Good and its essential ``generosity,'' we realize that act (the foundation of the good) is essentially ``expansive'': it ``fills'' the ``space'' allowed for it by the potency it is composed with or else---if the potency is entirely absent---it is infinite and unlimited. Indeed, only a potency can limit act. Since we find \emph{supposita} with differing intensities of being (as reflected by qualitatively superior and inferior levels of activity---ranging from merely ``being,'' to ``living,'' to ``sensitive,'' to ``rational'') and since the\emph{}re are even distinct substances without any matter (namely, angels), it must follow that the original \gk{ἐνέργεια} that actuates \emph{ens} (which we call \emph{esse ut actus} or \emph{actus essendi}) is composed with a potency (which we call \emph{essentia ut potentia essendi}). \emph{Esse ut actus} must be \emph{prior} to all the proper actuality in the \emph{suppositum}, and the \emph{essentia ut potentia} must be the radical \emph{subiectum} that receives and measures the radical act, more radically ``potential'' than any other potency.

\emph{Esse ut actus} and \emph{essentia ut potentia essendi}, then, are the ultimate active and potential principles of \emph{ens}, and all \emph{ens}---even on the intrinsic level---reduces to a single active principle: \emph{actus essendi}, which is also revealed to be the \emph{virtus essendi}, the source and, so to speak, store of ``power'' for all the proper actuality in a \emph{suppositum}. It is a \emph{virtus essendi} also in the sense that it is not ``realized'' or ``fulfilled'' unless it ``expands'' into all the levels of act (which we call \emph{esse in actu}) that we used \emph{in via causalitatis} to arrive at the radical active priciple: \emph{operari}, \emph{esse accidentale} (\emph{proprium}), and \emph{esse substantiale}. The \emph{actus} or \emph{virtus essendi} is much more ``powerful'' than any of the single levels of \emph{esse in actu}, however, and we are to imagine it as exerting ``pressure'' on the essence so as to ``expand'' as far as it can. It is also a ``\emph{virtus}'' \emph{essendi} because it is not properly speaking \emph{in actu} but \emph{quo agit}; it is, so to speak, \emph{in actu} only in the \emph{esse in actu} that flows from it. Although we did not discuss it in detail, the final step of the \emph{resolutio} would have been \emph{resolutio secundum rem}, to discover the \emph{extrinsic} causes of \emph{ens}: namely, using the \emph{quarta} and \emph{quinta viae} to discover God as the efficient, exemplary (both as cause of \emph{esse participatum} and as the divine idea of the creature), and final cause of all \emph{ens}. \emph{Ens}, therefore, proceeds from God and reaches its fulfillment by returning to him and imitating him. 

The next step is to perform a \emph{compositio}, beginning with the Creator. In order to create a \emph{suppositum}, he must effect what we have called a \emph{Diremtion} or ``split'' between \emph{esse} and essence, because unless there is a \emph{subiectum} to receive it, there can be no other \emph{esse} than \emph{Ipsum Esse Subsistens}. The \emph{Diremtion} also entails a \emph{contractio}: because the \emph{actus essendi} is received, it is much less in intensity than \emph{Ipsum Esse}, and the intensity diminishes in each successive participation (each of which, therefore, also entails a \emph{contractio}). \emph{Actus essendi} and \emph{essentia ut potentia essendi} are best described, therefore, as co-created, and God creates them in order to manifest his glory (because he himself can suffer no movement from potency to act whatsoever) and so that his creatures will return to him and imitate him. \emph{Actus essendi} and \emph{essentia ut potentia} are the radical principles, considered ``before'' the constitution of \emph{ens}. On the other hand, in an \emph{ens} considered as already constituted, we find that the \emph{actus essendi} necessarily communicates itself so as to actuate the essence, thus producing the \emph{essentia in actu}, which is \emph{in actu} by means of its proper \emph{esse in actu}; namely \emph{esse substantiale}. In turn, the \emph{esse substantiale} communicates itself to the various \emph{accidentia propria}, whose \emph{esse in actu} (\emph{esse accidentale}) has as its \emph{subiectum} the \emph{essentia in actu}, considered as formal, but limited (and hence partly potential) act.  This ``expansion'' of the \emph{actus essendi} results precisely because the \emph{actus essendi} is a rich and potent \emph{virtus essendi} that cannot be ``contained'' by the essence (even if the essence does measure and limit it). Indeed, for \emph{esse substantiale} and \emph{accidentale} (at least of the \emph{accidentia propria}), the procession is \emph{necessary}: there is no \emph{suppositum} that lacks them. Some of the \emph{accidentia propria}---namely, the qualities of the second species, which we call ``powers'' or ``active potencies''---act as mediators that ``channel'' the substance's \emph{esse substantiale} into specific \emph{operari}, because the substance merely ``is,'' and in order for it to ``do'' something, some potency with a similarity to the operation proposed must mediate. \emph{Operari}, then, is the ultimate ``expansion'' of the \emph{actus} or \emph{virtus essendi}, and unlike the \emph{accidentia propria}, it is sometimes possible to impede its realization. Doing so, however, results in ``frustration'' or ``stunting'' of the \emph{suppositum}: the \emph{virtus essendi}, in such a case, is impeded from expanding as much as it ``wants'' to. (Strictly speaking, of course, it is the \emph{suppositum} that tends to or desires its own fulfillment, not one of its principles, because \emph{actiones sunt suppositorum}.)

The \emph{suppositum} reveals itself to be a true microcosm of divine causality. The \emph{actus} or \emph{virtus essendi} is, evidently, the efficient cause of the various levels of \emph{esse in actu}; however, it is also the exemplary cause, because it is the ``model'' from which the lesser kinds \emph{esse} are communicated, and it contains them virtually. The \emph{essentia ut potentia essendi} is an exemplary cause in a different way, because it measures and determines the ``intensive quantity'' of \emph{virtus essendi} and specifies the types of \emph{esse in actu} that are proper to the \emph{suppositum} and the extent to which they should expand. \emph{Esse in actu}, on its various levels, is a kind of ``extensive quantity'' when all the levels are considered together, because \emph{esse accidentale} and \emph{operari} are \emph{superadditi}: the \emph{suppositum} acquires more \emph{actus secundus} by obtaining additional and more perfect action. Crucially, we find that \emph{esse in actu} is the \emph{final cause} of the \emph{actus} or \emph{virtus essendi}, precisely because of the intrinsic ``generosity'' of act, and its consequent tendency to ``expand.'' In this way, \emph{omnis substantia est propter suam operationem.} In a different way, the \emph{actus essendi} is the final cause of the \emph{suppositum}, which seeks to be fulfilled by realizing all of the ``potential'' that is found in its \emph{virtus essendi}. This imitation of divine causality is, of course, affected by \emph{Diremtion}: in God, efficient, exemplary, and formal cause all coincide in the same \emph{Ipsum Esse}. \emph{Esse} (\emph{per essentiam}), essence, \emph{operari}, and \emph{Summum Bonum} all coincide in him. In creatures, on the other hand, the causes are ``split'' among different principles, as described above. Moreover, whereas God communicates his \emph{esse} directly without mediation (with respect to the creature), \emph{actus essendi} itself plays a mediating role between \emph{Ipsum Esse} and the \emph{suppositum}, and each successive level of \emph{esse in actu} is communicated by means of a mediation. In this way, we correctly apply the Neoplatonist cosmogenesis, which posits successive necessary and mediated communications of act, not to the \emph{extrinsic} causes of \emph{ens}, but to its \emph{intrinsic} causes. We must stress, however, that \emph{Diremtion} and \emph{contractio} are positive realities, because they \emph{permit} created \emph{ens} to exist, providing the \emph{suppositum} with its receptive capacity for \emph{esse}.

This reflection resolves the \emph{aporiae} regarding the ``richness'' and ``poverty'' of \emph{esse}, as well as the apparent contradiction in having the substance be both the \emph{subiectum} and the cause of the accidents, both \emph{in actu} and \emph{in potentia} with respect to them. The \emph{poverty} of \emph{ens} results from the decrease in its ``intensive quantity,'' the \emph{contractio} that it experiences as it ``cascades,'' first from \emph{Ipsum Esse} into \emph{actus essendi}, and then from \emph{actus essendi} into the levels of \emph{esse in actu}. The ``richness'' of \emph{esse} resides virtually in the \emph{actus essendi}, and reveals itself as it ``expands'' into \emph{esse in actu}, which in fact is a type of ``extensive quantity.'' Likewise, the essence can be both in act and in potency precisely because it \emph{receives} its act from the \emph{actus essendi}, but it only receives it \emph{partially} (\emph{esse substantiale} suffers a \emph{contractio} like any other \emph{esse in actu}); therefore, it is a ``limited'' act that has ``room'' to accomodate the inherence of the accidents. Inasmuch as it is act, it communicates its act and ``produces'' the \emph{accidentia propria}. Inasmuch as it is limited, and even ``deprived'' of the accidents, it provides a \emph{subiectum} in which they can inhere. The dynamic of both \emph{virtus essendi} and the \emph{essentia in actu} bear a resemblance to active potencies or ``powers'': all three are mediators that ``receive'' act from a superior entity and ``re-transmit'' it to an inferior one.

In this way, we have resolved our initial question: why does \emph{ens} seek its own fulfillment? In summary, it does so because God, \emph{Ipsum Esse}, endows it with a certain ``intensive quantity'' of \emph{actus essendi}, which we can characterize as a \emph{virtus essendi}, a quantity measured and limited by \emph{essentia ut potentia essendi}, but not completely contained by it. This \emph{actus essendi} is so ``powerful'' that it necessarily expands into \emph{esse substantiale} and \emph{esse accidentale}. This same power produces an inexorable tendency in the \emph{suppositum} to expand its \emph{esse in actu} as far as it can possibly go, through its \emph{operari}. 


\section{Implications for Ethics and Theology}

\subsection{The Ontological Criterion of Goodness}

The intrinsic tendency of \emph{esse} to ``expand'' suggests what could be termed a criterion for ``metaphysically good'' action: a \emph{suppositum} ``should'' act in whatever way expands its \emph{esse in actu} to the maximum extent allowed by its essence. We could call reaching such a state ``metaphysical happiness or \gk{εὐδαιμονία}.''
For non-spiritual creatures, the criterion naturally has no moral value whatsoever. However, even in such creatures, we witness a certain ``metaphysical frustration'' on the part of those that are impeded from reaching the perfection proper to them. For example, there is a certain deprivation involved when oak trees fall victim to blight or drought, or when an animal dies. It is not, of course, a true tragedy, because the \emph{suppositum} involved in either case is not spiritual, and hence cannot truly be aware of what it suffers. However, when a similar tragedy occurs in a person (man or angel)---above all when it has to do with those \emph{operationes} that perfect the person \emph{simpliciter}; that is, free and voluntary acts that are \emph{contra naturam}---it is a true privation, even in the metaphysical sense, and if it is serious, it constitutes a metaphysical catastrophe.%
%
\footnote{De Finance, however, considers \gk{εὐδαιμονία} to be insufficient as a criterion for morality, because it imposes itself, not as a physical necessity, but as a necessity of \emph{convenientia}. I am \emph{obliged} (morally) to act according to what makes me happy, but not \emph{constrained} to do so. See \cite[259]{definance:essai}: ``l'impossibilité où je suis  de ma volonté-nature, ne constitue ancore, \emph{si l'on s'en tient là}, qu'une nécessité physique et subjective. En voulant mon bonheur, en ne pouvant pas m'empêcher de le vouloir, je conmprends aussi que rien, objectivement, n'impose cette nécessité à ma raison.'' See also his critique of eudemonism in \cite[114--119]{definance:ethique}.

It seems to me that what we have learned about the dynamism of \emph{ens} can help give \gk{εὐδαιμονία} its proper place. Metaphysically speaking what \emph{founds} the reality of moral obligation is the expansiveness of the \emph{actus essendi} and the ``measure'' provided by the essence. The desired end result---fulfillment or \gk{ἐντελέχεια}, which consists in the expansion of \emph{esse} into the \emph{esse in actu} due to it---is what constitutes happiness.}
%

This metaphysical grounding is enormously helpful for ethics, especially to understand the dynamics of human acts. All actions emerge from a \emph{suppositum}'s \emph{actus essendi}, unless an outside agent influences it (as in the case of iron heated by fire). For spiritual substances---persons---this includes, naturally, the cognitive powers (intellect and faculties for sensible knowledge) and the tendencies or appetites that stem from them (the will and the sensitive appetites). Even these powers' \emph{operari}---immanent acts of the will, its external actuation, and even the passions that spring forth from the sensitive tendencies---flow ultimately from the same \emph{actus essendi}. However, this \emph{actus essendi} is measured and determined by the essence that is co-created with it. It specifies the kinds of \emph{esse in actu} that can spring forth, some of which, as we saw, appear necessarily, and others of which can fail to appear. In spiritual creatures, some of these actuations appear entirely at the discretion of the \emph{suppositum} (in this case a person): the free and voluntary acts that are mediated by the will, as well as those \emph{habitus} and \emph{dispositiones} resulting from those acts.

Again, it is the essence that determines and specifies whether the presence of these ``voluntary'' expansions of \emph{esse} are necessary for the person's happiness or not. Clearly, for spiritual creatures, some of these expansions will be strictly necessary for fulfillment (\emph{ad bene esse}) and others ``facultative'' but most fitting (\emph{ad melius esse}). The essence fixes a certain ``minimum'' level of \emph{esse in actu} necessary for fulfillment---which forms the basis for the natural law---but the concrete distinction between \emph{bene esse} and \emph{melius esse}, naturally, depends the essence and as well on many other factors, especially the various \emph{habitus} and \emph{dispositiones} that constitute the person's particular situation.%
%
\footnote{For a discussion of the role of essence, or nature, in determining the morality of acts, see chapter III of \cite[31--44]{lucas:absoluto}, and also \cite[459--475]{millan-puelles:libre}.}
%
For example, remaining chaste is necessary for the happiness of all men (\emph{ad bene esse}), but choosing a celibate lifestyle such as a priestly vocation is fitting \emph{ad melius esse} for those persons whom God has called. As can be seen, a metaphysics of \emph{agere}, properly conceived in light of an intensive and emergent \emph{actus} (\emph{virtus}) \emph{essendi}, helps to make a coherent whole out of the many apparently disparate aspects of ethics---for example, happiness, virtue, freedom, the will, right reason, human acts, and obligation---all of which have their metaphysical root in the emergent \emph{esse ut actus}.

\subsection{Trinitarian Theology}

Although this paper is, of course, philosophical in nature, we will make some mention in passing of the implications that a good metaphysics of \emph{actus essendi}, thus extended to \emph{operari}, has for dogmatic theology. First of all, it cannot have escaped the reader that the three great metaphysical domains---\emph{esse}, \emph{essentia}, and \emph{operari}---and especially the three corresponding types of extrinsic causality---efficient, exemplary, and final---can readily be attributed to the three Persons of the Trinity.%
%
\footnote{For a discussion of this topic see \cite[23--31]{contat:esse-essentia-ordo}.}
%
Efficient causality is attributable to the Father, who is Principle without Principle, the Unbegotten Begetter of the Son. To the Son, who is the Eternal Word, the Image of the Father, and the Mediator between Father and Spirit, is easily attributable exemplary causality. (To him is attributed the place of the divine ideas, like the Neoplatonic \gk{Νοῦς}, but appropriately purified.) Finally, to the Spirit, who proceeds from the love of Father and Son, is attributed divine action. The intrinsic dynamism of \emph{ens}, therefore, is a true \emph{vestigium} of the Holy Trinity.

As a speculation, it seems to me that it would be fruitful to investigate whether the metaphysics of \emph{actus essendi}, as applied to \emph{operari}, would shed light into the problem of the \emph{Filioque}. On a theological level, the \emph{Filioque} controversy arises from a difference between the Greek and Latin conceptions of procession (\gk{ἐκπόρευσις} and \emph{processio}): the former always includes the \emph{ultimate} origin of the procession (hence the Spirit \gk{ἐκπόρευται} from the Father alone); the latter is a more generic notion that does not include the ultimate origin (hence the Spirit proceeds from the Father and the Son as from a single principle).%
%
\footnote{This distinction is very well explained in \cite{doc:processionhs}.}
%
It is worth exploring the striking similarity here between these notions of procession and what we have discussed regarding the distinction between \emph{esse ut actus} and \emph{esse in actu}.

\subsection{Christology}

The metaphysics of \emph{actus essendi} makes it much easier, it seems to me, to develop a Christology that truly takes into account the great Christological dogmas. Jesus is one \gk{ὑπόστασις} in two unmixed, immutable, undivided, and unseparated natures.%
%
\footnote{For an excellent overview of the contribution of the Council of Chalcedon and Christological speculation to metaphysics, see chapters IV and V of \cite[49--62]{lucas:absoluto}.}
%
Using the thesis we have exposed in this paper, it is easy to show the absurdity of the various Christological heresies: against Apollinarianism, Monotheletism, and Monoenergism, it is evident that from Jesus' human nature must flow a complete human soul (\emph{esse substantiale}), a complete human intellect and will (\emph{esse accidentale}), and properly human acts (\emph{operari}). We can see that Monophysism (at least that of Eutyches) cannot be true, because it would violate what we learned about \emph{Diremtion} and \emph{contractio}: Jesus cannot be a true man without a human nature that is really distinct from the Divine Nature (\emph{Ipsum Esse}) and from its own \emph{esse}. What about Jesus' \emph{actus essendi}? A human nature endowed with its own, ``contracted'' \emph{actus essendi} would be a fully formed \emph{suppositum} (\gk{ὑπόστασις}): this is the error of Nestorius. Christ does not have his own human \gk{ὑπόστασις}; therefore, neither does he have his own human, contracted \emph{actus essendi}. Rather, he is hypostatically united to the Word, which is the entire \emph{Ipsum Esse Subsistens}, or if you will, \emph{Virtus Essendi Subsistens}. This condition explains the \gk{ἐνυποστασία} of Neochalcedonianism, as well as the \emph{communicatio idiomatum}: did Jesus the man create the universe? Yes, he certainly did, through his \emph{Virtus Essendi}, which is the \emph{Virtus Essendi} of the Divine Word.

\subsection{Grace and the Supernatural}
\label{sec:grace}

We made a brief mention of the implications that the metaphysics of \emph{actus essendi} has for the doctrine of grace and its relationship with nature. With the notions we have developed, we see clearly that no type of grace can be an \emph{accidens proprium}, but rather must be applied to the soul by an outside agent (namely, God).
Sanctifying grace does, of course, confer a type of \emph{esse accidentale}, but it is communicated from God, and (not unlike the \emph{esse substantiale}) it permeates all of the actuality of the soul.
In particular, it endows the intellect and the will with new virtuality, which makes possible acts of faith, hope, and charity, and endows them with infused moral virtues.
(Some of the virtues---faith and hope---can remain even in the absence of grace, unless counteracted by contrary acts of the will.)
From this point of view, we can see clearly that grace takes nothing away from nature, but rather enhances it.
Grace \emph{increases} man's freedom, and does not diminish it.
We see also that the ``contracted'' \emph{actus essendi} of a person, even though it is much more ``powerful,'' so to speak, than the essence, is nevertheless limited, and so it does not have sufficient \emph{virtus} to impel the person to reach the end it longs for: the Beatific Vision.
The \emph{desire} to see God arises from having a \emph{virtus essendi} sufficiently powerful to allow the person to be united intentionally to every form (to be \emph{quodammodo omnia}). Since man is capable knowing and desiring all forms, \emph{a fortiori} he must (at least implicitly) desire the Creator of those forms, their \emph{Summum Bonum}---hence the \emph{desiderium naturale vivendi Deum}.
The \emph{virtus essendi}, however, cannot raise the soul to God all by itself: it needs the help of grace. Similarly, actual graces work by enhancing the powers of the soul (intellect and will), communicating to them an increased \emph{esse in actu}; there is no question, therefore, of the grace being inefficacious if the person refuses to put it into action. In this way, grace in no way impedes freedom, and freedom in no way impedes divine efficacy.
A metaphysics of \emph{actus essendi}, therefore, goes a long way to answering the classic problem raised by H. de Lubac in his \emph{Mystère du surnaturel}, regarding the legitimacy of the hypothesis of ``pure nature.''\,%
%
\footnote{See especially chapter V, ``Le « \emph{donum perfectum} »,'' in \cite[105--134]{delubac:mystere}%
%, English translation in \cite[75--100]{delubac:mystere:en}
.

For example, the doctrine of \emph{actus essendi} that we have exposed effectively resolves the \emph{aporia} that de Lubac raises regarding the ``gift of being'' (106--108). In summary, he says that \emph{esse} cannot be regarded as ``gift'' in exactly the same way as a man gives a gift to another, because unlike the ``gift'' of \emph{esse}, ordinary gifts presuppose that the receiver already exists. (In reality, ``gift'' is simply an image for any sort of communication of a perfection.) We concede the point, of course, but what we learned about \emph{Diremtion} sheds light on the problem: as long as we admit that God co-creates the \emph{essentia ut potentia essendi} together with the \emph{actus essendi}, then we can validly say that the essence \emph{receives} the \emph{actus essendi}; we saw in section \ref{arguments-real-composition} that Thomas himself asserts this in the \emph{Summa theologiae}, I, q.~7, a.~1. In that very precise sense, we can say that a substance receives the gift of \emph{esse}.}
%


\subsection{Sacramental Theology}

Finally, it seems to me that the doctrine of the \emph{actus essendi} sheds much light on the question of how the sacraments transmit grace. The sacraments work \emph{ex opere operato}, certainly, but this fact does not mean that the minister, recipient, and the ``matter'' for the sacrament have no causative role to play. Clearly, when one of these elements is missing or somehow impeded (by a lack of intention, for example), the sacrament is invalid: \emph{sublata causa, tollitur effectus}. It seems to be a case of the classic problem of the \emph{concursus divinus}: how can God and man both be the cause of the sacrament at the same time? The issue becomes much less problematic if we assert that God communicates to the minister the \emph{power} (active potency) to confer the sacrament, again through a communicated \emph{esse in actu}. Therefore, the powers of the minister can be characterized as a kind of instrumental causality, in which the \emph{virtus} to produce the effect originates from God, but in such a way that the real causality and freedom of the minister are in no way impeded.%
%
\footnote{This is, in fact, exactly how Thomas characterizes the causality of the sacraments: ``Secundo autem modo homo potest operari ad interiorem effectum sacramenti, inquantum operatur per modum ministri. Nam eadem ratio est ministri et instrumenti, utriusque enim actio exterius adhibetur, sed sortitur effectum interiorem ex virtute principalis agentis, quod est Deus'' \parencite[III, q.~64, a.~1, co.]{st:summa}. In a similar way, when the sacrament imposes a character, the character behaves as a power that flows from the principle agent (God): ``Character etiam, qui est interior quorundam sacramentorum effectus, est virtus instrumentalis, quae manat a principali agente, quod est Deus'' \parencite[III, q.~64, a.~1, co.]{st:summa}.}
%


\section{In Dialogue with Philosophies of Action}

The theses proposed in this paper also provide many possibilities for dialogue with many of the philosophies of action that arose in the nineteenth and twentieth centuries.%
\footnote{For an excellent discussion of the ways in which Thomism, in the light of the metaphysics of \emph{actus essendi}, can dialogue with contemporary philosophies, see \cite{clarke:thomism}.}
%
Modern philosophy, which has its roots in the metaphysics of essence, is commonly accused of ``objectifying'' reality, reducing it to the aspects that are measurable and observable (especially, through mathematical consideration): this tendency is certainly discernible in the era beginning with Descartes and ending with Kant. In reaction against this tendency, Blondel, for example, proposed a new \emph{initium philosophandi} based on the will, as we saw. In a similar way, Heidegger decried the objectivization of \emph{Seiendes} as present-at-hand,%
%
\footnote{See especially Heidegger's critique of Descartes in \cite[89--101 (122--134 in the English translation)]{heidegger:being}.}
%
and Sartre maintained man's absolute freedom.%
%
\footnote{For an overview of Sartre's concept of freedom, see \cite[35--37]{lucas:orizzonte}.}
%
All of these currents have in common the desire to emphasize the dynamic and ``vital'' character of man. ``Substance,'' ``nature,'' and ``essence'' seem to be concepts too cold and calculating to apply to man. In a sense, the critique of these philosophies is justified, when it is applied to a philosophical tradition that has its roots in the metaphysics of essence (as is the case with modern philosophy).
%
For example, Heidegger and Sartre are not entirely wrong when they make  authenticity the criterion of morality. As Sartre puts it,
%
\begin{quotation}
[L]'authenti­cité et l'individualité se gagnent : je ne serai ma propre authenticité que si, sous l'influence de l'appel de la conscience (\emph{Ruf des Gewissens}), je m'élance vers la mort, avec décision-résolue (\emph{Entschlossenheit}), comme vers ma possibilité la plus propre. A ce moment, je me dévoile à moi-même dans l'authenticité et les autres aussi je les élève avec moi vers l'authentique.\footcite[285]{sartre:etre-et-neant}
\end{quotation}
%
Of course, we must hasten to point out, against Sartre and Heidegger's anthropology,%
%
\footnote{``Anthropology,'' however, is not a term that either one would have applied to a metaphysics of man.}
%
that man is much more than a ``thrown projection'' (\emph{geworfen Entwurf}) or ``absolute freedom.'' The radical source of his power to make free choices is his \emph{actus essendi}, which he possesses thanks to the essence that measures it. However, in acting well---that is, in accord with the measure provided by the essence---man indeed becomes fulfilled, or, if you will, authentic: more a man, or more perfectly a man.
%
The metaphysics of \emph{actus essendi} seems to be the answer to the concerns of these philosophers, while providing a much more solid metaphysical basis: each \emph{suppositum} has a ``source'' of \gk{ἐνέργεια} all its own, and hence has real autonomy. It has an essence, which provides it with a certain amount of determination (we are not ``pure freedom''), but it is also free and dynamic, to the degree that the essence, measure of \emph{actus essendi}, permits it.

\section{Final Conclusions}

I think we can conclude, therefore, that the theory of \emph{actus essendi}, as applied to \emph{operari}, is revealed to have an enormous fecundity and explicative power. We could say that it simultaneously explains the complete dependence of the creature on its creator, while at the same time preserving its real, but participated, autonomy of being and action. The theory avoids entirely the problem of the ``conflict'' between divine and creaturely causality, distinguishing and respecting each ambit without placing them in opposition. It would be worth exploring the enormous fecundity it seems to have for ethics and as a preparation for dogmatic theology. Hence, the metaphysics of \emph{actus essendi} points the way to restoring philosophy in its role as \emph{ancilla theologiae}. It is also extremely helpful in dialogue with those philosophies emphasizing \emph{operari} that arose after the fall of modern philosophy: the action theory of Blondel, transcendental Thomism, and the various currents of Existentialism. A metaphysics of \emph{operari}, understood in the light of \emph{actus essendi} considered as \emph{virtus essendi}, it seems to me, is the answer to the excellent problems raised by these philosophies. 

The questions that we asked at the beginning of the paper were the following: why does \emph{ens} always seek its own perfection? Why does \emph{omne agens} always \emph{agit propter finem}? Is there an answer to this problem on the \emph{intrinsic} level, or must we make recourse to the extrinsic causes? We we are now in a position to answer them succintly: \emph{ens} seeks its own perfection, and \emph{omne agens agit propter finem}, because its act of being (\emph{esse ut actus}) is a source of virtuality (\emph{virtus essendi}) that is much greater than the essence that measures it. Act is ``generous'' and expansive, and hence the \emph{actus} or \emph{virtus essendi} necessarily flows out into the \emph{esse accidentale} of the \emph{accidentia propria}. Even this expansion, however, does not exhaust the power of the \emph{actus essendi}, and hence the \emph{suppositum} seeks to expand its \emph{esse in actu} as far as possible into the \emph{operari} and \emph{habitus} that the essence determines are suitable to it. The very intrinsic structure of \emph{ens}, therefore, produces a ``desire'' in every \emph{ens} for its own perfection, which could also be called an \emph{ordo ad finem}. The intrinsic causes of this ``desire,'' however, are an image of and a participation in the extrinsic causes of \emph{ens}: God as efficient, exemplary, and final cause. Can we go from an \emph{is} to an \emph{ought}? We most certainly can, if the ``is'' represents an act of being that is a \emph{virtus essendi}, the source of every proper actuality.

\begin{DONE}

\begin{NOTES}
\begin{itemize}
  
  \item FOR INTRO:

  \item actus essendi is remote principle of operation.

  \item Thomas problem: how does esse found agere? that is the open question.

  \item Esse and agere are not heterogeneous!

  \item Esse cannot help but expand! (done)

  \item Fran O'Rourke. Part et Causa. Etre et Agir.

  \item FOR THE HISTORICAL PART

  \item Aristotle, Hegel, Kant, Blondel
  
  \item Aristotle analyses many types of act: entitative (uparchein); operative (movement/ab extrinseco); transitive (vital); immanent (spiritual). See also Etre et Agir. Greatest act of all is Sophia. Just as with \gk{οὐσία}/\gk{ἐνέργεια}, the multiplicity of activity resolves only in noesis noeseos.

  \item the structure of the First Part: Exitus from God, so speak; exitus and reditus of angels; but only exitus of man, and the intellectual dimension. Man's free action (hence reditus) is dealt with in the Ia-IIae (and esp. the IIIa). See Patfoorte.



  \item The \emph{esse ut actus} / \emph{essentia} composition implies \emph{ordo ad finem}.
  
  \item The key notion behind this is that act tends to communicate itself.
  
  \item The criterion is the maximization of operational \emph{esse in actu}.
  
  \item include: diagram of hydroelectric dam.


  \item include: steresis and privation (De Finance etre et agir) as a reason for the need to expand. I.e., \emph{actus essendi} is ``deprived'' until it can expand.

  \item include: a comparison of this \emph{schema} with the schema of Plotinus. One-->Nous-->Soul = Esse ut actus --> substance --> accident/operation.

  
  \item \cite[64]{clarke:action}: \enquote{Thus it is proper to every being, insofar as it is in act, to overflow into action, to act according to its nature. [\ldots] The act of existence of any being (its \enquote{to be} or \emph{esse}) is its \enquote{first act,} its abiding inner act, which tends naturally, by the very innate dynamism of the act of existence itself, to overflow into a \enquote{second act,} which is called action or activity. Every second act of a being points back toward its first act as to its ground and source, and every first act, in turn, points forward to its natural self-expression in a second act.}

  \item{The Measuring Capacity of Essence}

  \begin{itemize}
    \item How essence mediates and measures \emph{virtus essendi}, \enquote{distributing} it to its various \emph{esse in actu}.

    \item This means in particular that the final \enquote{expansion} depends on the degree of 

    \item The ultimate criterion is \enquote{what expands esse is to be done; what diminishes it is to be avoided} (in different degrees).

    \item Hence Heidegger and Sartre and Nietzsche are \emph{right} if affirming that authenticity and self-fulfillment are to be sought. It cannot be sought, however, \emph{except in the measure provided by essence (for man, human nature)}.
  \end{itemize}
\end{itemize}
	
God as efficient, exemplary, and final extrinsic cause.

The \emph{virtus essendi} as efficient, exemplary, and final cause of the \emph{operari}. (Just as God is efficient, exemplary, and final cause of \emph{ens}.)

Hence, just as \emph{efficient} and \emph{exemplary} cause are both intrinsic and extrinsic, so too \emph{final}.

Ultimate end as full \enquote{expansion} of \emph{virtus essendi}.

\end{NOTES}

\end{DONE}