% !TEX TS-program = xelatex
% !TEX encoding = UTF-8 Unicode
% !BIB TS-program = biber

\chapter{\emph{Esse ut Actus} as \emph{Virtus Essendi}}
\label{chap:ens_resolutio}

In the previous chapter, we were able to show from the notion of \emph{bonum} that its foundation is \emph{esse}. We have, however, already observed that this \emph{esse} comes in various types, which explain how \emph{ens} can be said \emph{simpliciter} or \emph{secundum quid}: namely, there is the \emph{esse} proper to substance, that proper to the accidents (especially those that are relatively stable, such as the \emph{propria}), and finally \emph{operari}. There is evidently an analogy of proportionality among the three levels, and it is also clear—taking our cue from Aristotle’s \emph{Metaphysics}, Γ, 2—that these levels can all be reduced in some way to the \emph{esse} proper to substance (\emph{esse substantiale}). In this chapter, we will show that this reduction in fact arises from an intrinsic \emph{participation} of \emph{esse}, and that the three levels of \emph{esse}—which can be termed \emph{esse in actu} or \emph{esse} on the “predicamental” level—can ultimately be reduced to a radical principle of actuality that contains them “virtually”: the \emph{actus essendi} or \emph{esse ut actus}, which for this reason can also be termed \emph{virtus essendi}.

\section{Historical Background of \emph{Esse}}

\subsection{From the Presocratics to Aristotle}
\label{esse-presocratics-to-aristotle}

The first known investigator into the ultimate intrinsic principle of reality is Parmenides, who asserts that the world can be reduced to a unique, immovable principle called τὸ εἶναι—or more exactly τὸ ἐστίν, for he felt that the conjugated verb \emph{to be}, without a subject, best characterized this underlying principle:

\begin{quotation}\noindent
Εἰ δ᾽ ἄγ᾽ ἐγὼν ἐρέω, κόμισαι δὲ σὺ μῦθον ἀκούσας,\\ 
αἵπερ ὁδοὶ μοῦναι διζήσιός εἰσι νοῆσαι·\\
ἡ μὲν ὅπως ἔστιν τε καὶ ὡς οὐκ ἔστι μὴ εἶναι,\\
Πειθοῦς ἐστι κέλευθος (Ἀληθείῃ γὰρ ὀπηδεῖ)\\
ἡ δ᾽ ὡς οὐκ ἔστιν τε καὶ ὡς χρεών ἐστι μὴ εἶναι, \\
τὴν δή τοι φράζω παναπευθέα ἔμμεν ἀταρπόν· \\
οὔτε γὰρ ἂν γνοίης τό γε μὴ ἐὸν (οὐ γὰρ ἀνυστόν) \\
οὔτε φράσαις.%
%
\footnote{\Cite[DK28b2]{dk}: “For come, I say, do (you) take care, and listen to my word. There are only [two] ways of inquiry to be thought: the first, namely, that ‘is,’ and that it is impossible for anything not to be, is the path of persuasion (for it accompanies truth); the other, namely, that ‘is not’ and that there must be something that is not, this, I make clear to you, that it is a completely unknowable path. For you can never know what is not (for it cannot be done), nor can you say it” (my translation).}
%
\end{quotation}
%
All reality “is,” he argues, and it cannot “become” from something outside of being—because there is nothing outside of being—and nothing can come to be from non-being. It follows that the reality behind is simply τὸ ἐστίν; changes and multiplicity are mere appearance. Such an extreme monism is evidently difficult to maintain,%
%
\footnote{For example, one might ask, even if multiplicity and change are “appearances,” how can “appearance” be really distinct from “reality”?} and so it is easy to see why Sophists such as Gorgias, in critiquing Parmenides, fell into relativism: if being, knowledge, and discourse imply an absurd unity and immobility of reality, then they must be rejected.\footcite[93–94]{copleston:history:01} It was Plato who, in the face of such unsatisfactory claims, first attempted to reconcile Parmenides’ search for a stable and immutable principle with the evident changeability and multiplicity of the sensible world, and in so doing refute the Sophists’ relativism. As we saw above, he thought that only separate, unchangeable, intelligible realities—τὰ εἴδη—could account for stable and universal knowledge.%
%
\footnote{It is interesting to see that Plato has a marked similarity with Hume and Kant on one point of doctrine: all of them profoundly distrust sensible knowledge. The difference between Plato and the Empiricists, of course, is that Plato believes that through ἀνάμνησις and dialectic one can reach the intelligible realities; Hume and Kant, naturally, consider it impossible.}
%
In fact, every valid predication, he would argue, entails a participation in an idea: “Fifi is a cat” means that Fifi participates in the idea of “cat.” Plato sees, with Parmenides, that τὸ ὄν can be predicated of all things—sensible as well as intelligible—and hence all of these must participate in a unique “meta-idea” called τὸ ὄν. Participation, however, is not complete identity (τὸ ταὐτὸν): it necessarily means a merely \emph{partial} identification of the \emph{participans} with the \emph{particpatum}. Plato famously concluded that, for this reason, all \emph{entia} must also participate in the “different” (τὸ ἕτερον), a type of non-being.%
%
\footnote{The exact word used by Plato in the dialogue is the related τὸ θάτερον, which is essentially a synonym.} This conclusion constitutes the famous “parricide of Parmenides”: asserting that being (τὸ ὄν) and non-being (of which τὸ ἕτερον is a type) are present in \emph{every} reality.%
%
\footnote{For Plato’s derivation of the meta-ideas, see \cite[251b–259a]{plato:sophist}, especially 254b–257a.}
%
Since τὸ ὄν is a form or idea (a “meta-idea,” certainly, but ontologically similar to the lower ideas), and ideas are necessarily “separate,” we can say that in order to avoid monism, Plato reduced Parmenides’ τὸ εἶναι to an extrinsic and formal participation in τὸ ὄν, that takes place in a sort of dialectic solidarity with non-being.%
%
\footnote{In other words, being and non-being are contradictory concepts when taken without qualification. Plato, lacking analogy and a doctrine of composition, resorts to dialectic in order to explain how they can coexist in the same reality: while affirming that something \emph{is}, we affirm at the same time that it \emph{is not}.}

Aristotle, as it were, restored to τὸ εἶναι its intrinsic place in τὸ ὄν. Whereas Plato, it could be said, would have considered the First Science to be dialectic, the study of the relations among the ideas, Aristotle considered it to be the study of \emph{ens qua ens} (τοῦ ὄντος ᾗ ὄν), which—as he argues in the famous passage from \emph{Metaphysics}, Γ, 2, cited above—reduces to οὐσία. He recognized that τὸ ὄν is differentiated into various degrees, both on the “horizontal” level (accidents with respect to the substance in which they are inherent) and on the “vertical” level (the various types of substances: sublunar, superlunar, and ultimately the First Mover).%
%
\footnote{Regarding the different levels of movement (and the corresponding substances) in Aristotle, see \cite[401–402]{mitchell:four}.}
%
Most significantly, he recognized that τὸ ὄν could be reduced to two intrinsic principles: οὐσία and ἐνέργεια, and that οὐσία could be characterized as τὸ τί ἦν εἶναι.%
%
\footnote{We will make a more detailed examination of how Aristotle reduces the object of the First Science to its intrinsic principles.} It follows that τὸ εἶναι is in a way at the root of the actuality of all οὐσία.  Aristotle did not, however, explore this route explicitly any further; it was for the Neoplatonists, especially Pseudo-Dionysius, to discover the virtuality and “fecundity,” of τὸ εἶναι.

\subsection{The Unlimitedness of τὸ εἶναι}

In his hymn in praise of τὸ ἐστίν, after attributing to it a number of attributes that would apply very well to God, Parmenides gives the following description, which may seem surprising: κρατερὴ γὰρ Ἀνάγκη / \emph{πείρατος} ἐν δεσμοῖσιν ἔχει, τό μιν ἀμφὶς ἐέργει;%
%
\footnote{“A strong force keeps it in the chains of \emph{limit}, which shuts it in on both sides” (my translation; emphasis added), \cite[DK28b8, l.~30]{dk}. Regarding the ἄπειρον in Greek philosophy, see \cite[42–49]{definance:etre-et-agir}.} in other words, among the attributes of τὸ ἐστίν is its \emph{limitedness}. What Parmenides seemingly is trying to say is that τὸ ἐστίν is well \emph{defined}; he is denying that it is ἄπειρον, which could mean either “limitless” or “indefinite,” but he does not distinguish. Melissus of Samo, his disciple, seems to have been the first to make the distinction, calling the totality of reality ἄπειρον.%
%
\footnote{“Melissus [dicit] hoc quod esset infinitum et inmutabile et fuisse semper et fore;” “Διογένης καὶ Μέλισσος [λέγουσιν] τὸ μὲν πᾶν ἄπειρον, τὸν δὲ κόσμον πεπεράνθαι” \parencite[DK30a8]{dk}. These are not original fragments, but witnesses found in Cicero’s \emph{Lucullus} and from Theodoret of Cyrus’ \emph{Graecarum affectionum curatio}.} The Neoplatonists, however, were the first ones to understand fully that the infinity (τὸ ἄπειρον) of the Principle entails the fullness of “power” (δύναμις or \emph{virtus}).%
%
\footnote{It should be said that Aristotle may have laid the basis for this understanding. He called the First Mover “Pure Act” and νόησις νοήσεως, which for Aristotle is the exercise of the most perfect act by the supreme being. While the notion of “fullness” is not explicit, it is implied by the very notion of act—a conclusion that the Neoplatonists came to see more clearly. See \cite[Λ,~7, 1072a24–25]{aristotle:metaphysics}: ἐπεὶ δὲ τὸ κινούμενον καὶ κινοῦν καὶ μέσον, τοίνυν ἔστι τι ὃ οὐ κινούμενον κινεῖ, ἀΐδιον καὶ οὐσία καὶ ἐνέργεια οὖσα, “For since what is moved and moving is an intermediate, therefore there is something that is not moved by which it moves, being eternal substance and act.” See also \cite[Λ,~9, 1074b34]{aristotle:metaphysics}: αὑτὸν [τὸ νοῦς] ἄρα νοεῖ, εἴπερ ἐστὶ τὸ κράτιστον, καὶ ἔστιν ἡ νόησις νοήσεως νόησις, “Therefore it [the divine intellect] thinks itself, since it is the mightiest, and the thinking [of this intellect] is a thinking of thinking” (my translation in both cases).} For example, Plotinus says of the One, “[τὸ ἕν] Ληπτέον δὲ καὶ ἄπειρον αὐτὸν οὐ τῷ ἀδιεξιτήτῳ ἢ τοῦ μεγέθους ἢ τοῦ ἀριθμοῦ, ἀλλὰ τῷ ἀπεριλήπτῳ τῆς δυνάμεως.”\,%
%
\footnote{\cite[VI,~9,~6]{plotinus:enneads}: “[The One] is to be taken also as infinite, not in measurelessness of magnitude or number, but in unboundedness of power” (my translation). Plotinus frames the distinction in a way that is similar to a distinction we will make in section \ref{sec:virtus-essendi} between intensive and extensive quantity.} The choice of the term δύναμις is significant, because the it means that the Supreme Principle—which for Plotinus is “beyond” being, as we saw—is not simply infinite in “size,” but infinite in \emph{power}.

Dionysius took up this idea, connecting it to τὸ εἶναι:%
%
\footnote{For an excellent  discussion of Pseudo-Dionysius’ intensive conception of being, see \cite[50–56]{orourke:pseudo-dionysius}.} although for Dionysius τὸ ἀγαθόν is the most proper name for God, nevertheless τὸ εἶναι or, more correctly, τὸ ὑπερ\-εῖναι, is attributable to God because τὸ εἶναι is God’s primary gift or effect;%
%
\footnote{See \cite[V, 5]{pg:dionysius:DN}: Πάντων οὖν εἰκότως τῶν ἄλλων ἀρχηγικώτερον ὡς ὢν ὁ θεὸς ἐκ τῆς πρεσβυτέρας τῶν ἄλλων αὐτοῦ δωρεῶν ὑμνεῖται, “Among all other things, therefore, God is suitably and more originally named as ‘he who is,’ from the ‘elder’ [or first] among his gifts” (my translation).} moreover, he possesses τὸ εἶναι before all things and in a preeminent way: “Καὶ γὰρ τὸ προεῖναι καὶ ὑπερ\-εῖναι προέχων καὶ ὑπερ\-έχων τὸ εἶναι πᾶν, αὐτό φημι καθ᾿ αὑτὸ τὸ εἶναι, προϋπεστήσατο καὶ τῷ εἶναι αὐτῷ πᾶν τὸ ὁπωσοῦν ὂν ὑπεστήσατο.”\,%
%
\footnote{\Cite[V, 5]{pg:dionysius:DN}: “Also, therefore, possessing pre-being before all, and preeminently having super-being, he pre-caused all being (I mean being itself [αὐτό καθ᾿ αὑτὸ τὸ εἶναι]) and caused everything in every way whatever in its very being” (my translation).} God communicates τὸ εἶναι first, and the other gifts—intelligence and life, for example—must flow from the first; otherwise it is impossible to understand how those creatures who possess only God’s greatest gift (τὸ εἶναι) are inferior to those who also possess life and intelligence: these perfections are communicated by God and entail an increasing participation in God and closeness to him.%
%
\footnote{See \cite[V, 2]{pg:dionysius:DN}.} In fact, not only is τὸ εἶναι prior to the other perfections, it contains them \emph{virtually}: “μᾶλλον δὲ καὶ αὐτὰ καθ᾿ αὑτὰ πάντα, ὧν τὰ ὄντα μετέχει, τοῦ αὐτὸ καθ᾿ αὑτὸ εἶναι μετέχει, καὶ οὐδὲν ἔστιν ὄν, οὗ μὴ ἔστιν οὐσία καὶ αἰὼν τὸ αὐτὸ εἶναι”\,%
%
\footnote{\Cite[V, 5]{pg:dionysius:DN}: “Moreover, all of the \emph{per-se} realities, in which all \emph{entia} participate, participate in \emph{per-se} being itself, and there is no \emph{ens} whose substance and eternity is not being itself” (my translation).} What is most significant for our topic of discussion is that for Dionysius, “\emph{per-se} being itself” (αὐτὸ καθ᾿ αὑτὸ εἶναι) does not mean God (Saint Thomas’ \emph{Ipsum Esse}), but the gift of being given by God to his creatures.\footcite[54]{orourke:pseudo-dionysius} Hence Dionysius was the first to conceive of τὸ εἶναι as an intensive perfection that entails a greater or lesser participation in the Principle that possesses it \emph{per se}. Moreover, for him τὸ εἶναι, far from denoting the mere “fact” of existence, is the supreme perfection that is the source of all the others. 

\subsection{The Intrinsic Principles of \emph{Ens}}

Also central to our problem is the relationship between the intrinsic principles discovered by Aristotle: οὐσία and ἐνέργεια. It is difficult to find in the \emph{Metaphysics} a clear hierarchy between the two. In Θ, 6, he equates actuality with existence (τὸ ὑπάρχειν): “ἔστι δὴ ἐνέργεια τὸ ὑπάρχειν τὸ πρᾶγμα μὴ οὕτως ὥσπερ λέγομεν δυνάμει.”\,%
%
\footnote{\Cite[Θ, 6, 1048a31-32]{aristotle:metaphysics}: “Act is precisely that the thing exists, but not as when we say ‘in potency’\,” (my translation). He goes on to give several examples of act-and-potency pairs so as to make an analogy of proportionality, but without giving a definition. See section \ref{sec:reduction} and table \ref{tab:act-potency-theta-six}, below.}
%
The term τὸ ὑπάρχειν means the “fact of existence,” and hence does not necessarily entail all the richness of τὸ εἶναι, but this description of ἐνέργεια does suggest that τὸ εἶναι is at the root of all act. Hence, although Aristotle insists that the object of the First Science is unique, he never resolves the problem of how to overcome the \emph{reductio ad duo} he has accomplished. Subsequent commentators attempted various solutions, which could be grouped into four general types: those who affirm what we will call the metaphysics of essence, or “essentialism,” who tend to “fuse” οὐσία and ἐνέργεια; those who espouse what we could call the metaphysics of dual act, which seeks to separate the two principles radically, treating them almost as two \emph{res}; those who hold what we could call the metaphysics of action, which places the emphasis on \emph{operari}, as with Blondel’s \emph{l’Action} and “transcendental” Thomism; and finally those who place the emphasis on \emph{esse}, as in the “intensive” school represented by Étienne Gilson and Cornelio Fabro.%
%
\footnote{One of the objectives of this chapter is to show that this last interpretation is the most faithful to Thomas Aquinas himself. For an overview of the history of the interpretation of \emph{esse} and an introduction to the metaphysics of intensive being, see \cite{fabro:intensive}.}

\subsubsection{The Metaphysics of Essence}

In the first group, we could include Boethius, whose attempt to translate Aristotle’s works into Latin was tragically cut short by his death at the hands of Theodoric, king of Ostrogoths.%
\footnote{See \cite[458]{fabro:intensive}.}
%
In his translations, he sometimes interpreted τὸ εἶναι as \emph{essentia};%
%
%\suppressPagesNextCite%
\footnote{See, for example, his translation of Aristotle’s \emph{Categories} \parencite[285D–286C {[=Μ, 14b10–25 in Aristotle]}]{boethius:categories}.}
%
and since \emph{essentia} was commonly used already in ancient times to translate οὐσία,%
%
\footnote{Seneca made the first recorded translation of οὐσία using the term \emph{essentia}, in analogy with the Greek etymology. See \cite[339-340]{gilson:letre}. Boethius himself translated οὐσία with \emph{substantia}.} an identification was naturally made between τὸ εἶναι (\emph{esse}) and οὐσία (\emph{essentia} or \emph{substantia}). Boethius seems to have confirmed this interpretation with a famous expression from his \emph{De Hebdomadibus}: “Diversum est esse et id quod est; ipsum enim esse nondum est, at vero quod est accepta essendi forma est atque consistit.”\,%
\footcite[1311B]{boethius:dehebodmadibus}
Here, \emph{id quod est} is what we could call the \emph{suppositum}, the concrete individual, the compound with all of its parts. \emph{Esse} here does not mean the \emph{actus essendi}, but can be identified rather with the form. Hence, \emph{essentia}—which equates to τὸ εἶναι—is the abstract term for concrete \emph{esse}—roughly the Thomistic concept of \emph{esse commune}, albeit only on the “predicamental” level.%
%
\footnote{For a concise explanation of Boethius’ distinction between \emph{esse} and \emph{id quod est}, see \cite[105]{gilson:history}, and \cite[83–85]{definance:etre-et-agir}.}  In the language of Scholasticism, therefore, there is no real distinction between between \emph{esse} and \emph{essentia}, because the latter “contains” the former.

Boethius’ metaphysics was, therefore, one of essences in act, a model followed by the tradition that we could call the metaphysics of essence or “essentialism.” For example, Ibn Rushd (Averroes), in reaction to Avicenna, denied that existence could be an “accident” of essence, on the grounds that it is absurd to consider existence as one of the nine genera of accidents, much less a special category of accident common to all ten categories. Hence, \emph{esse} is not a “perfection,” he argues, but merely the \emph{copula}, and therefore existence is really simply an \emph{ens rationis}.\footcite[See][68–69]{gilson:letre} Henry of Ghent, on the other hand, considers that essences have a certain consistency (\emph{certitudo}, \emph{ratitudo}) even in the mind of God before they are created; they possess, that is \emph{esse essentiae}, the “being” proper to essence. When God creates, he simply grants the essence \emph{esse existentiae}. For Henry, therefore, \emph{esse} is simply a relation (\emph{respectus}) to the creator. For this reason, he specifically rejects any theory that posits a real distinction between essence and \emph{esse}, particularly that of Giles of Rome, but also that of Thomas Aquinas.%
%
\footnote{See \cite[474–475]{copleston:history:02}. For a summary of Henry of Ghent’s consideration of \emph{esse essentiae} and \emph{esse existentiae}, see \cite[450–451]{gilson:history}, and \cite[6–7]{stanford:henry-of-ghent}.} Duns Scotus does not deal with the problem of the real composition at length, but he distinguishes between \emph{esse existentiae} and \emph{esse essentiae}, which he describes as \emph{modi} of \emph{esse}—differences in the “intensity” of \emph{esse}, much as \emph{finitus} and \emph{infinitus}, for Scotus, are \emph{modi} of \emph{ens}.%
%
\footnote{For a concise description of Scotus’ modal distinction, see \cite[25–26]{companion:scotus}.} He defines \emph{ens} in such a way that it includes not only real, actual things but also \emph{possible} things; as a practical matter, \emph{ens} is reduced to \emph{essence}, which is either merely possible—in which case it has \emph{esse essentiae}—or real—in which case it has  \emph{esse existentiae}.%
%
\footnote{See \cite[q.~3, {[2]}, n.~1]{scotus:quodlibet}: “\emph{Ens ergo vel res} isto primo modo accipitur omnino communissime, et extendit se ad quodcumque quod non includit contradictionem, sive sit ens rationis, hoc est praecise habens esse in intellectu considerante, sive sit ens reale, habens aliquam entitatem extra considerationem intellectus. Et secundo, accipitur in isto membro minus communiter pro ente quod \emph{habet vel habere potest aliquam entitatem} non ex consideratione intellectus” (emphasis added). Note that he practically equates \emph{ens} and \emph{res}; this equivalence shows how \emph{ens} practically reduces to essence. Moreover, \emph{ens} can be not only \emph{ens in actu}, but even \emph{ens possibile} (\emph{quod habere potest aliquam entitatem}).}  Finally, F. Suárez is well known for denying the real distinction between existence and essence, on the grounds that having two principles in \emph{ens} would amount to having two \emph{res}.%
%
\footnote{See \cite[148–149]{gilson:letre}. Suárez was not entirely wrong, in my opinion, in making this critique. The real distinction that he knew was undoubtedly the one espoused by Cajetan. Since Cajetan proposes two original acts, as we will see below, the two principles really \emph{do} appear to be two \emph{res}.} In all of these philosophers, as can be seen, the emphasis of metaphysics is placed nearly entirely on essence; whether it exists or not almost seems to make no difference.

\subsubsection{The Metaphysics of Dual Act}

Paradigmatic of the second group is Ibn Sīnā (Avicenna),%
%
\footnote{For a summary of Avicenna’s doctrine on \emph{ens}, see \cite[124–132]{gilson:letre}. See also \cite[375–376]{contat:etant}.}
%
who interprets τὸ τί ἦν εἶναι (quidditas) as \emph{esse proprium}, and τὸ εἶναι or τὸ ὑπάρχειν (\emph{existentia}) as \emph{esse affirmativum}.%
%
\footnote{See \cite[Tract.~I, sec.~5, {[31]}, 34-35]{avicenna:metaphysics}: “Dico ergo quod intentio entis et intentio rei imaginantur in animabus duae intentiones; ens vero et aliquid sunt nomina multivoca unius intentionis nec dubitabis quin intentio istorum non sit iam impressa in anima legentis hunc librum. Sed res et quicquid aequipollet ei, significat etiam aliquid aliud in omnibus linguis; unaquaeque enim res habet certitudinem qua est id quod est, sicut triangulus habet certitudinem qua est triangulus, et albedo habet certitudinem qua est albedo. Et hoc est quod fortasse appellamus \emph{esse proprium}, nec intendimus per illud nisi intentionem \emph{esse affirmativi}, quia verbum ens significat etiam multas intentiones, ex quibus est certitudo qua est unaquaeque res, et est sicut \emph{esse proprium} rei” (emphasis added).} Avicenna is well known for considering \emph{esse affirmativum} as an “accident” of \emph{esse proprium}, in the sense that (at least in creatures), the existence is not included in the very essence of a thing but rather “happens” (\emph{accidit}) to it—in other words, all creatures are “possible” (contingent, in Thomistic parlance).%
%
\footnote{See \cite[190-192]{gilson:history} and \cite[68]{gilson:letre}. It is perhaps unfair to accuse Avicenna of considering existence as an accident, in the same sense as the nine non-essential categories are accidents; he seems to have understood that it transcends the categories somehow. He did not, however, consider \emph{esse} the source of all proper perfections, as Thomas did.} Hence, unlike Boethius and Averroes (who critiqued Avicenna on this point), he considers the two principles to be really distinct, and that \emph{esse affirmativum} has a certain priority over \emph{esse proprium}.\footcite[See][191]{gilson:history} Nevertheless, both are \emph{esse}, and hence a type of act; therefore, the composition is between an “existential” act and a “formal” act. Avicenna introduces this distinction because of an old problem raised by Protagoras: we can conceive of things without knowing whether they really exist or not. \emph{Esse affirmativum}, argues Avicenna, founds the \emph{esse} of judgments (which explains its name) and \emph{esse proprium} designates \emph{what} a thing is.%
%
\footnote{This argument is taken up in Thomas Aquinas’ earlier works, as we will see.} It follows that, although \emph{esse affirmativum} has the priority, a complete \emph{reductio ad unum} is impossible.

A similar doctrine was formulated by Thomas de Vio (known as Cajetan), the famous commentator of Thomas Aquinas’ \emph{Summa theologiae}.%
%
\footnote{Regarding late-Scholastic and neo-Scholastic interpreters of Saint Thomas, See \cite{contat:figure}, especially the portions on Jacques Maritain and John of Saint Tomas (98–115); \cite{fabro:obscurcissement}; and \cite[604–628]{fabro:partecipazione}, an expanded version of the article in \emph{Revue Thomiste}.} Taking his cue from the Thomistic maxim \emph{forma dat esse}, he maintains that the essence (whose more noble part—among composed substances—is the form) must be a kind of act. Naturally, \emph{esse} is also an act, since it is the \emph{actus actuum}, although he interprets \emph{esse} as \emph{existentia}, the mere “fact” of being or “placement” of the essence outside its causes.%
%
\footnote{In this regard, Cajetan’s conception of \emph{esse} is very similar to that of the “essentialists,” particularly that of Scotus. See \cite[272]{gilson:cajetan}.} There are, therefore, \emph{two} acts and \emph{two} potencies in \emph{ens}:
%
\begin{quotation}
Sicut enim duplex est actus, scilicet esse et forma, ita duplex est potentia receptiva, scilicet essentia et materia. Et ita duplex est receptio et irreceptio: et similiter duplex est finitas et infinitas.\footcite[73b, X]{cajetan:commentaries}
\end{quotation}
%
Although he vigorously defends the real distinction between \emph{existentia} and \emph{essentia}, he borrows from Scotus the notion of \emph{esse} as \emph{esse existentiae actualis}.\footcite[272–273]{gilson:cajetan} Nevertheless, because he considers essence to be an original act that is communicated to the \emph{ens} directly by God, it is impossible for the \emph{existentia} to do more than “actuate” the essence; that is, essence already has a consistency, in a way, “before” it is “placed” into existence outside its causes.%
%
\footnote{See \cite[274]{gilson:cajetan}. We can illustrate the system with an image—valid for both the metaphysics of essence and that of dual act—influenced by a passage in C.S.~Lewis’ \emph{Lion, the Witch, and the Wardrobe}: we are to imagine a storehouse with stone statues. These represent the essences “before” they are created, as they “exist” in God’s mind. When God decides to create one of them, he “breathes” (communicates) his existence, and they come to life. As can be seen, there are two distinct participations involved: the exemplar (statue) that participates in God’s essence by imitation, and the existence (breath) that participates in God as \emph{Ipsum Esse}. In Cajetan’s system, these participations are irreducible. To see an example of this precise argument, see \cite[36–73]{geiger:participation}.} A common thread in this second group is the reduction of \emph{esse} to \emph{existentia}, which becomes practically a formal, almost a univocal notion: all it does, so to speak, is make the essence exist. The resulting conception of \emph{ens} is rather static: since a substance’s \emph{existentia} cannot communicate itself, it follows that the accidents must also be composed of existence and essence and, in fact, be created directly by God, in accord with a certain aptness (\emph{aptitudo} or \emph{dispositio}) in the substance.\footcite[See][278]{gilson:cajetan} Even \emph{operari} must be brought about by divine motion.%
%
\footnote{In this context, one understands why Baroque theologians such as Domingo Báñez and Luis de Molina had a difficult time understanding how actual grace could coexist with human freedom. If the actuation of the will depends on a divine motion for its very existence, and God’s actions never fail, Báñez’ theory of \emph{praemotio physica} seems almost inescapable: it is a case of the problem of \emph{concursus divinus} (two coinciding efficient causes, divine and created). De Molina was forced to assert, in essence, that God ordains the divine motion \emph{in genere}, leaving room for man to “specify” it. It seems to me that the problem can be entirely avoided by following the position described in this paper: the soul’s powers, inasmuch as they are \emph{accidentia propria}, receive their \emph{esse} ultimately from the substance’s \emph{actus essendi}, and they can be the subject of \emph{esse accidentale} received from outside. We touch on this topic briefly in section \ref{sec:grace}, below. For an overview of the \emph{De auxiliis} controversy, see \cite[342–344]{copleston:history:03}.}
%
In fact, the positions of both the metaphysics of essence and of dual act place their emphasis on the essence, and as a result both systems suffer from a similar “staticness” that makes it difficult for them to explain \emph{operari}.%
%
\footnote{In my opinion, if we are going to place Plato in one of these three groups, he would best fit among those who make \emph{esse} a “formal” reality (that is, the metaphysics of essence), even though some characteristics of his metaphysics are unique, in particular his quasi-hypostatization of the ideas. However, he has the following affinities to the metaphysics of essence: first, in his system, the fundamental realities—the ideas—are above all \emph{formal}, not actual; second, \emph{ens} is for all intents and purposes \emph{extrinsic} to what we would call essence (namely, the formal contents of the ideas). It is like another property or “accident” of an essence.}

\subsubsection{The Metaphysics of Action}

Cajetan’s interpretation of Thomas became practically the dominant one among Thomists, and persisted with different variations (especially as regards the priority of one act over the other) until the twentieth century, when various scholars—in large part thanks to the insight of M. Heidegger regarding the priority of being—sought to place the emphasis of metaphysics on actuality, not essence. These authors, who constitute what I call the metaphysics of action, agree with Heidegger that the other schools (the metaphysics of essence and classical Thomism) place an inordinate emphasis on essence and hence are insufficient to explain \emph{esse} and \emph{operari}.
M. Blondel (in anticipation of Heidegger) well represents this current, even though he is more interested in theodicy than the intrinsic structure of \emph{ens}. Convinced that modern philosophy’s experiment with the rationalism had failed, he proposed a philosophy based, no longer on the \emph{via intellectiva}, but on the \emph{via volitiva}.%
%
\footnote{For an overview of the philosophical background of Blondel’s works, see \cite{nicolosi:azione}, especially 263–270.}
%
 Whereas Descartes considers the most evident and fundamental notion to be the \emph{cogito}, Blondel considers it to be the \emph{volo}, the fact of the freedom of the will. He distinguishes between the \emph{volonté voulante}, signifying the will itself, which desires an infinite but unreachable ideal goodness; and the \emph{volonté voulue}, which signifies the real object of the will, those finite, limited, and ultimately unsatisfying goods that man \emph{actually} obtains. Hence, man is internally conflicted by an interior dialectic between the limited goods he can really obtain and the unrealizable ideal that he actually wants. Blondel’s reasoning leads to what he calls \emph{l’antibolie}:%
%
\footnote{Blondel explains this notion in \cite[323–324]{blondel:action}. See also a description in \cite[274–276]{nicolosi:azione}.}
%
if there really is an Absolute capable of fulfilling the ideal, then man’s life has meaning; otherwise, life is absurd, because no number of limited goods can possibly give it meaning. Since the second option is repugnant, the ideal must in fact exist. In Blondel’s own words:
%
\begin{quotation}
Dans notre connaissance, dans notre action, il subsiste une disproportion constante
entre l’objet même et la pensée, entre l’œuvre et la volonté. Sans cesse l’idéal conçu
est dépassé par l’opération réelle, et sans cesse la réalité obtenue est dépassée par
un idéal toujours renaissant. Tour à tour, la pensée devance la pratique, et la
pratique devance la pensée ; il faut donc que le réel et l’idéal coïncident, puisque cette identité nous est donnée en fait ; mais elle ne nous est donnée que pour nous échapper aussitôt.\footcite[344–345]{blondel:action}
\end{quotation}
%
At its root, this reasoning constitutes an ontological argument, very similar in structure to Descartes’, albeit using the will; its starting point is therefore entirely immanent.%
%
\footnote{There are profound structural similarities between Blondel’s system and that of Heidegger and Sartre. Heidegger’s \emph{Sein} is like Blondel’s unrealizable ideal, and the limited goods correspond to \emph{Seiendes}. Similarly, the real but limited goods would correspond to Sartre’s \emph{être-en-soi}, whereas the ideal would correspond to the impossible \emph{être-en-soi-pour-soi} (that is, God). Sartre, of course, essentially chooses the opposite side of Blondel’s \emph{antiboulie}, opting for the absurd. Regarding Sartre’s rejection of the existence of God, see \cite[32–35]{lucas:orizzonte}. Regarding the \emph{être-en-soi} and \emph{être-pour-soi}, see \cite[115–119]{definance:essai}.

While it seems to me that Blondel’s argument suffers from the same problem that all “ontological” arguments face—namely, that it starts with the immanent acts of the soul and hence is unable to demonstrate the reality of anything outside the intellect and will—it would still be valid as an argument of fittingness or \emph{convenientia}. In proving the incorruptibility of the soul, Thomas says,
“Potest etiam huius rei [i.e., incorruptibilitatis animae] accipi signum ex hoc, quod unumquodque naturaliter suo modo esse desiderat. Desiderium autem in rebus cognoscentibus sequitur cognitionem. Sensus autem non cognoscit esse nisi sub hic et nunc, sed intellectus apprehendit esse absolute, et secundum omne tempus. Unde omne habens intellectum naturaliter desiderat esse semper. \emph{Naturale autem desiderium non potest esse inane}. Omnis igitur intellectualis substantia est incorruptibilis” (emphasis added).
A natural desire is, however, only a \emph{sign} of the soul’s incorruptibility. The profound reason is the simplicity and nobility of the soul.}
%
Blondel applies a very similar argument to being in his chief metaphysical work, \emph{L’être et les êtres}. Being (\emph{l’être}), he argues, is an unalterable given, so much so that nothingness cannot even be thought of.%
%
\footnote{See \cite[8]{blondel:etre}: “En ce sens, antérieur à toute doctrine et à toute volonté, il n’est point de nihilisme possible. L’être ne fait point question pour qui est ; et ce qu’on appelait naguère la présence totale de l’être à lui-même élimine la possibilité de sortir du réel, fût-ce pour la pensée la plus experte en critique, en négation, en destruction.” The point is explained more fully in \cite[37–38]{blondel:etre}.}
%
The mind is confronted with an expectation of unity and permanence, but a reality that is manifold and contingent.%
%
\footnote{See \cite[67–68]{blondel:action}: “D’un côté nous paraissons trouver l’être en nous et autour de nous dans les réalités qui s’imposent à notre expérience et à notre action ; comme si c’était de ces données impérieuses que nous abstrayions ensuite une notion plus ou moins généralisée et estompée de l’être. — Mais d’autre part nous ne pouvons nous empêcher de conférer à l’être des attributs tout à fait différents de ceux que les réalités expérimentées en nous et hors de nous nous suggèrent. Ces notions mêmes et les principes qui servent à les mettre en œuvre et à susciter le développement de notre connaissance sont transcendants à l’ordre immanent de ce monde : idée d’unité, de permanence, d’absolu, d’autonomie, de cause productrice de finalité suprême, de substantialité et de perfection, voilà des évidences qui, avant même d’être explicitement reconnues et attribuées à un être, à «~l’Etre en soi et par soi~», sont réellement et nécessairement impliquées en nous et pour nous.”}
%
This dialectic is, of course, analogous to that between the ideal of the \emph{volonté voulante} and the reality of the \emph{volonté voulue}, and it is never resolved intrinsically to \emph{l’être}, but only in \emph{l’Etre absolu} (God).%
%
\footnote{Blondel’s ontological argument for \emph{l’Etre absolu} is summed up in the introduction to the second part of his work, in three questions. See \cite[149]{blondel:action}: “[L]’Etre absolu est-il vraiment conçu par nous~? [\ldots] [L]’Etre absolu peut-il être effectivement affirmé~? [\ldots] [E]st-il possible et comment est-il obligatoire de reconnaître en cet Etre le Dieu de charité~?”}
%
For Blondel, the most authentic or fundamental meaning of the term \emph{l’être} is that of the active-voice verb, and especially not that of the noun or the “result” considered passively:
%
\begin{quotation}
[L’Etre] c’est une source de réalité et, s’il y a dans la pensée priorité au point de vue de l’émergence préalable à la réflexion, c’est donc à la spontanéité du verbe actif, plutôt qu’à l’enregistrement du fait d’exister, que doit être reconnue cette originalité foncière.\footcite[48–49]{blondel:etre}
\end{quotation}
%
In other words, for Blondel, the “act” of being is much more important than the “fact” of being.%
%
\footnote{Together with Fabro, we assume the notion—well explained here by Blondel—of \emph{esse ut actus} or \emph{actus essendi} as “emergent” act. The difference between the metaphysics of “intensive” \emph{actus essendi} and the metaphysics of action is that the former (which we are assuming in this paper) considers \emph{actus essendi} to be an intrinsic principle that mediates between \emph{Ipsum Esse} and the “fact” of being (\emph{esse in actu}).}
%
It should be noted, however, that for Blondel, as for all the authors of the metaphysics of action that we are considering, the act of being (\emph{l’être} as action) cannot truly be characterized as an intrinsic principle of \emph{ens}, but only something that it \emph{does}.

The “transcendental” school of Thomism places a similar emphasis on \emph{operari}. J.~Maréchal, for example, seeks to overcome Kant’s immanentism by analyzing the dynamics of the intellect. Man, he argues, is driven toward knowledge of what is absolute (literally “unbound”) and unconditional, just as by his volition he implicitly desires beatitude. In the affirmation of \emph{esse}, therefore, man has a certain anticipation or foretaste of God.%
%
\footnote{This doctrine is found especially in \cite{marechal:cahier05}. For a concise summary on Maréchal, see \cite[265–269]{copleston:history:09}. For a summary of his theory of knowledge, see \cite{daros:apriori}, especially 404–403. }
%
J.B.~Lotz’ point of departure, on the other hand, is the act of affirmation, of which he considers the conditions of possibility.%
%
\footnote{For an overview of Lotz’ ontology, see \cite[116–119]{contat:quarta-via}; also \cite[216–229]{contat:confronto}.}
%
Focusing (like Blondel) on the \emph{copula}—the verb \emph{to be} that joins the subject and predicate in a proposition—he deduces (like Maréchal) that each affirmation “anticipates” an absolute and unlimited “horizon,” which encompasses all of the possible \emph{ennunciati} allowed by the principle of non-contradiction.%
%
\footnote{See \cite[117–118]{contat:quarta-via}: “Di conseguenza, la condizione ultima di possibilità di giudizio è l’anticipazione di un orizzonte senza il quale il finito non può manifestarsi alla coscienza, ma che non è finito in se stesso.”}
%
In this way he hopes to overcome (in a process not unlike Hegel’s \emph{Aufhebung}) the critique of Kant and in this way “recover”—through a recapitulation of philosophical history—a position of Thomistic realism.%
%
\footnote{This doctrine is exposed especially in \cite{lotz:esperienza}.}
%
In contrast to Cajetan’s position, and in harmony with metaphysics of “intensive” \emph{actus essendi}, transcendental Thomism sees essence fundamentally as a type of potency.%
%
\footnote{Cajetan did not deny that essence is a potency, as we saw, but he did not recognize that it receives its actuality from the \emph{actus essendi}. Its actuality, for him is \emph{original}—not so for Lotz.}
%
Its function, however, is chiefly negative: it imposes a limit on \emph{esse}, which otherwise would be free to expand infinitely so as to fill the entire “horizon.”\,%
%
\footnote{See \cite[104–105]{lotz:esperienza}: “Però, questa espressione [\emph{esse commune}] non va intesa nel senso di un concetto universale, ma come una \emph{pienezza assoluta} od onnicomprensiva, che per se stessa non ammette aggiunta alcuna (\emph{sine additione}) o non implica nulla di determinato.” The key expression is \emph{pienezza assoluta}: for Lotz, \emph{esse commune} is not hindered by any sort of determination whatsoever. We will see below that Saint Thomas, at least, would disagree, for he holds that \emph{esse commune} is, in fact, \emph{always} determined by an essence; it is merely \emph{considered} without \emph{additione}.}
%
This limitation produces a sort of “nostalgia” in the substance for the limitless \emph{esse} that it is unable to reach, which impels it to “recover” its “lost” \emph{esse} through \emph{operari}.%
%
\footnote{See \cite[227-228]{contat:confronto}.}
%
The relationship among \emph{actus essendi}, \emph{essentia}, and \emph{operari} is best described as dialectic—in the style of Hegel’s thesis, antithesis, and synthesis—rather than one of composition and participation.%
%
\footnote{One cannot help but notice the structural similarity between transcendental Thomism and the various modern systems that employ dialectic: Kant’s (pre-dialectic) system of categories, all of which entail a \emph{positio}, a \emph{negatio}, and synthesis between the two; Hegel’s dynamic of thesis, antithesis, and synthesis; and Heidegger’s dynamic of \emph{Sein}, which “withdraws as it discloses itself\,” in \emph{Seiendes} to \emph{Dasein}. We could say that Lotz’ substance is a “thrown projection” just like \emph{Dasein}: it is \emph{actus essendi} thrown into \emph{essentia} and must labor to recover the unreachable \emph{esse}. It is also interesting that, like Kant’s transcendental ideas and Heidegger’s \emph{Sein}, the fullness of \emph{esse}—the “horizon”—is unreachable by operation, at least by operation alone. Of course, Lotz wishes to be fully Christian, and so he “baptizes” the “horizon of being,” practically identifying it with \emph{Ipsum Esse}.
Regarding this similarity, see \cite{contat:confronto}.}

\subsubsection{The Metaphysics of Actus Essendi}

Finally, the metaphysics of \emph{esse} as “intensive” act—which is the one that we will adopt—recognizes that \emph{actus essendi} is an intrinsic principle that is the source of all actuality in a substance, containing that actuality virtually. Each substance “possesses” an \emph{actus essendi} as its own, to a degree of intensity that is determined by the essence. Hence, essence is revealed as the receptive capacity or measure of \emph{esse} in a substance. The \emph{actus essendi} is dynamic and “fruitful”: it has far more actuality than that required to make the essence “exist,” and hence is able to communicate itself (indeed, it cannot help but do so). In so doing, it produces the various levels of actuality: substance, first of all, then accidents (especially “proper” accidents), and finally \emph{operari}. Although essence imposes a limit and functions as a potency (on the “transcendental” level, as we will see) its role is chiefly a positive one: it \emph{permits} the substance to “receive” its act of being (otherwise it could not be distinct from God). The tendency of the \emph{actus essendi} to expand results in an impetus for the substance to fulfill itself in its operation (in this the metaphysics of “intensive” \emph{actus essendi} is in agreement with the metaphysics of action); however, the final cause of this expansion is the \emph{actus essendi} itself, not an ideal “horizon” of \emph{esse}. Creatures are, so to speak, happy to remain the creatures they are and do not (or should not) seek to exceed the limits imposed by their essence.

As can be seen, each position emphasizes one of the three great metaphysical domains that arise from the three types of extrinsic causality: \emph{esse}, \emph{essentia}, and \emph{operari}. The essentialist and classical Thomist positions both place the emphasis on \emph{essentia}, and in so doing compromise both \emph{esse} and \emph{operari}. The metaphysics of action emphasizes \emph{operari} at the expense of both \emph{esse} and \emph{essentia}. The metaphysics of intensive \emph{actus essendi}, however, emphasizes \emph{esse}, and in so doing can safeguard both \emph{essentia} and \emph{operari}. Just from this very brief survey of the history of the notion of \emph{esse}, therefore, we can see that \emph{esse} is the keystone of any metaphysical system. We will see, moreover, that it is the metaphysics of \emph{esse} as intensive act that, it seems to me, best characterizes Thomas’ own position regarding \emph{actus essendi} and—more importantly—best explains substance as such, as well as its tendency to seek its own fulfillment.%
%
\footnote{For an overview of how Thomas’ metaphysics of \emph{actus essendi} makes a coherent whole or “Gestalt,” see \cite{villagrasa:gestalt}.}
%
In the remainder of this chapter, we will propose a \emph{resolutio secundum rationis} of \emph{ens} into its radical principles—\emph{esse ut actus} and \emph{essentia}, perform a \emph{compositio} to see how \emph{esse ut actus} is communicated to the various “expansions” of \emph{esse} (substance, accident, operation), and show that the \emph{esse ut actus} can be considered a kind of \emph{quantitas virtualis} or \emph{virtutis} that varies in intensity according to the ontological degree of the substance in question.

\section{\emph{Resolutio Secundum Rationem} of \emph{Ens}}

The \emph{resolutio} of \emph{ens} into its intrinsic principles being proposed requires three stages: first, a logical and semantic analysis that examines how the verb \emph{to be} (εἰμί, \emph{sum}) is used in making enunciations; second, a reduction of the relevant meanings to intrinsic principles that underly them (or, to put it another way, the ontological foundations of the results discovered by the logical analysis); and third a “radicalization” and “intensification” of these principles, so as to discover the ultimate, radical intrinsic causes (which we will discover to be \emph{esse ut actus} and \emph{essentia}). In performing the first two stages we will be following the masterful analysis done by Aristotle on τὸ ὄν in \emph{Metaphysics} Δ,~7, with supporting passages in Books Γ, Ζ, and Θ. The third stage will follow the “intensification” of \emph{esse} performed by Saint Thomas Aquinas, in a number of locations throughout his works.

\subsection{The Fourfold Division of \emph{Ens}}
\label{fourfold-division}

\begin{DONE}
Fourfold division of \emph{ens} (\emph{Metaphysics} Δ,7; \emph{In V Metaph.})
\end{DONE}

We begin our logical and semantic analysis with the famous passage from \emph{Metaphysics} Γ, 2: “τὸ δὲ ὂν λέγεται μὲν πολλαχῶς, ἀλλὰ πρὸς ἓν καὶ μίαν τινὰ φύσιν καὶ οὐχ ὁμωνύμως.”\,%
%
\footnote{\Cite[Γ, 2, 1003a33-24]{aristotle:metaphysics}:  “\emph{Ens}, however, is said in many ways, but in reference to one and only nature, not as a [mere] homonym” (my translation).} As we did with \emph{bonum}, we simply observe that \emph{ens}  is a most fundamental notion—indeed the most fundamental—and that it is grasped \emph{in actu exercito} in every act of knowledge. It is clearly not known explicitly (“thematically” or, to use Scholastic terminology, \emph{in actu signato})—for such a thesis would be tantamount to ontologism—however, simple reflection reveals that the simplest, most incontrovertible fact that the intellect learns about a thing is that it simply \emph{is}. At least logically (not yet ontologically), it is clear that the verb \emph{to be} contains all other types of predication virtually; every declarative sentence could be reduced to the form “\emph{S}~is~\emph{P}.”\,%
%
\footnote{The English language is probably clearer in this regard than Latin and its derivatives, or even Aristotle’s Greek, because of the frequent use of the so-called “present progressive” tense. What in French would be “Étienne dort,” or in Italian “Stefano dorme,” English would render “Stephen \emph{is} sleeping.”} It follows that the notion of \emph{ens} (\emph{ratio entis}) acts as a mediator, logically speaking, for every other notion; in other words, every other \emph{ratio} must “add” something to \emph{ens} (including \emph{bonum}, as we saw in the previous chapter). Every predication is an explicit or implicit use of \emph{to be}, but clearly not every predication has the same value: it is not the same thing to say “John is a man” as to say “John is sleeping,” or even “Fifi is a cat.” Nevertheless, there clearly is something in common. An analogy of proportionality can be be made between “John is a man” and “Fifi is a cat” or between “John is sleeping” and “Fifi is sleeping;” and even between “Fifi is sleeping” and “Fifi is black in color.” Aristotle implicitly makes these types of analogies when he gives the following description of τὸ ὄν:

\begin{quotation}
ὄντα λέγεται, τὰ δ᾽ ὅτι πάθη οὐσίας, τὰ δ᾽ ὅτι ὁδὸς εἰς οὐσίαν ἢ φθοραὶ ἢ στερήσεις ἢ ποιότητες ἢ ποιητικὰ ἢ γεννητικὰ οὐσίας ἢ τῶν πρὸς τὴν οὐσίαν λεγομένων, ἢ τούτων τινὸς ἀποφάσεις ἢ οὐσίας%
%
\footnote{\Cite[Γ, 2, 100ba8-10]{aristotle:metaphysics}: “For some things are said to ‘be’ because they are substances; others because they are modifications of substance; others because they are a process towards substance, or destructions or privations or qualities of substance, or productive or generative of substance or of terms relating to substance, or negations of certain of these terms or of substance” (translation from \cite{aristotle:metaphysics:en}).}
\end{quotation}
%
As Aristotle observes, every use of the verb \emph{to be} is in some way related to what he calls οὐσία (“substance,” in this context), referring to what Aquinas called \emph{ens simpliciter}: the things that can be said to “be” without qualification, without a need for a substrate to sustain them: for example, rocks, trees, animals, and men. All other uses of “to be” imply a greater or lesser ontological “distance” from οὐσία, ranging from “zero” (οὐσία itself) to practically infinite (when one says that non-being is non-being). From an initial analogy of proportionality, therefore, Aristotle deduces an analogy of reference (what Cajetan would call an analogy of attribution): τὸ ὄν in its manifold meanings refers to \emph{one} (πρὸς ἓν); that is, to οὐσία.%
%
\footnote{It remains to be seen whether the analogy of reference is merely “extrinsic” (if \emph{ens} is to be found only in οὐσία, the way “health” is found—strictly speaking—only in the animal) or whether it is “intrinsic” and hence entails a participation of more “distant” \emph{entia} in οὐσία.}

In order to study the relation among the various meanings of τὸ ὄν in greater detail, it is necessary to turn to Aristotle’s much more systematic analysis found in \emph{Metaphysics} Δ, 7: the so-called fourfold division of τὸ ὄν.%
%
\footnote{The passage is found at \cite[Δ,~7, 1017a7-1017b9]{aristotle:metaphysics}. Also of note is \cite{brentano:several}, especially the chapters regarding τὸ ὄν as predicated according to the figures of the categories, pages 91–194.}
%
Aristotle begins by noting that τὸ ὄν can be predicated either \emph{per accidens} (κατὰ συμβεβηκός) or \emph{per se} (καθ᾽ αὑτό). As Thomas Aquinas notes in his commentary, by κατὰ συμβεβηκός, Aristotle does not mean the division into substance and accident (which, in any case, he categorizes as a \emph{per se} predication further on).%
%
\footnote{See \cite[lib.~5, l.~9, n.~1 (Marietti n.~885)]{st:metaph}: “Sciendum tamen est quod illa divisio entis non est eadem cum illa divisione qua dividitur ens in substantiam et accidens. Quod ex hoc patet, quia ipse postmodum, ens secundum se dividit in decem praedicamenta, quorum novem sunt de genere accidentis.”} Based on the examples he gives, we can see that Aristotle is referring to the “accident of predication”: statements of fact that do not imply a necessary link between the subject and predicate. When, on the other hand, τὸ ὄν is predicated \emph{per se}, the link is necessary.%
%
\footnote{For example, when I say, “The musician is building a house,” the predication is \emph{per accidens}, because it is not necessary in any way for the builder to be a musician; he just “happens to be” one. On the other hand, “The builder is building” is a \emph{per se} predication, because one needs training—that is, one needs to be a builder—in order to build a house properly. Some statements are “mixed;” that is, there is a link of necessity only from a certain point of view: for example, if I say, “The musician is playing a piano sonata by Beethoven,” there is nothing stopping the musician from playing instead a piano sonata by Mozart (hence in this respect the statement is \emph{per accidens}), but only a musician can play a piano sonata properly (and hence in this latter respect the statement is \emph{per se}). A statement can be part of a science as long as there is some necessity—as long as it is \emph{per se} to some degree.} Aristotle divides this type of predication into three:%
%
\footnote{Throughout this passage, Aristotle subtly switches between τὸ εἶναι and τὸ ἔστιν, in addition to one use of τὸ ὄν. This alteration shows that he is, linguistically speaking, analyzing the use of the conjugated verb εἰμί. He will eventually show that it can be reduced through analogy to οὐσία and ἐνέργεια.} it can signify what is indicated by the “figures of predication” (τὰ σχήματα τῆς κατηγορίας)—that is, the ten categories; it can signify truth or falsehood;%
%
\footnote{This is what we could call the “logical” use of the copula—the only meaning accepted, for example, by Kant. It reflects the use of the operation of composition or division. In this sense, “is” signifies a composition, and “is not,” a division.} or it can signify act and potency.%
%
\footnote{What Aristotle means by this is that sometimes we predicate \emph{to be} of things that are right now (for example, “The builder is building”) and sometimes of things that could be in the future (“Hermes is in the block of marble”)—in other words, things that are \emph{in actu} or \emph{in potentia}.}

Since a science is necessary, certain, and universal knowledge of an object’s causes and properties, it follows that τὸ ὄν predicated \emph{per accidens} cannot be the object of any kind of science, least of all the First Science.%
%
\footnote{See \cite[Ε,~2, 1026b1-3]{aristotle:metaphysics}: ἐπεὶ δὴ πολλαχῶς λέγεται τὸ ὄν, πρῶτον περὶ τοῦ κατὰ συμβεβηκὸς λεκτέον, ὅτι οὐδεμία ἐστὶ περὶ αὐτὸ θεωρία, “Precisely since τὸ ὄν is said in many ways, it is first to be said about it [when it is predicated] \emph{per accidens}, that in no way can there be speculation about it” (my translation).}
%
Moreover, Aristotle argues that \emph{ens} as true and \emph{non-ens} as false (τὸ δὲ ὡς ἀληθὲς ὄν, καὶ μὴ ὂν ὡς ψεῦδος), which refers to something—namely, composition and division (σύνθεσις καὶ διαίρεσις)—that occurs in the intellect, is not a “proper” meaning of τὸ ὄν at all. We might say that truth and falsehood belong to the science of gnosiology or epistemology, not the First Science, which studies \emph{ens} as such.%
%
\footnote{We might also use a slightly different strategy, which seems to have been followed by Saint Thomas: there is a type of truth that is intrinsic to \emph{ens}, and in fact is coextensive with it, just like \emph{bonum}. We could call it the intrinsic intelligibility of \emph{ens}, the \emph{verum} as a transcendental. Of course, the most proper meaning of “truth” is the “formal” kind, the \emph{adaequatio} of the intellect to reality, and this is indeed the subject of epistemology. However, as we saw above, \emph{verum} is founded on \emph{ens} and in that sense reduces to it. See \cite[q.~1, a.~1, co.:]{st:deveritate}: “Illud autem quod primo intellectus concipit quasi notissimum, et in quod conceptiones omnes resolvit, est ens.”}

\subsection{Reduction to the Intrinsic Principles}

\subsubsection{Reduction of \emph{Ens} to Substance}

We are left, therefore, with \emph{ens} as divided into the ten categories and as act and potency. We have already seen that Aristotle in Γ, 2, considers οὐσία as the most proper meaning of τὸ ὄν, with the other categories implying a greater or lesser “distance” from οὐσία. This doctrine is verifiable by simple reflection on sensible experience; for example, we never see colors, shapes, or sizes independently of some substrate, but we do see trees, animals, stones, and men—all of which exhibit properties such as color, shape, and size. Nevertheless, Aristotle provides a more rigorous justification in Ζ,~1.%
%
\footnote{See \cite[Ζ,~1, 1028a10–1028b8]{aristotle:metaphysics}. These texts are not chosen at random from the \emph{Metaphysics}; it seems clear that Aristotle himself connected them, because all of them—Γ,~2; Δ,~7; Ε,~2; and Ζ,~1—begin with a variation on “τὸ ὂν λέγεται πολλαχῶς.”} He argues that the essential “properties” of οὐσία are its being τόδε τι (literally, “this what”) and χωριστόν (“separately”): in other words, substance (οὐσία) is both a definite, determined thing and something that can stand by itself. It is “primary” precisely because none of the other categories can do so. Therefore, Aristotle argues, οὐσία enjoys a priority in every respect: “ὅμως δὲ πάντως ἡ οὐσία πρῶτον, καὶ λόγῳ καὶ γνώσει καὶ χρόνῳ.”\,%
%
\footnote{\Cite[Ζ,~1, 1028a33]{aristotle:metaphysics}: “But nevertheless, οὐσία is first in every way: in notion, in knowledge, and in time” (my translation).} It is first in “time” in the sense that it has an ontological priority, for, as we noted, only substance can exist separately (and of course, some accidents only appear some time after the substance has been generated). It is first in “notion” because the accidents are always predicated as inherent in a substance; that is, the logical structure of substance and accident follows the ontological. Finally, it is first in knowledge, because knowledge of the accident presupposes knowledge of the substance.%
%
\footnote{This doctrine is very important, because it signals a point of divergence between Aristotle and Thomas, on the one hand, and most of the philosophers since Duns Scotus, on the other. Thanks to their theory of intentional identity, Aristotle and Aquinas both assert that knowledge is first of the \emph{whole}—however confused it may be—and only afterwards of the parts (accidents and so forth). An example illustrates: suppose I awaken in the middle of the night in the dark, and I decide to rise and walk over to my desk. Suppose that on the way, I trip over something on the floor (without injuring myself). At the moment of contact, I have no idea what the object is, but I do know immediately and infallibly that \emph{something} has made me trip. Until I examine it and reflect on it, I know practically nothing about it except that it \emph{is}—\emph{ens}, as we saw, is the \emph{primum cognitum}—and only gradually do I fill in the details: it is material, hard, mobile (I heard it slide a few feet after I tripped on it), inanimate (it did not scamper away), and so on. Although I only discover that it is a book—Heidegger’s \emph{Sein und Zeit}, in fact—after I reach down to feel the object, or turn on the light, I know the thing, the \emph{whole} thing (however confusedly), from the beginning.}

\subsubsection{Reduction of Potency to Act}
\label{sec:reduction}

Aristotle makes a similar reduction of potency to act in Book Θ. As with substance, act can be said to be prior to potency in three ways: in notion, in substance, and (sometimes) in time.%
%
\footnote{See \cite[Θ,~8, 1049b11–12]{aristotle:metaphysics}: πάσης δὴ τῆς τοιαύτης προτέρα ἐστὶν ἡ ἐνέργεια καὶ λόγῳ καὶ τῇ οὐσίᾳ: χρόνῳ δ᾽ ἔστι μὲν ὥς, ἔστι δὲ ὡς οὔ, “To every one [of the potencies mentioned above] such as these, act is prior both in notion and in substance; in time, however, it is sometimes, and sometimes it is not” (my translation).} Aristotle argues that the notional priority is evident: potency is understood precisely because it \emph{can be} act; in other words, potency is for the sake of act.%
%
\footnote{Using Aristotle’s own example, “builder” signifies a potency, because it need not be actuated at all times. I know a man is a builder precisely because I know he \emph{can in fact} build houses.}%
%
Aristotle does not explain explicitly why act is prior “in substance”—that is, ontologically—but we can deduce such a priority from what he says about its (partial) priority it time. Act is prior in time, in the sense that only a thing \emph{in actu} can reduce a potency to act (in something else). Using Aristotle’s own example, it takes a (different) full-grown plant to produce a seed; in this case, something \emph{in actu} (the full-grown, fully developed plant) is chronologically prior to something \emph{in potentia} (the seed). It takes an individual that has reached the fullness of the form allowable by its species (the full-grown plant), to produce a new individual (the seed).%
%
\footnote{Aristotle, perhaps because of his reaction against Plato, seems to resist speaking in terms of participation. However, he has stumbled upon an example of what Fabro calls “predicamental participation;” that is, the participation of an individual in its species. It should be noted that the form has the role of “measuring” the maximum extent of this fullness. Regarding participation present at least implicitly in Aristotle, see \cite[307–316]{fabro:partecipazione}, and \cite[456]{fabro:intensive}: “However, one must bear in mind that Aristotle came to realize the inadequacy of the Platonic doctrine of participation only gradually, since he himself had adhered to it in his youthful dialogues of Platonic inspiration. The very titles of these dialogucs (\emph{Eudemus}, \emph{Symposion}, \emph{Eroticus}, \emph{Protreticus}, \emph{Politicus}, \emph{Sophistes}, etc.) seem to bear out this view. Moreover, traces of that doctrine can still be found in the \emph{Organon} where he discusses the logical relation of universals, i.e., of individuals to species and of species to genus. From these latter there arose that important nucleus of the Thomistic doctrine of predicamental participation.”}
%
As can be seen, this chronological priority is, in reality, an ontological one. On the other hand, in a given individual, it is potency that has the priority in time: returning to the example of a seed, the plant starts out as a seed, then germinates, develops, and finally becomes a fully grown plant, whereas in the beginning it is only a full-grown plant \emph{in potentia}.

Aristotle does not discuss ἐνέργεια extensively; most of the relevant passages can be found in Book Θ,~6–8. He recognizes that ἐνέργεια implies an order to an end, because he employs the term ἐντελέχεια (“fulfillment”) synonymously.%
%
\footnote{See \cite[Θ,~6, 1050a23]{aristotle:metaphysics}: διὸ καὶ τοὔνομα ἐνέργεια λέγεται κατὰ τὸ ἔργον καὶ συντείνει πρὸς τὴν ἐντελέχειαν, “Therefore, the term ἐνέργεια is said according to [i.e, is derived from] τὸ ἔργον [‘work’], and leads toward ἐντελέχεια [‘fulfillment’]” (my translation). As can be seen, ἐντελέχεια, derived from τὸ τέλος, has the connotation of “finished product” or “perfection.”} As we saw above, Aristotle practically equates actuality and existence (τὸ ὑπάρχειν), when taken with respect to the “thing” (τὸ πράγμα). This identification is in the context of his description of act in Θ,~6, in which he never arrives at a definition, but rather makes an analogy of proportionality with various examples (\autoref{tab:act-potency-theta-six}).\footcite[See][Θ,~6, 1048a25–1048b35]{aristotle:metaphysics}
%
\begin{table}
  \centering
  \begin{OnehalfSpacing}
    \begin{tabular}{llr}
      \toprule
        \textbf{Active Principle} & \textbf{Potential Principle}        \\
      \midrule
        Existence (τὸ ὑπάρχειν) & Thing (τὸ πράγμα)           \\
        Something actually building  & Something that can build         \\
        Something awake              & Something asleep                 \\
        Something that sees          & Something with eyes shut         \\
        Motion (κίνησις)        & Potency (δύναμις)           \\
        Substance (οὐσία)       & Particular matter (τις ὕλη) \\
      \bottomrule
    \end{tabular}
  \end{OnehalfSpacing}
  \caption{The analogy of act in \emph{Metaphysics} Θ,~6}
  \label{tab:act-potency-theta-six}
\end{table}
%
Aristotle is making a comparison of four general “species” of compositions: existence with “thing” (which from the context is understood as the concrete individual); operation with substance; operation with potency; and form with matter. The first “species” seems to be the “root” of all the others. Nevertheless, Aristotle is not perfectly clear on this point, and he does not investigate the matter much further.

\subsubsection{Reduction of οὐσία to τὸ τί ἦν εἶναι}
\label{sec:aristotle-reduction}

Having established that \emph{ens} can be reduced to substance (οὐσία) and act (ἐνέργεια), Aristotle turns his attention to the precise meaning of οὐσία: this is the central topic of Book Ζ. He must first refute two proposals made by previous philosphers: substance as universal (τὸ καθόλου) and as substrate (τὸ ὑποκείμενον).\footcite[See][Ζ,~3, 1028b33–35]{aristotle:metaphysics} The former refers to Plato, who—as we saw—considered the intelligible, separate ideas (which function as genera or universals for particular, sensible things) to be the primary \emph{locus} of reality. The latter, we might say, brings together the speculations of the Physicists, who sought the underlying ἀρχή of all reality (water, air, ἄπειρον, fire, some combination of these, and so on), which, in Aristotelian terms, means attempting to explain reality only in terms of material causes. Aristotle makes use of the characteristics of substance that he has discovered in Ζ, 1—τόδε τι and χωριστόν—as his criteria. Although Aristotle does not dwell on this fact, the universal fails the test, because although (according to Plato) it is separate (χωριστόν), it is abstract, not τόδε τι.%
%
\footnote{As we saw above, Aristotle rejected the separateness of the ideas anyhow; form (in this case εἶδος, just like Plato’s “idea”) is always the form \emph{of a substance}. See \cite[Ζ,~3, 1029a30–33]{aristotle:metaphysics}.}
%
The substrate (τὸ ὑποκείμενον) fails as well, because it is insufficient: he says, “λέγω δ᾽ ὕλην ἣ καθ᾽ αὑτὴν μήτε τὶ μήτε ποσὸν μήτε ἄλλο μηδὲν λέγεται οἷς ὥρισται τὸ ὄν.”\,%
%
\footnote{\Cite[Ζ,~3, 1029a20]{aristotle:metaphysics}: “But by ‘matter’ I mean what is \emph{per se} neither a ‘what,’ nor a ‘how much,’ nor is anything else meant by which \emph{ens} is defined” (my translation).}
%
In this context, “matter” (ὑλή) is synonymous with “substrate,” but although substrate in a way is “separate” (it is not an accident that inheres in something else), it is also not a “what”—not τόδε τι, not definite. It can only be definite if there is a form to define it. As Aristotle puts it, “ἐκ μὲν οὖν τούτων θεωροῦσι συμβαίνει οὐσίαν εἶναι τὴν ὕλην: ἀδύνατον δέ: καὶ γὰρ τὸ χωριστὸν καὶ τὸ τόδε τι ὑπάρχειν δοκεῖ μάλιστα τῇ οὐσίᾳ.”\,%
%
\footnote{\Cite[Ζ,~3, 1029a28–29]{aristotle:metaphysics}: “Therefore, from this reasoning, it follows that substance is matter; however, this is impossible. For both ‘separately’ and ‘this what’ are considered especially proper to substance” (my translation).}
%

Substance, therefore, can be identified neither with the Platonic καθόλου (at least not primarily), nor with the Physicists’ ὑποκείμενον. Aristotle proposes his own solution: the curious phrase τὸ τί ἦν εἶναι—sometimes shortened to τί ἐστι—which literally could be translated “the being of what it is.” Aristotle uses this because, if οὐσία is both τόδε τι and χωριστόν, then it must be predicable καθ᾽ αὑτό (“according to itself,” or \emph{per se}).%
%
\footnote{See \cite[Ζ,~3, 1029b14-15]{aristotle:metaphysics}: ἐστὶ τὸ τί ἦν εἶναι ἑκάστου ὃ λέγεται καθ᾽ αὑτό. οὐ γάρ ἐστι τὸ σοὶ εἶναι τὸ μουσικῷ εἶναι: οὐ γὰρ κατὰ σαυτὸν εἶ μουσικός, “The essence [τὸ τί ἦν εἶναι] of each is what is said according to itself [\emph{per se}]. Being ‘musical’ is not being ‘you’: for you are not musical according to yourself [\emph{per te}]” (my translation). In the translation, the parallel between τὸ τί ἦν εἶναι and the other uses of the infinitive τὸ εἶναι—τὸ σοὶ εἶναι and τὸ μουσικῷ εἶναι—is lost. The term \emph{per se} (καθ᾽ αὑτό, and its variant for the second person, κατὰ σαυτὸν) used here would seem to refer to what is commonly called \emph{primo modo per se}, in which the predicate is contained analytically in the subject. See \cite[I,~10]{st:post-anal}.}
%
No mere attribute (such as “musical”) will do, but only that which makes a thing what it is; as Aristotle succinctly puts it, τὸ τί ἦν εἶναι is “ὃ ἄρα κατὰ σαυτόν” (“what [you are] according to yourself”), or “what you are inasmuch as you are \emph{you}.”\,\footcite[Ζ,~3, 1029b16]{aristotle:metaphysics} After eliminating the other candidates and seeing that nothing else makes a substance a substance, we conclude that οὐσία is best described as τὸ τί ἦν εἶναι. Aristotle justifies his choice positively in Ζ,~17.\footcite[See][Ζ,~17, 1041a5–1041b31]{aristotle:metaphysics} He notes that in order to be able to ask \emph{why} something is the way it is (seek its causes), we must first verify the fact, and then ask the question “Why?” To use Aristotle’s own example, we first observe a musical man, and then ask, “Why is he musical?” In our case, the “fact” is something that is (τὸ ὄν); by analogy, the question is, “Why is this \emph{ens} what it is?” What particularly interests Aristotle is the unity of compounds: to use his own example, why do bricks and stones make a house? Clearly the house cannot be reduced to its components, for if they were not arranged in a very particular way, they would be something else (perhaps an office building or a formless mass). In a way, it is the house itself (more precisely, the accidental form of the house) that makes these things a house. In a similar way, what makes the various components of \emph{ens}—all of its accidents and operations—a unique substance (οὐσία)? Aristotle argues that the answer can only be τὸ τί ἦν εἶναι (what it is simply because it is), which must, therefore, be identifiable with οὐσία.%
%
\footnote{This is a powerful argument against the nominalism of Ockham, Locke, and Hume: a substance (even an artificial substance) cannot be reduced to its components.}

\subsubsection{Conclusions Regarding οὐσία and ἐνέργεια}

The final result of Aristotle’s investigation into the intrinsic causes of τὸ ὄν is that it can be reduced to οὐσία (specified as τὸ τί ἦν εἶναι) and ἐνέργεια, or better said, to substance in act, which amounts to things (πράγματα) that exist (ὑπάρχοντα). It is important to note that he is not considering \emph{ens} primarily as a \emph{possible} being—as many late Medieval philosophers, such as Henry of Ghent and John Duns Scotus, did—but \emph{ens in actu}. By saying that οὐσία is τὸ τί ἦν εἶναι, he shows that τὸ εἶναι is central to the very notion of substance. Aristotle leaves many questions open: in particular, the relationship between the two principles. As we saw, Boethius and Averroes tended to identify οὐσία and ἐνέργεια, thus paving the way for the essentialism of Henry of Ghent, Scotus, and Suárez, whereas Avicenna tended to separate them. Saint Thomas Aquinas proposed a unique solution that interpreted ἐνέργεια (understood as τὸ εἶναι or \emph{esse ut actus}) and οὐσία as a true and radical act-and-potency pair; this intensification and radicalization of the principles discovered by Aristotle are the topic of the next section.

\subsection{Intensification of the Principles}

Although Aristotle reduces τὸ ὄν to οὐσία and ἐνέργεια, the most radical composition of act and potency that he discusses is that of substantial form and prime matter. Therefore, many interpreters—in particular the Franciscan school, including Saint Bonaventure—posited a universal hylomorphism of all creatures.%
%
\footnote{Authors who adhere to this doctrine generally consider Augustine to be their inspiration. Other important proponents include Solomon Ibn Gabirol (Avicebron), Dominic Gundisalvi, Thomas of York, and the \emph{Summa philosophiae} (apparently falsely) attributed to Robert Grosseteste. See \cite{stanford:binarium}.}
%
Thomas Aquinas has at least two reasons for rejecting this view: first of all, angels are beings without bodies, and so attributing to them a “subtle” matter is incongruous. Second, the matter-form distinction is sufficient to account for multiple individuals in a single species; however, it is incapable of accounting for different ontological “grades,” especially the most evident difference, that between non-rational creatures and man.%
%
\footnote{In order to prove that angels are incorporeal, he shows first that any substance possessing an intellect—which is capable of knowing things in an immaterial way—must itself be immaterial. If there is a creature intermediate (so to speak) between God and man, then it must be incorporeal and hence without matter. See \cite[I, q.~50, a.~1, co., and a.~2]{st:summa}, as well as \cite[208]{fabro:nozione}.
For a discussion of the ontological grades that give rise to this \emph{resolutio}, see \cite[209–212]{izquierdo:vita}; also \cite[38–43]{lucas:hombre}.

We could also add a third, theological reason for the insufficiency the matter-form composition: it does a poor job of explaining man’s intermediate eschatology. How can a man be the same man when his body has been dissolved in death? Unless his soul—now a pure form, albeit \emph{in potentia} toward a body—is \emph{this} essence actuated by \emph{this} \emph{actus essendi}, it would be impossible to distinguish from other souls before the resurrection of the body.}
%

\subsubsection{Arguments Used to Demonstrate the Real Composition}
\label{arguments-real-composition}

Aquinas makes use of three arguments to prove that there is a composition between \emph{essentia} and \emph{actus essendi} that is more radical than the matter-form composition.%
%
\footnote{For a discussion of arguments in favor of the real composition, see \cite[211-215]{fabro:nozione}, and \cite[133–161]{wippel:metaphysical_themes_1}. An excellent analysis of the relevant texts can be found in \cite[94-107]{definance:etre-et-agir}.} The first one, inspired by Avicenna and found only in his earlier works, most notably  his \emph{De ente et essentia}, argues from the \emph{intellectus essentia}.%
%
\footnote{It can also be found in his commentary on the \emph{Sentences}. While listing ways to demonstrate the existence of God, he makes explicit reference to Avicenna: “ita tamen quod ipsarum rerum naturae non sunt hoc ipsum esse quod habent: alias esse esset de intellectu cujuslibet quidditatis, quod falsum est, cum quidditas cujuslibet rei possit intelligi esse non intelligendo de ea an sit. [\ldots] haec est via Avicennae” \parencite[lib.~2, d.~1, q.~1, a.~1, co.]{st:sent}.}
%
It is evident, he argues, that whatever belongs to something, but is not contained in the thing’s \emph{quidditas}, must be composed with it (for example, whiteness with respect to “cat”), but \emph{esse} is never contained in a \emph{quidditas}. This fact can be proved by the example of the Phoenix, whose quiddity is understandable, but which does not exist.%
%
\footnote{\Cite[cap.~3]{st:deente}: “Quicquid enim non est de intellectu essentiae vel quiditatis, hoc est adveniens extra et faciens compositionem cum essentia, quia nulla essentia sine his, quae sunt partes essentiae, intelligi potest. Omnis autem essentia vel quiditas potest intelligi sine hoc quod aliquid intelligatur de esse suo; possum enim intelligere quid est homo vel Phoenix et tamen ignorare an esse habeat in rerum natura. Ergo patet quod esse est aliud ab essentia vel quiditate, nisi forte sit aliqua res, cuius quiditas sit ipsum suum esse.” For a discussion of the real distinction in \emph{De ente}, see Chapter V of \cite[107–132]{wippel:metaphysical_themes_1}.
For a general overview of \emph{De ente}, see \cite{giorgini:ente}, especially, as regards our topic, 138–146.}
%
This argument, however, is notably absent from later works such as the \emph{Summa contra gentiles} and the \emph{Summa theologiae}, indicating a possible maturation in Thomas’ thought. The \emph{intellectus essentia} argument is in line with the correlation that Thomas finds between the \emph{esse}–\emph{essentia} composition and the two operations of the intellect:
\begin{quotation}
[D]uplex est operatio intellectus. Una, quae dicitur intelligentia indivisibilium, qua cognoscit de unoquoque, quid est. Alia vero, qua composition et dividit, scilicet enuntiationem affirmativam vel negativam formando. Et hae quidem duae operationes duobus, quae sunt in rebus, respondent. Prima quidem operatio respicit ipsam naturam rei. [\ldots]  Secunda vero operatio respicit ipsum esse rei.%
%
\footnote{\Cite[pars~3, q.~5 a.~3 co.]{st:detrinitate}.}
\end{quotation}
%
Perhaps Thomas realized that it can never be more than a probable argument, because, speaking rigorously, the Phoenix (or any other \emph{ens rationis}) is not an essence, but merely a figment. Hence, although it is true that \emph{esse} is not included in the \emph{ratio} of a thing, still there is no such thing as an essence devoid of \emph{esse}.%
%
\footnote{Hence, although the \emph{ratio} of this or that essence does not contain \emph{esse}, the \emph{ratio ipsae essentiae} (the notion of essence itself)—if I may use that term—does.}

More rigorous and “metaphysical” is an argument that starts with the fact that God, who is Pure and Limitless Act, hence utterly simple, cannot have any composition whatsoever. Hence—like the \emph{albedo subsistens}, if there were any—there can only be one \emph{Esse Subsistens} whose \emph{esse} is identical with his \emph{essentia} (because any multiplicity would entail a potency). It follows that any other substance—one really distinct from \emph{Esse Subsistens}—must be composed of a really distinct \emph{esse} and \emph{essentia}:
\begin{quotation}
Ostensum est autem supra,%
%
\footnote{The reference is to \cite[I, q.~3, a.~4]{st:summa}, which asks whether \emph{essentia} and \emph{esse} are the same in God. Thomas argues that anything outside the essence of a thing must come either from itself (as in the case of a \emph{proprium}) or from outside itself. If its \emph{esse} is distinct from its essence, the \emph{esse}, therefore, must come from outside itself. This case cannot apply to God, who is First Cause. The argument is not circular, because it depends on the proofs for the existence of God \parencite[I, q.~2, a.~3]{st:summa}.}
%
cum de divina simplicitate ageretur, quod Deus est ipsum esse per se subsistens. Et iterum ostensum est quod esse subsistens non potest esse nisi unum, sicut si albedo esset subsistens, non posset esse nisi una, cum albedines multiplicentur secundum recipientia. \emph{Relinquitur ergo quod omnia alia a Deo non sint suum esse, sed participant esse}. Necesse est igitur omnia quae diversificantur secundum diversam participationem essendi, ut sint perfectius vel minus perfecte, causari ab uno primo ente, quod perfectissime est. Unde et Plato dixit quod necesse est ante omnem multitudinem ponere unitatem.%
%
\footnote{\Cite[I, q.~44, a.~1, co.]{st:summa} (emphasis added).}
%
\end{quotation}
Thomas argues, it should be noted, that since only God is \emph{esse per essentiam}, it follows that all creatures \emph{participant esse} (possess \emph{esse per participationem}): whatever is the ultimate cause of a perfection must possess it as a \emph{proprium} and \emph{maxime}, and all others receive it by participation.%
%
\footnote{See \cite[I, q.~44, a.~1, co.]{st:summa}{}: “Si enim aliquid invenitur in aliquo per participationem, necesse est quod causetur in ipso ab eo cui essentialiter convenit; sicut ferrum fit ignitum ab igne.” Of course, \emph{esse} is unique among perfections in that \emph{only} the Creator possesses it as a \emph{proprium} and \emph{only} he can communicate it; hence there are no “intermediate” beings that transmit their \emph{esse} to the next rank, as Plotinus posited. “Transcendental” and “perfect” perfections such as goodness and life are similar, but that is because they convertible with and follow from \emph{esse}.} Thomas renders this point more explicit in his \emph{Compendium theologiae}: “Omne quod habet aliquid per participationem, reducitur in id quod habet illud per essentiam, sicut in principium et causam; sicut ferrum ignitum participat igneitatem ab eo quod est ignis per essentiam suam.”\,%
%
\footnote{\Cite[I,  cap.~68]{st:compendium}.} 

The argument just described is, in a way, somewhat unsatisfying for our investigation because in reality it works \emph{in via iudicii}, assuming that we have already worked out the extrinsic causes, and indicating the foundation \emph{propter quid}. It would be desirable, therefore, to have a rigorous demonstration \emph{in via inventionis} (or \emph{quia}) that establishes the fact of the composition from less radical principles. 

Beginning, therefore, at the point where Aristotle left off, we note that substance, or οὐσία—which in its most proper sense is the same as essence in act (τὸ τί ἦν εἶναι)—has its being thanks to an original ontological act (ἐνέργεια) that we can call \emph{esse} (τὸ εἶναι or τὸ ὑπάρχειν)—more precisely \emph{actus essendi} or \emph{esse ut actus}. The \emph{entia} that we encounter in ordinary experience without a doubt are limited, if only because they are multiple and (for the substances we can know immediately by abstraction) material. The very notion of act, however, makes it clear that act, inasmuch as it is act, cannot be limited except by a distinct potency: \emph{Actus non limitatur nisi per potentiam subiectivam realiter distinctam}. Thomas never formulates this axiom explicitly, but a number of texts suggest it (all of which show profound the influence of the Dionysian notion of the “generosity” and “fruitfulness” of the Supreme Cause). Perhaps the best examples can be found in texts in which Thomas demonstrates the infinity of God.%
%
\footnote{For a complete survey of the texts that contain this principle implicitly, see \cite[Chapter V, “Thomas Aquinas and the Axiom that Unreceived Act Is Unlimited”, 123–151]{wippel:metaphysical_themes_2}.
See also \cite[51-56]{definance:etre-et-agir}.
For a contrary opinion, see \cite{robert:principe}.}
%
For instance, in the \emph{Summa contra gentiles} he argues that if there were an \emph{albedo subsistens}, it would be unique and would lack none of the \emph{virtus} of whiteness:
“Actus igitur in nullo existens nullo terminatur: puta, si albedo esset per se existens, perfectio albedinis in ea non terminaretur, quominus haberet quicquid de perfectione albedinis haberi potest.”\,\footcite[I, cap.~43, n.~5 (Marietti n.~360)]{st:contragent}
An act that “exists” in nothing (that is not “received” by something) is not “terminated” (limited) either. In the \emph{Summa theologiae}, he is even more explicit:
“Cum igitur esse divinum non sit esse receptum in aliquo, sed ipse sit suum esse subsistens, ut supra ostensum est; manifestum est quod ipse Deus sit infinitus et perfectus.”\,\footcite[I, q.~7, a.~1, co.]{st:summa}
Unless the \emph{esse} is received in something, says Thomas, it is infinite. This potential principle—which cannot be the matter, because there are substances without it—must be really distinct; we can call it \emph{essentia}, or even \emph{potentia essendi}.

\subsubsection{Real Composition as Participation in \emph{Ipsum Esse}}

Using the Fourth Way for demonstrating the existence of God,\footcite[See][I, q.~2, a.~3, co.]{st:summa} we can go a step further. The \emph{a posteriori} fact that Thomas uses to begin his demonstration is the degrees that are found in reality: “Invenitur enim in rebus aliquid magis et minus bonum, et verum, et nobile, et sic de aliis huiusmodi.”\,\footcite[I, q.~2, a.~3, co.]{st:summa}
Perhaps the most puzzling affirmation is that what is \emph{magis et minus} should always be in reference to a \emph{maxime}: “Sed magis et minus dicuntur de diversis secundum quod appropinquant diversimode ad aliquid quod maxime est.”\,%
%
\footnote{\Cite[I, q.~2, a.~3, co.]{st:summa}. For a discussion of this problem, see \cite{couesnongle:mesure}. De Couesnongle holds that the “measure” of the degrees of \emph{ens} is insufficient as a proof for the existence of God: “Concluons la première partie de ce travail. La \emph{quarta via} parle de connaissance de degrés par référence à un maximum. Cette présentation, croyons-nous, n’annonce pas, dans l’esprit de celui qui a rédigé cette preuve, une interprétation par la théorie de la mesure, celle-ci suffisant pour assurer la conclusion” (p.~75). Instead, he argues, Thomas must be implicitly making recourse to a causal argument: “Pour saint Thomas, le passage des degrés au Maximum, l’affirmation de l’existence de Dieu comme Maximum, se fait à la lumière de la métaphysique de l’être : le maximum est affirmé comme cause de l’existence des degrés” (p. 284). We will see below what seems to me a more convincing way to understand this principle: the \emph{maximum} in a genus is whatever possesses the perfection in question \emph{proprie et per se}. The other members all possess it by participation in the first one, as Thomas’ analogy with the fire (the \emph{maxime calidum}) expresses perfectly.}
%
Clearly heat can be greater or lesser in intensity, but there is no such thing as a “maximum temperature.” It is important to keep in mind that the perfections mentioned—\emph{bonum}, \emph{verum}, and \emph{nobile}—as should be clear by now, are \emph{transcendentia}, and hence coextensive with and consequent to \emph{ens}. Therefore, the variations in goodness, truth, and nobility that we encounter are not limited to “predicamental” differences (differences among the categories of the same substance, or between individuals of the same species) but apply as well to “transcendental” differences (among species). Certain classes of creatures are simply more perfect thanks to the type of creature they are: angels are more perfect than men, who are more perfect than non-rational animals, which are more perfect than plants, which are more perfect than inanimate objects. This differential in perfection turns out to be based on \emph{esse} (based on the type of \emph{resolutio} we did above with \emph{bonum}).

From what we saw above, this \emph{esse} is “received” by a potential principle called \emph{essentia}; hence \emph{esse} can vary in intensity, in accord with the “measure” provided for it by the essence that receives it. \emph{Esse}, or \emph{esse ut actus}, however, is a unique perfection, because it provides not an \emph{actus secundus} that inheres in a substance, but a radical act that makes the substance “be” in absolute terms. The \emph{subiectum} does not produce \emph{esse} (as it would a \emph{proprium}); hence, the \emph{esse} must come from something that is \emph{esse per essentiam}, which—thanks to the inherent unlimitedness of act—possesses that perfection \emph{maxime}. That \emph{esse per essentiam} is, of course, \emph{Ipsum Esse}. It follows from this reflection that the \emph{esse} received by substances at different levels of intensity (\emph{magis et minus}) indicates a participation of the substance in \emph{Ipsum Esse}: he communicates his \emph{esse per essentiam} to his creatures, who possess \emph{esse per participationem}. In other words, \emph{Ipsum Esse} creates by communicating his \emph{esse} and “co-creates” the essence at the same time, so as to provide for it a measure and receptive capacity.

\subsubsection{Intensification of \emph{Esse} and \emph{Essentia} as Act and Potency}
\label{intensification}

The final stage in the \emph{resolutio} is to show that the principles discovered—\emph{actus essendi} and \emph{essentia}—are a true act-and-potency pair, indeed the most radical one possible. It should be clear by now that every perfection found in a substance can be reduced to a type of \emph{esse}: the redness, sweetness, size, figure, and even the substance of an apple are such above all simply because they \emph{are}; all of these are \emph{ens}. To use terminology adopted by C. Fabro, we can characterize this reduction as “formal,” because it stems from the notions of perfection and \emph{ens}.%
%
\footnote{We used this type of reasoning to resolve \emph{bonum} into \emph{esse} above.} The perfections participate in \emph{esse}, certainly, but the participation is “static,” accessible by making an analogy of proportionality; in other words, each perfection has, so to speak, its own parcel of \emph{esse}. However, as we have seen, potency cannot be reduced to act except by something in act. By analogy to the way \emph{esse per participationem} receives its \emph{esse} from \emph{esse per essentiam}, the individual perfections internal to a substance must flow from a single source. This source, the \emph{actus essendi} or \emph{esse ut actus}, is therefore an “original,” radical act. Moreover, being the source of all perfections, it is act with respect to the entire substance, which must be regarded, from this point of view, as a radical \emph{potentia essendi}. The \emph{actus essendi} communicates itself to the perfections, and hence effects a “dynamic” participation. Fabro calls the reduction of individual perfections to the \emph{actus essendi} a “real” reduction, which brings the Aristotelian notions of act and potency to their theoretical limits.%
%
\footnote{For Fabro’s notions of “formal” and “real” reduction, see \cite[108-109]{fabro:problematica}, and \cite[186-187]{fabro:partecipazione}.}

A good source for seeing this “real reduction” is Aquinas’ famous passage from \emph{De potentia} q.~7, in which he shows that he is well aware of his contribution to Metaphysics:

\begin{quotation}
[H]oc quod dico esse est inter omnia perfectissimum: quod ex hoc patet quia actus est semper perfectio[r] potentia. Quaelibet autem forma signata non intelligitur in actu nisi per hoc quod esse ponitur. Nam humanitas vel igneitas potest considerari ut in potentia materiae existens, vel ut in virtute agentis, aut etiam ut in intellectu: sed hoc quod habet esse, efficitur actu existens. Unde patet quod hoc quod dico esse est actualitas omnium actuum, et propter hoc est perfectio omnium perfectionum. Nec intelligendum est, quod ei quod dico esse, aliquid addatur quod sit eo formalius, ipsum determinans, sicut actus potentiam: esse enim quod huiusmodi est, est aliud secundum essentiam ab eo cui additur determinandum.\,%
%
\footnote{\Cite[q.~7, a.~2 ad~9]{st:depotentia}. The edition I used has \emph{semper perfectio potentia}, but it only makes grammatical (and philosophical) sense if \emph{perfectio} is edited to \emph{perfectior}. This is, in any case, the same expression used in q.~1, a.~1, arg.~4: “habitus est perfectior potentia”—which is a special case of the principle in question, applied to operative potencies and their dispositions.}
\end{quotation}
%
\emph{Quaestio} 7 has to do with whether God’s substance or essence is the same as his \emph{esse}, which Aquinas naturally answers in the affirmative. The ninth objection attempts to refute this claim by saying that \emph{esse}, like prime matter, is maximally determinable, and hence very imperfect, because it can be determined by all of its proper predicaments.%
\footnote{In other words, using the language from \emph{De veritate}, \emph{esse}, considered in general, can be “contracted” into any of the ten categories. We will discuss \emph{contractio} in detail in section \ref{sec:diremtion-contractio}.}
%
Therefore, the reasoning goes, \emph{esse} should not be ascribed to God.%
%
\footnote{\Cite[See][q.~7, a.~2 ad~9]{st:depotentia}: “Sed esse est imperfectissimum, sicut prima materia: sicut enim materia prima determinatur per omnes formas, ita esse, cum sit imperfectissimum, determinari habet per omncreatedia propria praedicamenta.” The objection sounds strange (if God has no \emph{esse}, it seems that he would not exist), but Thomas probably has in mind the metaphysical systems of a Platonic stripe that place the priority on the One or the Good, which usually hold that  God is “beyond” \emph{esse}.} Thomas answers that, on the contrary, \emph{esse} (or “hoc quod dico esse”) is the most perfect of all perfections (“inter omnia perfectissimum”). He justifies this reduction using a method similar to Aristotle’s reduction of accident to substance and potency to act (as we saw in section \ref{sec:aristotle-reduction} above), establishing its noetic and ontological priority.%
%
\footnote{Why no chronological priority? Perhaps because he felt that Aristotle dealt with it sufficiently with respect to all act, or more likely because in creation there is never a “temporal” priority as such: the essence is co-created together with the \emph{actus essendi}.}
%
Indeed, Thomas’ reasoning presupposes Aristotle’ reductions, for “patet quia actus est semper perfectior potentia.” \emph{Esse} is noetically prior because “Quaelibet autem forma signata non intelligitur in actu nisi per hoc quod esse ponitur”: any form discovered to be \emph{in actu} presupposes \emph{esse}, for it cannot exist unless it has \emph{esse}. The noetical priority is founded on an ontological priority: “Unde patet quod hoc quod dico esse est actualitas omnium actuum, et propter hoc est perfectio omnium perfectionum.” What Thomas calls \emph{esse} is much more than the simple fact of existence, but the root and source of all perfections. It is not a potential principle that can be determined by an ulterior act, but rather an act that is determined by potency.%
%
\footnote{See \cite[q.~7, a.~2 ad~9]{st:depotentia}: “Unde non sic determinatur esse per aliud sicut potentia per actum, sed magis sicut actus per potentiam.”} This potency, of course, is the limiting, measuring principle that we call essence. If the \emph{actus essendi} is truly an act, and the ultimate act, then it must truly communicate act: as Thomas says, in the context demonstrating that a body can be active,
%
\begin{quotation}
Secundum enim quod participatur aliquid, secundum hoc est necessarium quod participetur id quod est proprium ei, sicut quantum participatur de lumine, tantum participatur de ratione visibilis. Agere autem, quod nihil est aliud quam facere aliquid actu, est per se proprium actus, inquantum est actus, unde et omne agens agit sibi simile.\footcite[I, q.~115, a.~1, co.]{st:summa}
\end{quotation}
%
Things communicate whatever it is proper to them (as fire communicates heat), but proper to any action is act; hence, it must communicate act. (Thus Thomas justifies the maxim \emph{omne agens agit sibi simile}.) This reasoning applies to any act, but especially to the act of acts. We have, then, at last achieved the desired \emph{reductio ad unum} of \emph{ens} to its most formal intrinsic principle: the \emph{actus essendi}, or as we will call it whenever we need to distinguish it from other perfections, \emph{esse ut actus}.%
%
\footnote{For more on this principle, see also \cite[II, cap.~54]{st:contragent}. This is, incidentally, the only place in the entire \emph{Corpus Thomisticum} that contains the phrase \emph{esse ut actus}: “Deinde quia ad ipsam etiam formam comparatur ipsum esse ut actus” (no.~5 [Marietti n.~1291]).}
%

\section{\emph{Compositio} of \emph{Ens}}

\subsection{God as Efficient, Exemplary, and Final Cause of all \emph{Entia}}

With the discovery of \emph{actus essendi} and \emph{essentia} as the original act and potency, the \emph{resolutio} of \emph{ens} to its intrinsic principles is complete. The next stage is to begin to investigate how the \emph{actus essendi} communicates itself and “expands” into the \emph{suppositum}’s various levels of actuality (\emph{esse in actu}). To do this, we begin with God, who is \emph{Ipsum Esse Subsistens}. As can be discovered by applying the method of \emph{resolutio secundum rem} (the analysis of extrinsic causes),%
%
\footnote{Thomas’ method of \emph{resolutio secundum rem} follows a path similar to the one described in this paper for \emph{resolutio secundum rationem}: a phase that follows Aristotle’s analysis of extrinsic causes (chiefly the “efficient” and “final” causes), and then—with the help of the extrinsic “exemplary” cause borrowed from Plato through Neoplatonism—a radicalization of those causes so as to reach the ultimate extrinsic cause, which is \emph{Ipsum Esse Subsistens}. The most concise expression of this \emph{resolutio} is found in the demonstrations of the existence of God, to be found in \cite[I, q.~2, a.~3]{st:summa}, and \cite[I, cap.~13]{st:contragent}. For a good overview of \emph{resolutio secundum rem}, see \cite{mitchell:resolutio-secundum-rem}.}
%
Ipsum Esse is \emph{esse}—which we found in our \emph{resolutio secundum rationem} of \emph{bonum} and \emph{ens} to be the fullness and foundation of act—without any limitation or “measurement” whatsoever, hence pure and infinite act. He possesses \emph{esse} (with the caveat that it is better to say that he \emph{is} his own \emph{esse}) \emph{proprie et per se}. He is, however, the only one who can claim that privilege, because, as we saw, in every other being the \emph{actus essendi} is ontologically prior to and really distinct from the \emph{essentia} that receives it and is co-created with it. When God, who is \emph{esse per essentiam}, creates something, he produces an \emph{actus essendi} in his creature, hence giving it its \emph{esse per participationem}, as we saw: “Creare autem est dare esse rei creatae.”\,\footcite[cap.~1, lc.~5, n.~133]{st:superio} This aspect of the act of creation, which we could call “production,” is evidently an efficient or agent cause. However, it also entails, on the creature’s part, imitation and participation in the divine \emph{Esse}.

In order to create a creature distinct from himself, God co-creates an essence that can receive the act of being, using his own divine essence—which with respect to the creature created is termed a “divine idea”—as the “model.”\,%
%
\footnote{For a discussion of the divine ideas, see \cite[I, q.~15]{st:summa}. Thus, Plato’s doctrine of ideas finds its fulfillment in Thomas. With respect to the creature, God’s essence is a divine idea, but in reality, the idea is identical with God’s essence. “Essentialist” philosophers such as Henry of Ghent tended to see the divine ideas as being quasi-independent even before they received the \emph{respectus} or \emph{modus} of existence. See \cite[7]{stanford:henry-of-ghent}.}
%
In this way, God is the exemplary cause of his creatures in two respects: as the \emph{esse per essentiam} that produces \emph{esse} in his creatures, and as the divine idea that is the model for their essences. As we saw in our \emph{resolutio} of \emph{bonum}, \emph{esse} is at the foundation of perfection, which in turns founds \emph{bonum}. Although not every perfection in a creature causes a corresponding desire in another creature, since God is the exemplar of \emph{every} perfection, because he is \emph{Ipsum Esse}, it follows that God is the also the universal \emph{final} cause for all of his creatures.%
%
\footnote{This reasoning is essentially that of the \emph{quinta via}. See \cite[I, q.~2, a.~3, co.]{st:summa}: “Ea autem quae non habent cognitionem, non tendunt in finem nisi directa ab aliquo cognoscente et intelligente, sicut sagitta a sagittante. Ergo est aliquid intelligens, a quo omnes res naturales ordinantur ad finem, et hoc dicimus Deum.” Evidently, \emph{a fortiori}, those creatures that \emph{are} intelligent also have God as their end.}
%
All creatures seek their perfection and fulfillment in him, to the degree that their essence permits it.%
%
\footnote{It is difficult, therefore, to imagine the creation of a spiritual creature—ofning glory is the exercise of its intellectual power to become \emph{quodammodo omnia}—that does not desire the Beatific Vision of its Creator as its ultimate end. The hypothesis of “pure nature,” therefore, seems difficult to maintain. Our reflection does not deny either the gratuity of grace (or, for that matter, the gratuity of creation) or its necessity for reaching glory. In fact, the doctrine of intensive \emph{esse} makes it clearer: the \emph{actus essendi}, on its own, simply does not have the \emph{virtus} to reach that end. We will take this topic up briefly in section \ref{sec:grace}, below.}
%
God is, therefore, simultaneously the efficient, exemplary, and final cause of all things: \emph{Ipsum Esse}, \emph{Summum Verum}, and \emph{Summum Bonum}.%
%
\footnote{The very structure of the \emph{Summa theologiae}, and even of the \emph{prima pars} reflects the \emph{exitus} from God and the \emph{reditus} of creatures to him. Thomas deals with both the \emph{exitus} and the \emph{reditus} of angles within the \emph{prima pars}, but he leaves man’s \emph{reditus} for the second and third parts, since the \emph{Summa} is intended especially to cover \emph{man’s} creation, redemption, and salvation. Regarding the structure of the \emph{Summa}, see \cite{patfoort:unite}.}
%
We must stress that although there are three kinds of \emph{causality}, they constitute a single act of \emph{creation}. Properly speaking, therefore, only one perfection (the \emph{actus essendi}) is communicated immediately to the creature; that the \emph{esse per participationem} imitates God’s \emph{esse per essentiam} is a consequence of its being composed with \emph{essentia} and being ordered to God’s glory and its own fulfillment.%
%
\footnote{L.B. Geiger argued for the opposite conclusion: “La limitation des formes est première dans son ordre, irréductible. On ne peut espérer en rendre raison par l’appel à une composition avec d’autres éléments, ou à l’inhérence dans quelque sujet, car ces éléments comme ce sujet doivent être eux-mêmes déterminés et limités pour être, et leur limitation demanderait à être expliquée à son tour” \parencite[65]{geiger:participation}. In other words, the participation of the essence in its exemplar is distinct from and prior to the participation of existence in \emph{Ipsum Esse}.

Fabro’s reply can be found in \cite[52–60]{fabro:partecipazione}. See \cite[26–29]{fabro:nozione}, especially 28–29: “Sono persuaso per parte mia che una classificazione metodica dei testi combinata con l’armonia intrinseca alla dottrina del nostro comune Maestro, mette fuori dubbio che ogni partecipazione comporta e similitudine (meglio: similitudine–dissimilitudine) e composizione, altrimenti la similitudine sola porta difilato all’identità e all’immanenza formale, come fecero coerentemente il Platonismo e l’Averroismo.”

See also \cite[469]{fabro:intensive}: “To assert, as has been done (Geiger), that Thomas holds as distinct participation by similitude (\emph{secundum similitudinem}) and participation by composition (\emph{secundum compositionem}), is to break the Thomistic synthesis at its center, which is the assimilation and mutual subordination of the couples of act-potency and \emph{participatum-participans} in the emergence of the new concept of \emph{esse}.” }%

\subsection{The \emph{Diremtion} and \emph{Contractio} of \emph{Esse}}
\label{sec:diremtion-contractio}

We note immediately that when God creates something—even the most perfect of the angels—the creature must in a certain sense be internally divided, because (according to the argument by Thomas presented in section \ref{arguments-real-composition}) no being really distinct from God can exist unless it has a potency (essence) to receive its being.%
%
\footnote{For those creatures that have a material nature, in addition, their substantial form must be received in matter.} This “division” intrinsic to every \emph{ens per participationem}, Fabro calls \emph{Diremtion}, borrowing a term from Hegel.%
%
\footnote{For Fabro’s use of the term, see especially \cite[350]{fabro:partecipazione}: “Ma l’esse partecipato è «caduto» nella \emph{Diremtion} della differenza ontologica e quindi non è più sufficiente in se stesso: se la forma delle cose materiali abbisogna della materia come soggetto, altrettanto — anzi di più — l’\emph{esse} ha bisogno della forma ovvero dell’atto formale come sua potenza. Infatti con la \emph{Diremtion} che fa cadere l’\emph{esse} dalla sua semplice identità nella differenza ontologica, con l’intervallo del nulla (creazione), l’esse diventa partecipato e quindi commensurato e attribuito a «qualcosa».”

Fabro uses the term throughout this work. The term, a German word derived ultimately from the Latin \emph{dirimo}, etymologically means “separation,” but can be used broadly as well to mean “determination.” Hegel uses it to refer to the Spirit’s “self-denial” that is necessary for its development towards the Absolute Spirit. \emph{Diremtion} could similarly be applied to the “outpouring” of the One proposed by Neoplatonist philosophers that results in its “division” among the participants. Thomas, of course, attributes neither “becoming” to the Creator nor necessity to the act of creating, but we can consider the communication of \emph{esse} as a sort of “falling” from God that produces a “division” from him and within the creature.
For a good overview of how Fabro uses the term, see \cite[191–196]{mitchell:being}.}
%
The concept is well described by the following passage already quoted above: 
\begin{quotation}
Et iterum ostensum est quod esse subsistens non potest esse nisi unum, sicut si albedo esset subsistens, non posset esse nisi una, cum albedines multiplicentur secundum recipientia. Relinquitur ergo quod omnia alia a Deo non sint suum esse, sed participant esse.\footcite[I, q.~44, a.~1, co.]{st:summa}
\end{quotation}
Only when \emph{esse} is “split” does it admit of different degrees of intensity, and can it be communicated to manifold subjects. In reality, \emph{Diremtion} was discovered (or first exposed) by Plato in the \emph{Sophist}: as we saw, in order for the meta-idea of τὸ ὄν to be communicated among many participants, it must be \emph{different from} the ideas that participate in it, and hence the \emph{participans} must participate in both τὸ ὄν and τὸ ἕτερον. In other words, participation requires an internal division in the \emph{participans}. \emph{Diremtion} is the basis, therefore, for what Thomas calls \emph{esse commune}: created \emph{esse} considered in general (or as Thomas puts it, \emph{sine additione})%
%
\footnote{See \cite[I, cap.~26, n.~11 (Marietti n.~247)]{st:contragent}: “id quod commune est vel universale [i.e., esse commune] sine additione esse non potest, sed sine additione consideratur.” Thomas is refuting the idea that God is the formal cause of all things, and hence identical to \emph{esse commune}. \emph{Esse commune}, cannot, of course, subsist (\emph{sine additione esse}), but it can be considered without its \emph{additiones}.} and as common to all \emph{entia} in different degrees.%
%
\footnote{See \cite[379]{fabro:partecipazione}: “La «Diremtion» dell’essere si compie pertanto nel primo momento della costituzione del reale. È vero che il termine proprio della creazione è l’\emph{esse}, ch’è perciò l’effetto proprio di Dio, ma si tratta dell’\emph{esse commune}; perchè \emph{l’esse per essentiam} è Dio stesso ch’è impartecipabile.”} The \emph{Diremtion} does not entail an “emanation” of any sort (neither Neoplatonist, nor Spinozan, nor Hegelian)—God does not “lose” any of his \emph{esse}, nor is he in any way obliged to act—but rather it results from the act of creation: by creating an essence together with its \emph{actus essendi}, God provides the “space” or receptive capacity necessary for that creature to subsist and at the same time gives the creature its \emph{esse}.%
%
\footnote{See \cite[366]{fabro:partecipazione}: “Si deve ammettere che alla prima origine delle cose, è Dio stesso che fa la prima «Diremtion» dell’esse partecipato nei suoi princìpi e che procede da Dio non solo l’esse ma anche l’essenza e la forma che lo riceve.”} 
%

Since every \emph{ens} results from the first \emph{Diremtion} of \emph{esse} effected by God (which we can call the “transcendental” \emph{Diremtion}), it is possible to consider its \emph{esse} and \emph{essentia} as they are “before” that \emph{ens} is constituted.%
%
\footnote{Naturally, the act of creation transcends time, and so the use of such temporal terms is strictly analogical.}
%
From this point of view, \emph{esse} and \emph{essentia} are a true act-and-potency pair, such that the \emph{essentia} is in no way \emph{in actu} except through the \emph{esse} that actuates it:%
%
\footnote{See \cite[40-41]{fabro:dallessere}: “Per S.~Tommaso (a differenza di tutta la tradizione patristica e scolastica, prima e dopo di lui) l’essenza va detta potenza e in potenza rispetto all’\emph{esse partecipatum} ch’è l’atto primo metafisico, derivato da Dio, ch’è l’\emph{esse per essentiam}.” In his essay “La verità dell’essere e l’inizio del pensiero” \parencite[11–69]{fabro:dallessere}, Fabro makes a fascinating comparison between Hegel’s dialectic of \emph{Sein} and Thomas Aquinas’ dialectic of \emph{esse}.}
%
in this sense, we may, following Fabro, term the active principle \emph{esse ut actus} and the passive principle, \emph{potentia essendi}.%
%
\footnote{Thomas does not use the expression \emph{potentia essendi} often, but it grasps the intended concept well. The only time it is used in the sense proposed in this paper is in \cite[VIII, lc.~21, n.~13]{st:phys}: “In omni ergo substantia quantumcumque simplici, post primam substantiam simplicem, est potentia essendi. [\ldots] Non potest ergo evadere inconveniens per hoc quod dicit quod in corpore caelesti non est potentia essendi: hoc enim est manifeste falsum, et contra intentionem Aristotelis.” See also \cite[30]{contat:esse-essentia-ordo}.}
%
As we will see in greater detail in \autoref{chap:dynamism}, however, there is also what could be called a “predicamental” \emph{Diremtion} in which the \emph{esse ut actus}, which is always unique, becomes “divided” into various levels of \emph{esse in actu} (what could be termed the “fact” of being): that of the substance itself, that of the inherent accidents inherent, and that of \emph{operari}.%
%
\footnote{The distinction between \emph{esse ut actus} and \emph{esse in actu} is dealt with in \cite[60–68]{fabro:partecipazione}, and throughout the work; also in \cite[117-125]{fabro:problematica}.}
%
Therefore, when the \emph{ens} is considered “after” its constitution (or better said, when one investigates “within” the \emph{ens} itself) then the essence is \emph{in actu}: it is the οὐσία, or τὸ τί ἦν εἶναι, discovered and described by Aristotle.

Although Thomas does not always use the terms coined by Fabro—\emph{esse ut actus} and \emph{esse in actu}—it is Thomas’ use of the term \emph{esse} that suggests this distinction. Although, as we saw above, Thomas sometimes refers to \emph{esse} as \emph{actualitas omnium actuum} (which corresponds to our \emph{esse ut actus}), he sometimes uses \emph{esse} in expressions such as \emph{esse substantiale} and \emph{esse accidentale}, which must refer to the “divided” \emph{esse} that is found “within” \emph{ens}.%
%
\footnote{For mature works with this terminology, see for example \cite[I, q.~76, a.~4, s.c.:]{st:summa}: “unius rei est unum esse substantiale. Sed forma substantialis dat esse substantiale;”  and \cite[I, q.~28, a.~2, co.:]{st:summa}: “Si vero consideretur relatio secundum quod est accidens, sic est inhaerens subiecto, et habens esse accidentale in ipso.” The former quotation is one of many formulations of the maxim \emph{forma dat esse}. See also \cite[199]{fabro:partecipazione}: “Una conferma ed un’applicazione dell’\emph{esse essentiae} (l’essenza metafisica), è la divisione dell’\emph{esse} in \emph{esse substantiale} ed \emph{esse accidentale} che non può riguardare direttamente l’esse come \emph{actus essendi}, il quale è l’atto proprio della sostanza completa (\emph{substantia prima}).”}
%
If one assumes that Thomas makes a quasi-univocal of the term \emph{esse}, then the temptation is to conclude that each “level” of \emph{esse}—substance, accident, and even \emph{operari}—has its own, practically independent \emph{actus essendi}.%
%
\footnote{Cajetan and his school came to this very conclusion. I say “quasi”-univocal, because Cajetan effectively admits an analogy of proportionality in \emph{esse}. See \cite[III, n.~29]{cajetan:denominum}: “Scimus quidem secundum hanc analogiam rerum intrinsecas entitates, bonitates, veritates etc., quod ex priori analogia non scitur. Unde sine huius analogiae notitia, processus metaphysicales absque arte dicuntur.”}
%
It seems to me that the \emph{resolutio} that we did in section \ref{intensification}, especially our analysis of the passage from \emph{De potentia}, q.~7, shows clearly that Aquinas in fact made an \emph{analogical} use of the term \emph{esse}: an analogy of reference (or intrinsic attribution), with reference to the single original act that is the \emph{actus essendi} or \emph{esse ut actus}.

A notion related to \emph{Diremtion} and actually found in Thomas’ \emph{corpus} is that of \emph{contractio}. As should by now be clear, the only way for \emph{esse} to be  “divided” is for it to be limited by a potency (that is, essence); created \emph{esse} is, therefore, “smaller” than \emph{Ipsum Esse}. Likewise, the instances of \emph{esse in actu} posterior to \emph{esse ut actus} are “smaller” than their original, because \emph{esse in actu} is not identical to \emph{esse ut actus} but only participates in it. The first important passage that talks about \emph{contractio} is the one that discusses \emph{additio} in \emph{De veritate}, q.~21, a.~1:

\begin{quotation}
Alio modo dicitur aliquid addere super alterum per modum \emph{contrahendi} et determinandi; sicut homo addit aliquid super animal: non quidem ita quod sit in homine aliqua res quae sit penitus extra essentiam animalis, alias oporteret dicere, quod non totum quod est homo esset animal, sed animal esset pars hominis; sed animal per hominem \emph{contrahitur}, quia id quod determinate et actualiter continetur in ratione hominis, implicite et quasi potentialiter continetur in ratione animalis.%
%
\footnote{\cite[q.~21, a.~1, co.]{st:deveritate} (emphasis added).}
\end{quotation}
%
In our \emph{resolutio} of \emph{bonum}, we saw that \emph{additiones} can be \emph{reales} or \emph{rationis}, and we saw that the \emph{ratio boni} adds something \emph{sine contractione}. However, an \emph{additio rationis} can also entail a \emph{contractio}, in the sense of specifying concept’s scope or extension. For example, adding \emph{rational} to \emph{animal} results in \emph{man}, which is reduced in scope with respect to \emph{animal}. As can be seen, \emph{contractio} in this context is primarily a noetical notion. However, the passage from \emph{De veritate} can be compared to one in Thomas’ commentary on the \emph{Metaphysics}, in which he comments the very passage we saw above Book Δ, 7:

\begin{quotation}
Sciendum est enim quod ens non potest hoc modo \emph{contrahi} ad aliquid determinatum, sicut genus \emph{contrahitur} ad species per differentias. [\ldots] Unde oportet, quod ens \emph{contrahatur} ad diversa genera secundum diversum modum praedicandi, qui consequitur diversum modum essendi; quia quoties ens dicitur, idest quot modis aliquid praedicatur, toties esse significatur, idest tot modis significatur aliquid esse.%
%
\footnote{\Cite[V, l.~9, n.~5–6 (Marietti n.~889–890)]{st:metaph} (emphasis added).}
%
\end{quotation}
%
Even though \emph{ens} is not strictly speaking a genus, it is still possible to “contract” it according to the different ways that it can be predicated (namely, the “figures” or  “schemata” of predication; that is, the categories). In as many ways as \emph{ens} is predicated, argues Thomas, in just so many ways it signifies \emph{esse}. To put it another way, whereas the \emph{contractio} of a genus into a species, or of a larger genus into a smaller one, does not imply a \emph{real} diminution of \emph{esse} (because, for example, “animal” and “cat” both signify substances), the \emph{contractio} of \emph{ens} does, because different \emph{modi} possess different degrees or “measures” of \emph{esse}.%
%
\footnote{In fact, Aquinas defines \emph{modus} as “quem mensura praefigit.” Therefore, he says, “unde importat quandam determinationem secundum aliquam mensuram” \parencite[I-II, q.~49, a.~2, co.]{st:summa}. He is, in this context, describing the category of quality, so as to explain what a first-species quality (\emph{habitus} or \emph{dispositio}) is. Here, he describes quality as a “mode of a substance,” but by analogy, the categories could be called “modes of \emph{ens}.”}
%
Therefore, the modes of predication of \emph{ens} indicate ontological “degrees” of possession of \emph{esse}. It should be noted that \emph{contractio} applies to all of the categories without exception, including substance: a little later in the passage, Thomas says, “Quia igitur eorum quae praedicantur, quaedam significant \emph{quid}, idest \emph{substantiam}, quaedam quale, quaedam quantum, et sic de aliis.”\,%
%
\footnote{\Cite[V, lib.~5, l.~9 n.~6 (Marietti n.~890)]{st:metaph} (emphasis added).}
%

Coupling this reflection on \emph{contractio} with Aquinas’ notion of \emph{esse commune} yields a fascinating result: the \emph{esse} that is “contracted” in the various predications of \emph{ens} must be \emph{esse commune}—that is, \emph{esse} considered “prior” to its \emph{additiones} (which never, however, subsists without these \emph{additiones}). However, \emph{esse commune} can be considered, as it were, independently of the \emph{modus entis} (category) that it belongs to, and even independently of the essence that determines it. In a substance, therefore, the \emph{contractio} is twofold: first, \emph{esse commune} has a (conceptually) limitless application to essence and is therefore “contracted” to a particular “level” or “measure” when it is considered together with this or that essence.%
%
\footnote{Hence, when understood correctly, the term \emph{modus essendi} is a most apt appellative for essence.}
%
Second, \emph{esse} considered as “within” a substance or “after” its constitution is “contracted” when applied to one of its categories. The first \emph{contractio} we could term “transcendental,” since it occurs “prior” to the division of \emph{ens} into categories; the second, “predicamental” or “categorial.” It is telling that for Thomas even substance, which Aristotle considers \emph{ens} in its fullness, entails a \emph{contractio} in the second sense.%
%
\footnote{Naturally, the categories other than substance possess progressively less \emph{esse}.}
%
We can deduce from this reflection that the “categorial” \emph{esse} (\emph{esse in actu}) of the various \emph{modi entis} flows from and participates in an ontological ἐνέργεια that is “prior” to all of the categories (that is, transcendental) and contains them virtually: none other than \emph{esse ut actus}. The essence mediates between the two levels by receiving the \emph{esse ut actus} and then communicating its own \emph{esse in actu} to all of the other accidents, and ultimately to its \emph{operari}.

\subsection{\emph{Esse ut Actus} and \emph{Esse in Actu}}
A confirmation of this reflection can be found in Thomas’ treatise on separated substances, where he distinguishes between form and subject, even in angels:%
%
\footnote{Regarding the distinction between form and subject, see \cite[54]{contat:esse-essentia-ordo}.}
%
\begin{quotation}
ratio formae opponitur rationi subiecti. Nam omnis forma, in quantum huiusmodi, est actus; omne autem subiectum comparatur ad id cuius est subiectum, ut potentia ad actum. Si quae ergo forma est quae sit actus tantum, ut divina essentia, illa nullo modo potest esse subiectum; et de hac Boetius loquitur. Si autem aliqua forma sit quae secundum aliquid sit in actu, et secundum aliquid in potentia; secundum hoc tantum erit subiectum, secundum quod est in potentia. Substantiae autem spirituales, licet sint formae subsistentes, sunt tamen in potentia, in quantum habent esse finitum et limitatum.\footcite[a.~1, ad~1]{st:spiritualibus}
\end{quotation}
%
A subject, Thomas argues, can only be a subject inasmuch as it is in potency. (It is something that is reduced to act). Even a pure spirit (other than God) can be a subject, because it is limited and in that sense \emph{in potentia}. In fact, in can only be a subject inasmuch as it is \emph{in potentia}. Clearly, a pure form is \emph{in actu}. How be both in act and in potency at the same time? The answer lies in the fact that form is not the \emph{ultimate} act of a substance—not even a spiritual one—but in fact receives its actuality from a superior source: the \emph{esse ut actus}. The form, we might say is a type of \emph{esse}: it is what Thomas calls \emph{esse substantiale}, or what we have called a substance’s \emph{esse in actu}. This \emph{esse in actu}, however, is not identical to the \emph{esse ut actus} but rather is the result of a \emph{contractio}.

\subsection{Conclusions from the \emph{Compositio}}

Summing up the results of our \emph{compositio} thus far, we see that \emph{ens} proceeds from God, not, of course, in a necessary way, but as “production;” that God is the efficient, exemplary, and final cause of all of his creatures, and that these creatures proceed by way of a \emph{Diremtion} (or “division”) of \emph{esse} that entails a \emph{contractio}. This \emph{esse}, when it is considered \emph{sine additione}, is what Thomas calls \emph{esse commune}. In fact, \emph{esse} always subsists \emph{cum additione}, both \emph{sine contractione} (the \emph{transcendentia}) and \emph{cum contractione} (the \emph{modus specialis entis} or \emph{ens} as divided into categories). The \emph{contractio entis} (and likewise the \emph{Diremtion}) is both “transcendental” (“prior” to the constitution of \emph{ens} and its division into categories) and “predicamental” (“after” the constitution of \emph{ens} and “within” the categories). The \emph{actus essendi} or \emph{esse ut actus}, therefore acts as a mediator between \emph{Ipsum Esse} and the substance in act, with its various actuations. The essence, in turn acts as a mediator between the \emph{actus essendi} and the various \emph{esse in actu} of the substance. On the “transcendental” level, we might say, it gathers behind it all of the “power” (δύναμις, \emph{virtus}) that God grants to the substance, measuring and determining it. On the “predicamental” level, it communicates the \emph{esse} that it receives from the \emph{actus essendi}; and since it is always the form that contains the actuality of the essence (whether it is a pure form or a compound of matter and form), we may say with Thomas that \emph{forma dat esse}.%
%
\footnote{See \cite[I, q.~76, a.~4, co.]{st:summa}: “Forma autem substantialis dat esse simpliciter.” Naturally, the substantial form is opposed to the accidental form, which gives \emph{esse secundum quid}.}
%

We should stress that strictly speaking it is not the \emph{actus essendi} that “is,” nor is the \emph{actus essendi} strictly \emph{in actu}. Rather, it is the principle \emph{by which} \emph{ens} is (\emph{quo est}); it is the \emph{ens} that “is,” and the substance and its various actuations that are \emph{in actu}. In fact, until the \emph{actus essendi} is allowed to expand fully, it is \emph{deprived} of being in act (\emph{esse in actu}); it “wants” to expand, so to speak. The inner workings of \emph{ens}, therefore, seem to form a microcosm of divine causality: God is efficient, exemplary, and final cause of \emph{ens}; it seems that \emph{actus essendi} functions as the efficient, exemplary, and final cause of the \emph{suppositum}. Finally, viewing the predication of \emph{ens} as a double \emph{contractio} helps us to see that the “quantity” of \emph{esse} that a substance has, as it were, two “dimensions”: “vertical” or transcendental, and “horizontal” or predicamental. The very \emph{actus essendi} is a \emph{contractio} of \emph{esse commune}; therefore, \emph{actus essendi} can be found in various degrees of intensity or “virtuality.” Predicamental \emph{esse} (\emph{esse in actu}), on the other hand is, as it were, “additive”: accidents and operation are \emph{superadditum}, as if they constituted an “extensive” quantity. How \emph{actus essendi} functions as a mediator, and how these various levels interact will be a topic for the next chapter. Table \ref{tab:esse-ut-actus-in-actu} summarizes our findings so far.

\begin{table}
  \centering
    \begin{tabular}{lll}
      \toprule
        \rule[-6pt]{0pt}{22pt}&
        \textbf{Transcendental Level} &
        \textbf{Predicamental Level} \\
      \midrule
        \rule[-6pt]{0pt}{22pt}How \emph{ens} is considered &
        \emph{Ens constituendum} &
        \emph{Ens constitutum} \\
        %
        \rule[-6pt]{0pt}{22pt}\emph{Esse} &
        \emph{Esse ut actus} &
        \emph{Esse in actu} \\
        %
        \rule{0pt}{16pt}\emph{Essentia} &
        \emph{Essentia ut potentia essendi} &
        \emph{Essentia in actu} \\
        %
        \rule[-6pt]{0pt}{6pt}&
        &
        (\emph{actus formalis}) \\

      \bottomrule
    \end{tabular}%
    \caption{\emph{Esse ut actus} and \emph{esse in actu}}
  \label{tab:esse-ut-actus-in-actu}%
\end{table}%

\begin{DONE}

Now that we know the principles, \enquote{reassemble} \emph{ens} from the top down. (Follow closely \emph{esse, essentia, ordo}.)

Distinguish \emph{ens constituendum} and \emph{ens constitutum}.

\emph{Esse ut actus} vs. \emph{esse in actu} and \emph{diremptio} and all that; the threefold \enquote{expansion} into various \emph{esse in actu}.

\end{DONE}

\section{\emph{Actus Essendi} as \emph{Virtus Essendi} }
\label{sec:virtus-essendi}

Saint Thomas’ conception of \emph{actus essendi} (at least the interpretation given by Fabro and those with a similar vision, and assumed by this paper) is unique in affirming three characteristics of \emph{esse}: first, a \emph{suppositum} possesses its \emph{actus essendi} as its “own;” \emph{esse ut actus} is truly an intrinsic principle, not merely a shadow of the divine \emph{Esse}. Second, \emph{esse ut actus} is the source of all the actuality in that \emph{suppositum}. Finally, each species possess it according to a different degree of intensity, thanks to the receptive capacity and measure that the essence provides: we can, therefore, speak of \emph{esse} as if it were a \emph{quantitas virtutis} or \emph{virtualis}.%
%
\footnote{We must be careful, however, to affirm that once a \emph{suppositum} is constituted, for as long as it endures, neither its essence nor its \emph{actus essendi} change. At most, a non-spiritual creature can undergo corruption, but its own \emph{actus essendi} can neither increase nor diminish. This is true for the simple reason that it is God himself, \emph{Ipsum Esse}, who provides the \emph{actus essendi}: it depends on no other principle. See \cite[I, cap.~20, n.~27 (Marietti n.~179)]{st:contragent}: “Esse est aliquid fixum et quietum in ente;” also, \cite[37]{contat:esse-essentia-ordo}.} The other systems that we saw—the metaphysics of essence (Henry of Ghent, Scotus, Suárez), classical Thomism (Cajetan and his followers), and even transcendental Thomism—all agree in attributing to \emph{esse} a nearly univocal status. For the metaphysics of essence, this stance is quite clear, for \emph{esse} (\emph{existentia}) is the mere passage of an essence already constituted to an independent status (\emph{modus} or \emph{respectus}) outside its causes. For classical Thomism, although many followers of this school (notably Domingo Báñez and Jacques Maritain) recognized the primacy of \emph{esse} over \emph{essentia}, their concept concept of \emph{esse} remains fundamentally that of \emph{existentia}: \emph{positio extra causas}.%
%
\footnote{Whereas Cajetan held that \emph{existentia} is an \emph{actus secundus}, Báñez upheld its status as \emph{actus primus}, a position upheld by Maritain. See \cite[612–613]{fabro:partecipazione}, and \cite[109-111]{contat:figure}. Cajtan, in fact, felt obliged to propose \emph{subsistentia} as a third principle in \emph{ens}, after \emph{existentia} and \emph{essentia}, to “glue together” the various acts found in a \emph{suppositum}. See \cite[617]{fabro:partecipazione}.} Even with transcendental Thomism, \emph{esse} is thought of principally in terms of its function as the \emph{copula} that is in reference to an unlimited “horizon.” Although they effectively consider \emph{esse} to be a “virtuality” or “power,” it seems to be \emph{extrinsic} to the \emph{suppositum}. It is not possessed or “received” by essence as its own, and hence does not truly admit of degrees in intensity.%
%
\footnote{The fact that Lotz makes such a unique interpretation of the \emph{quarta via} is indicative of this conception of \emph{esse}. In brief, Lotz says in that every judgment, man makes what he calls a “small analogy,” by which he discovers that \emph{ens} is immersed in a limitless “horizon,” which we can call \emph{esse}. In order to avoid making this horizon temporal and immanent (as in Heidegger), man has recourse to the “great analogy,” in which he sees that \emph{ens} exists thanks to \emph{Ipsum Esse}—in other words, that this “horizon” is a real, consistent Being. The reasoning is very similar to Blondel’s regarding the \emph{antiboulie}. See \cite[228–229]{contat:confronto}.}
%
A brief discussion, therefore, of how Thomas views \emph{actus essendi} as a \emph{virtus essendi}—a sort of \emph{quantitas virtualis}—is in order.

\subsection{Intensive and Extensive Quantity}
\label{sec:intensive-extensive}

Aristotle is the first philosopher to discuss quantity (τὸ ποσόν) at length: it is normally the first category to be listed, after substance,%
%
\footnote{See, for example, the two complete lists of categories: \cite[I, 1b25-2a4]{aristotle:categories}: Τῶν κατὰ μηδεμίαν συμπλοκὴν λεγομένων ἕκαστον ἤτοι οὐσίαν σημαινει ἢ ποσὸν ἢ ποιὸν ἢ πρὸς τι ἢ ποῦ ἢ ποτὲ ἢ κεῖσθαι ἢ ἔχειν ἢ ποιεῖν ἢ πάσχειν., “Each of the words in no way combined truly signifies substance or ‘how much’ or ‘which’ or ‘toward which’ or ‘where’ or ‘when’ or ‘lying down’ or ‘having’ or ‘making’ or ‘undergoing;’ and \cite[Α, 9, 103b23–24]{aristotle:topics}: ἐστί δὲ ταῦτα τὸν ἀριθμόν δέκα, τί εστι, ποσὸν, ποιὸν, πρὸς τι, ποῦ, ποτὲ, κεῖσθαι, ἔχειν, ποιεῖν, πάσχειν, “These, however, are ten in number: ‘what is it,’ ‘how much,’ ‘which,’ ‘toward which,’ ‘where,’ ‘when,’ ‘lying down,’ ‘having,’ ‘making,’ ‘undergoing” (my translation in both cases).}
%
and strictly speaking, of course, it is an accident found only in material \emph{entia}: either three-dimensional, continuous extension, or else an enumeration of discrete units.%
%
\footnote{For Aristotle’s treatment of quantity, see especially \cite[Ζ, 4b20-2a35]{aristotle:categories}.}
%
Quantity is, therefore, “additive” and “extensive” by nature: two lines can be joined together so as to make a longer line, and two sets of books can be added to obtain a larger set of books. By analogy, however, other realities can be “quantified,” most often the qualities, but also any other perfection or measurable reality. Thomas explains:
\begin{quotation}
Est autem duplex quantitas: scilicet dimensiva, quae secundum extensionem consideratur; et virtualis, quae attenditur secundum intensionem: virtus enim rei est ipsius perfectio, secundum illud philosophi in VIII \emph{Physic.}: unumquodque perfectum est quando attingit propriae virtuti. Et sic quantitas virtualis uniuscuiusque formae attenditur secundum modum suae perfectionis. Utraque autem quantitas per multa diversificatur: nam sub quantitate dimensiva continetur longitudo, latitudo, et profundum, et numerus in potentia. Quantitas autem virtualis in tot distinguitur, quot sunt naturae vel formae; quarum perfectionis modus totam mensuram quantitatis facit.%
%
\footnote{\Cite[29, 3]{st:deveritate}. Interestingly, this reflection comes in the context of whether Jesus Christ possesses created grace or not. Therefore, it is abundantly clear that \emph{quantitas virtutis} or \emph{virtualis} applies analogically to even to purely spiritual realities.}
\end{quotation}
There are two types of quantity: dimensive or extensive, and intensive or virtual.%
%
\footnote{Regarding the two types of quantity, see \cite[53]{villagrasa:resolutio}.}
%
As an example to illustrate the difference, we will use one suggested by Thomas’ \emph{maxime calidum}, but taken from modern physics: the difference between heat and temperature. Heat can be described as the \emph{total} amount of energy (in the modern sense) in a body, whereas temperature is the \emph{average} or \emph{per-unit} amount of energy.%
%
\footnote{According to most modern models, the quality that we experience as heat results from the internal movements of the molecules and atoms that constitute a substance. This movement is not strictly mechanical, but involves complex interactions among the particles and their chemical bonds.} A swimming pool, for example, possesses much more heat than a lighted candle, but its temperature is much less.%
%
\footnote{A simple confirmation of this fact is that one candle would be unable to raise the temperature of the water appreciably, even if all of the heat it produces could be transferred to the water without loss.} Even scientists distinguish—in much the same way as Saint Thomas does in the passage above—between \emph{extensive} qualities (such as heat) and \emph{intensive} ones (such as temperature): add more water to the swimming pool, and more heat is added, even if the water added is colder. Increasing the water’s temperature is another matter; it would require a source of energy (a \emph{virtus} or δύναμις). The example is strictly material, but it does illustrate why Thomas affirms that \emph{quantitas virtualis} is considered \emph{secundum intensionem}: unless a thing already possesses the \emph{virtus} in question, it must receive that \emph{virtus} from an outside agent that can reduce it from potency to act. Intensity is the measure of \emph{virtus}, which in reality is the same as perfection.

\subsection{\emph{Esse} as a \emph{Quantitas Virtualis}}

The question now is whether Thomas applies the notion of intensive quantity to \emph{esse}: it seems to me that the affirmative can be abundantly demonstrated.%
%
\footnote{For an extensive discussion of this very topic, see especially \cite{orourke:virtus}.} Aquinas, as we saw, is indebted to Dionysius for many of his ideas about \emph{esse} (allowing for the fact that Dionysius affirmed the primacy of the Good). The concept borrowed from Dionysius is that of δύναμις, which in this case takes on the connotation of “power” rather than “potency.” For example, there is the passage, “πάσης δυνάμεως αἴτιος καὶ πάντα κατὰ δύναμιν ἄκλιτον καὶ ἀπεριόριστον παράγων καὶ ὡς αὐτοῦ τοῦ εἶναι δύναμιν,”\,%
%
\footnote{\Cite[8, 2]{pg:dionysius:DN}: “[He is] the cause of all power (πάσης δυνάμεως) and, producing all things according to uninclined and uncircumscribed power (δύναμιν) and as the power of being itself (αὐτοῦ τοῦ εἶναι δύναμιν)” (my translation). }
%
from which Thomas seems to have derived the term \emph{virtus essendi} (αὐτοῦ τοῦ εἶναι δύναμις).%
%
\footnote{See \cite[1]{gilson:virtus}. This article discusses at length the origin of the expression in Saint Thomas.}
%
We saw above that τὸ εἶναι is the first and greatest of God’s gifts. However, it receives its “power to be” from what is beyond being: “τὸ εἶναι δύναμιν εἰς τὸ εἶναι ἔχει παρὰ τῆς ὑπερ\-ουσίου δυνάμεως;”\,%
%
\footnote{\Cite[8, 3]{pg:dionysius:DN} “\emph{Esse} has power unto \emph{esse} [i.e., power to be] from the super-being power [the power that is beyond \emph{esse}]” (my translation).} and “ὁ ὢν ὅλου τοῦ εἶναι κατὰ δύναμιν ὑπερ\-ούσιός ἐστιν ὑποστάτις αἰτία καὶ δημιουργὸς ὄντος”\,%
%
\footnote{\Cite[5, 4]{pg:dionysius:DN}: “The one who is, is the substantial cause of all \emph{esse} is the creator (δημιουργὸς) of \emph{ens}” (my translation).} Commenting on this passage, Thomas affirms,
%
\begin{quotation}
ostendit quod omnia conveniunt Deo, quodammodo. Ad cuius evidentiam considerandum est quod omnis forma, recepta in aliquo, limitatur et finitur secundum capacitatem recipientis; unde, hoc corpus album non habet totam albedinem secundum totum posse albedinis. Sed si esset albedo separata, nihil deesset ei quod ad virtutem albedinis pertineret. Omnia autem alia, sicut superius dictum est, habent esse receptum et participatum et ideo non habent esse \emph{secundum totam virtutem essendi}, sed solus Deus, qui est ipsum esse subsistens, \emph{secundum totam virtutem essendi}, \emph{esse habet}.%
%
\footnote{\Cite[V, lc.~1, 629]{st:divnomin} (emphasis added). Evidently, Thomas is using a translation in which δύναμις is translated \emph{virtus}.}
\end{quotation}
%
Thomas argues—as we have seen in other passages as well—that when a perfection is received in a subject, it is limited and determined by the recipient; only a “separate” perfection could have all of the \emph{virtus} or intensity  it is capable of. \emph{Esse}, in fact, is a perfection (albeit a special one) and works in the same way: all creatures (\emph{omnia alia}, those that are not God) participate in \emph{esse} and receive it. Hence, they do not possess \emph{esse} \emph{secundum totam virtutem essendi}. Rather, they receive it from the one who does, namely God. Evidently, Thomas is referring to \emph{esse ut actus}, because he speaks of it as being received \emph{secundum capacitatem recipientis}—that is, into a potency, as whiteness is received into the subject. Perhaps even more clear is Thomas’ affirmation, from the \emph{Summa Contra Gentes}, justifying God’s infinite perfection: “Igitur si aliquid est cui competit \emph{tota virtus essendi}, ei nulla nobilitatum deesse potest quae alicui rei conveniat.”\,%
%
\footnote{\Cite[I, cap.~28, n.~2 (Marietti n.~260)]{st:contragent}, (emphasis added).} If there is something to which corresponds the whole “power to be” (\emph{virtus essendi})—that is, \emph{esse per essentiam} or \emph{Esse Subsistens}—it cannot be lacking in nobility.

Aquinas applies this very reasoning to his description of \emph{habitus} in the \emph{prima secundae}. He says, in \emph{quaestio} 52, which regards the possibility of an increase in an \emph{habitus}:
%
\begin{quotation}
Sic igitur patet quod, cum habitus et dispositiones dicantur secundum ordinem ad aliquid, ut dicitur in VII \emph{Physic.}, dupliciter potest intensio et remissio in habitibus et dispositionibus considerari. Uno modo, secundum se, prout dicitur maior vel minor sanitas; vel maior vel minor scientia, quae ad plura vel pauciora se extendit. Alio modo, secundum participationem subiecti, prout scilicet aequalis scientia vel sanitas magis recipitur in uno quam in alio, secundum diversam aptitudinem vel ex natura vel ex consuetudine.\footcite[I-II, q.~53, a.~1, co.]{st:summa}
\end{quotation}
%
The passage from Aristotle’s \emph{Physics} that Thomas refers to is “\,Ἔτι δὲ καί φαμεν ἁπάσας εἶναι τὰς ἀρετὰς ἐν τῷ πρός τι πὼς ἔχειν,”\,%
%
\footnote{\Cite[Η, 3, 246b 3-4]{aristotle:physics}: “Yet, however, we also say that all are \emph{virtutes} in having ‘toward-which’ [i.e., relation] in some way.” (my translation).} Aquinas is attributing to Aristotle’s ἀρετή the same meaning as Dionysius’ δύναμις.%
%
\footnote{See, for example, the commentary on this same passage in \cite[VII, lc.~6, 920]{st:phys}: “Ad hoc autem probandum assumit quandam propositionem, scilicet quod virtus sit perfectio quaedam. Quod quidem sic probat: quia unumquodque tunc est perfectum, quando pertingere potest ad propriam virtutem; sicut naturale corpus tunc perfectum est, quando potest aliud sibi simile facere, quod est virtus naturae. Quod etiam probat per hoc, quia tunc est aliquid maxime secundum naturam, quando naturae virtutem habet; virtus enim naturae est signum completionis naturae: cum autem aliquid habet complete suam naturam, tunc dicitur esse perfectum.”
This is not the \emph{virtus} of \emph{habitus}, but the more general sense that means “power.”  In Aristotle, unlike Dionysius, δύναμις ordinarily means \emph{potentia}.}
%
The intensity (\emph{intensio})  of an \emph{habitus}, Thomas argues, can be considered either as something that is similar to an extensive quantity, or else \emph{secundum participationem subiecti}. Evidently, Thomas is using the term \emph{intensio} broadly to mean roughly “quantity,” but the distinction he makes is the same as that above: using the example of knowledge, I can broaden the \emph{extension} of my knowledge, or increase its \emph{depth}. Differences in intensity, therefore, occur whenever there is a subject that can receive and participate in a perfection. Remove the subject, as we saw, and the perfection would be \emph{maxime}: this is precisely, however, the argument that Thomas uses in the \emph{quarta via} to prove the existence of \emph{Ipsum Esse}.

A confirmation is found in \emph{quaestio 42} of the prima pars, where Thomas, responding to the objection that the Son cannot be truly considered “equal” to the Father because God has no quantity, says,
%
\begin{quotation}
Ad primum ergo dicendum quod duplex est quantitas. Una scilicet quae dicitur quantitas molis, vel quantitas dimensiva, quae in solis rebus corporalibus est, unde in divinis personis locum non habet. Sed alia est quantitas virtutis, quae attenditur secundum perfectionem alicuius naturae vel formae, quae quidem quantitas designatur secundum quod dicitur aliquid magis vel minus calidum, inquantum est perfectius vel minus perfectum in caliditate. Huiusmodi autem quantitas virtualis attenditur primo quidem in radice, idest in ipsa perfectione formae vel naturae, et sic dicitur magnitudo spiritualis, sicut dicitur magnus calor propter suam intensionem et perfectionem. [\ldots] Secundo autem attenditur quantitas virtualis in effectibus formae. Primus autem effectus formae est esse, nam omnis res habet esse secundum suam formam. Secundus autem effectus est operatio, nam omne agens agit per suam formam. Attenditur igitur quantitas virtualis et secundum esse, et secundum operationem, secundum esse quidem, inquantum ea quae sunt perfectioris naturae, sunt maioris durationis; secundum operationem vero, inquantum ea quae sunt perfectioris naturae, sunt magis potentia ad agendum.\footcite[I, q.~42, a.~1, ad 1]{st:summa}
\end{quotation}
%
The first part is by now familiar: quantity can be dimensive (extensive) or virtual (intensive). Evidently God does not possess extensive quantity in any respect (least of all the quantity proper to material things). Virtual quantity, Thomas argues, can refer to \emph{actus secundus}, as in our example of the heat, or else it can refer to the effects of the form. It is difficult to determine in this passage whether by \emph{esse}  Thomas means \emph{esse ut actus} or the \emph{esse in actu} of the substance;%
%
\footnote{Both interpretations would fit the text, and so perhaps he is being deliberately ambiguous so as to include them both.}
%
however, since the \emph{esse substantiale} is in proportion to the virtuality of the \emph{esse ut actus}, what Thomas says is instructive in any case: “omnis res habet esse secundum suam formam”: it is the form, which is the active principle of the essence in material creatures and identical with the essence in pure spirits, that determines the intensity or \emph{virtus} of that \emph{esse}. Just as there are \emph{quantitates virtuales} proper to operation, \emph{esse} can also be considered such quantity. Thomas claims something similar in \emph{De potentia} q.~5, a.~4: “Nam quantum unicuique inest de forma, tantum inest ei de virtute essendi; unde et in I \emph{caeli et mundi} philosophus vult quod quaedam habeant virtutem et potentiam ut semper sint.”\,%
%
\footnote{\Cite[q.~5, a.~4, ad 1]{st:depotentia}. Thomas cites from \cite[I, 12, 281b 25-32]{aristotle:decaelo}: Ἅπαν ἄρα τὸ ἀεὶ ὂν ἁπλῶς ἄφθαρτον. Ὁμοίως δὲ καὶ ἀγένητον· εἰ γὰρ γενητόν, ἔσται δυνατὸν χρόνον τινὰ μὴ εἶναι — φθαρτὸν μὲν γάρ ἐστι τὸ πρότερον μὲν ὄν, νῦν δὲ μὴ ὂν ἢ ἐνδεχόμενόν ποτε ὕστερον μὴ εἶναι· γενητὸν δὲ ὃ ἐνδέχεται πρότερον μὴ εἶναι — ἀλλ᾿ οὐκ ἔστιν ἐν ᾧ χρόνῳ δυνατὸν τὸ ἀεὶ ὂν ὥστε μὴ εἶναι, οὔτ᾿ ἄπειρον οὔτε πεπερασμένον· καὶ γὰρ τὸν πεπερασμένον χρόνον δύναται εἶναι, εἴπερ καὶ τὸν ἄπειρον. “Therefore every eternal reality is plainly incorruptible. Similarly the ungenerated. For if [it were] generated, it is possible that it could not be during a certain time—for the perishable is what is before, but now is not, or what takes on non-\emph{esse} at some later time; the generated however is what takes on \emph{esse} later—but there is no possibility in time for an eternal being not to be as such, whether infinite or finite. For it has the power to be (δύναται εἶναι) for a finite time, if indeed it is infinite” (my translation). The key term δύναται εἶναι, which Thomas renders as \emph{virtus} or \emph{potentia essendi} (in the active sense, evidently).}
%
Aristotle in his work is arguing in essence that eternal beings are not subject to generation and corruption, and vice versa, precisely because they have δύναμις to sustain their being for all eternity. From this, Thomas concludes that inasmuch as a creature has form, that much it possesses the power to be (\emph{virtus essendi}).%
%
\footnote{Interestingly, E. Gilson in his work entitled “\emph{Virtus essendi}” essentially rejects the idea that \emph{esse} could be an intensive act. He says, “L’on ferait fausse route en cherchant dans saint Thomas une doctrine de l’être qui reconnaîtrait a l’\emph{esse} une intensité intrinsèque variable a laquelle correspondraient, dans la nature, les degrés différents de perfection qui distinguent les êtres. Le mouvement comporte des degrés de quantité qui permettent de le dire plus ou moins grand, l’être n’en a pas [\ldots]
% Pour l’imagination, une \emph{virtus}, une \emph{dunamis} est une force, et si on en parle comme de quelque chose qui peut être donné dans sa totalité, ou ne se rencontrer que sous forme de participation limitée, il est inévitable que nous l’imaginions comme une quantité variable. Le plus simple est de lui attribuer divers degrés d’intensité. C’est justement là que l’erreur d’interprétation guette le lecteur. Il convient de ne transposer les attributs du physique dans l’ordre du métaphysique.
Au delà de la nature il n’y a plus de matière, ni d’étendue, ni de quantité, ni de plus ou moins. L’\emph{esse} échappe a toutes ces déterminations, mais comme malgré tout il y a des différences d’être nous nous représentons des degrés de pureté et d’actualité formelle sous l’aspect de degrés d’intensité quantitative qui ne conviennent aucunement a l’être.” 

F. O’Rourke suggests that perhaps Gilson’s reluctance to attribute quantity to \emph{esse} arises because he does not distinguish well between intensive and dimensive (extensive) quantity \parencite[45]{orourke:virtus}. I would add that Gilson may also have missed that what permits intensive quantity is precisely communication of a perfection to a subject, not movement as such. Hence \emph{esse per participationem} qualifies as an intensive quantity without any difficulty.
In fact, Thomas addresses this very issue in \cite[I, q.~77, a.~6, ad 3]{st:summa}. See section \ref{sec:origin-of-powers}, below.}
%

\subsection{Systematic Presentation of \emph{Esse} as \emph{Virtus Essendi}}

Let us expose this doctrine more systematically: \emph{quantitas}, which refers primarily to the three-dimensional extension of material things and to the enumeration of discontinuous entities, is increased and decreased by “addition” and “subtraction.” This concept of quantity can be extended by analogy to other perfections. Some of these quantities be “added” and “subtracted” in a way similar to material quantity; such quantities can be terned “extensive” or “dimensive.” Other quantities, however, are non-additive. They exhibit the phenomenon of \emph{magis et minus}—that is, the perfection in question can suffer increase or decrease in intensity, or at least can be found in different degrees. However, simply “adding” together subjects containing these perfections does not result in their increase. Such quantities are termed “virtual” or “intensive.” In fact, if a perfection is found \emph{magis et minus} it can only be so because it is found in a \emph{subiectum} that participates to a greater or lesser extent in that perfection. Since only an agent \emph{in actu} can reduce a potency to act, that subject must receive the perfection from outside itself. If, however, the perfection were to be found independently of a subject, it would subsist \emph{maxime}, since act—which can only be limited by potency—is by nature “fecund” and “expansive.”  It follows that the presence of a \emph{quantitas virtualis} necessarily entails a participation.

The communication of this perfection could be mediate, in which a subject, upon receiving the perfection, can communicate it in turn to another subject: as, for example, in the case of heat. On the other hand, some perfections can be communicated only by an agent that possesses it \emph{per essentiam}. In either case, the perfections must reduce to an agent that has the perfection \emph{per essentiam} or as a \emph{proprium}.%
%
\footnote{For example, although there is in fact no absolutely \emph{maxime calidum}, heat must eventually come from some agent that produces heat on its own (such as a fire, a radiator, or an oven); that is, that possesses heat as a \emph{proprium}. Iron can only receive heat \emph{per participationem}. Unlike \emph{esse}, the heat of the iron can be re-transmitted; nevertheless, to keep the iron hot, it is eventually necessary to re-introduce it into the fire.}
%
The \emph{actus essendi}, since it is \emph{first} act and the source of all other perfections in a \emph{suppositum} and is not produced by the essence, must be communicated in an absolutely immediate way, as we demonstrated in our \emph{resolutio} above. Nevertheless, \emph{actus essendi} is indeed received into a potency; namely, the essence. Therefore, even though it is an \emph{actus primus}, not an \emph{actus secundus}, it is entirely appropriate to refer to it as a \emph{quantitas virtualis} that is capable of attaining different degrees of intensity. It follows that \emph{actus essendi} is rightly called \emph{virtus essendi}: it is not merely an \emph{existentia} that places an essence outside of its causes. Rather, it is a rich reality that is the source and principle of all perfections in a \emph{suppositum}, much more “powerful” than either the essence that measures it or the perfections in act that depend on it.

\begin{DONE}
  \begin{itemize}
    \item In general: The notion of \emph{virtus essendi} as a
          \emph{quantitas virtualis sive intensio}. Show that this is an
          appropriate model for 
    
    \item Quantity, in Aristotle and Thomas. Extensive versus intensive.
          Example of heat vs. temperature.

	\item Textually, prove that Saint Thomas used this concept to describe \emph{esse}.

	\item Concept from Pseudo-Dionysius: δύναμις vs. ἀρετή

    \item Reduction of all intrinsic actuality to \emph{esse ut actus}.

    \item Show how \emph{esse ut actus} is \enquote{prior} to the entire substance, hence is the \enquote{source} of all its actuality. It contains much more \enquote{power} than any of the \emph{esse in actu}.

    \item The \enquote{good} of a substance, hence its \enquote{happiness,} consists in \emph{actualizing} as much of that \enquote{virtus} as possible. (We will show that below.)

  \end{itemize}
\end{DONE}


\section{Conclusions Regarding \emph{Virtus Essendi}}

From the basic notions of \emph{bonum} and \emph{ens}, we have discovered that \emph{bonum} is founded on \emph{esse}, and that the \emph{esse} of \emph{ens} is rooted in a radical active principle called \emph{actus essendi} or \emph{esse ut actus} that is co-created together with a radical potential principle called \emph{essentia ut potentia essendi}. The \emph{esse ut actus} is not only the source of a \emph{suppositum}’s existence (although it is that too), but the source of every one of its proper perfections: \emph{actus essendi} is truly a \emph{virtus essendi}, the “storehouse” of all the perfections in a \emph{suppositum}. The \emph{actus essendi} contains these perfections in a precisely \emph{virtual} way: the perfections are not yet strictly \emph{in act} until the \emph{actus essendi} expands into the \emph{suppositum}’s various layers of \emph{esse in actu}. We saw that \emph{entia} are good to the degree that they are \emph{in actu} (specifically \emph{in actu secundo}), but that their \emph{actus secundus} is rooted in the radical act, which is \emph{actus essendi}. Not only does act render an \emph{ens} desirable as such, but creatures all seek their own perfection or “happiness.” The task before us in the next chapter is to show that this tendency toward perfection—which as we saw ultimately leads creatures to their ultimate final cause, God as the \emph{Summum Bonum}—results no the intrinsic level from the very structure of \emph{ens}: in short, we need to demonstrate that \emph{esse} as an intensive act measured by essence necessarily entails an \emph{ordo ad finem}, and that this \emph{ordo} is not only an extrinsic exigency, resulting from creatures’ tendency to return to their Creator, but an intrinsic exigency born of the very structure and dynamism of \emph{ens}.